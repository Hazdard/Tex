\documentclass[10pt]{beamer}
\usetheme{metropolis}           % Use metropolis theme
\usepackage[utf8]{inputenc}
\usepackage{amsmath,amsfonts,amssymb}
\usepackage{dsfont}
\usepackage{graphicx}
\usepackage{caption}
\usefonttheme[onlymath]{serif}


\title{Difféomorphismes du cercle et théorème de Nash-Moser}
\date{14 Mai 2024}
\author{Sacha Ben-Arous, Mathis Bordet}
\institute{ENS Paris-Saclay}
\begin{document}
  \maketitle
\begin{frame}
\tableofcontents
\end{frame}  



\section{Difféomorphismes du cercle}
\begin{frame}{Dynamique en dimension 1}
    On considère le cercle $\mathbb{S}^1 := \left\{z, |z|=1 \right\}$, ainsi que le plongement $\Pi : \mathbb{R} \mapsto \mathbb{S}^1, t \mapsto e^{2i\pi t}$
\end{frame}


\subsection*{Théorème d'Arnorld}
\begin{frame}
\frametitle{Cadre}
Prenons un relevé $F$ analytique de rotation $\alpha$ (et donc semi-conjugué à $R_\alpha$). Que l'on écrit :
\[ F(x) = x + \alpha + \eta (x) \]
avec $\eta$ analytique et "assez petit".

$\eta$ est également 1-périodique.
\end{frame}

\begin{frame}
\frametitle{Détermination}
Par théorème de semi-conjugaison, on a :
\[ F \circ H(x) = H( x + \alpha) \text{ et on cherche } H \text{ de la forme } H = \text{id} + U \]
\[ \Rightarrow U(x+ \alpha) - U(x) = \eta (x+U(x)) \text{ simplifié en } U(x+ \alpha) - U(x) = \eta (x) \]
On choisit de chercher \( U \) 1-périodique. On a que 
\[ (\exp(2 i \pi n \alpha)-1)\widehat{U} (n) = \widehat{\eta} (n) \]
\end{frame}

\begin{frame}
\frametitle{Perte de régularité et Petit diviseur}
Dans quelle mesure \( \widehat{U} \) est-elle liée à une série de Fourier convergente ?

En effet, \( \exp(2 i \pi n \alpha)-1 \) peut arbitrairement s'approcher de 0.

Petit diviseur avec \( \alpha \) irrationnel: \( \left\| \alpha - \frac{m}{n} \right\| \geq \frac{k}{n^\nu} \Rightarrow \left\| \exp(2 i \pi n \alpha)-1\right\| \geq \frac{4k}{n^{\nu-1}} \)

Induit la perte de régularité suivante
\[
\begin{aligned}
& \|F\|{H{\text{per}}^s} = \left(\sum_{n \in \mathbb{Z}}|n|^{2 s}|\hat{F}(n)|^2\right)^{1 / 2}, \quad s \geq 0 \\
& \|U\|{H{\text{per}}^s} \leq \frac{1}{4 K}\|\eta\|{H{\text{per}}^{s+\nu-1}} .
\end{aligned}
\]
\end{frame}

\begin{frame}
\frametitle{Enoncé et Preuve}
Le théorème d'Arnold affirme donc que si \( F \) a un relevé \( F(x) = x + \alpha + \eta (x) \) avec \( \eta \) analytique et assez petit (au sens d'une norme analytique), alors \( F \) est analytiquement conjugué à \( R_\alpha \).

Éléments de démonstration :
\begin{itemize}
    \item \( U_n \) la solution de \( U_n(x+\alpha)-U_n(x)=\eta_n(x)-\hat{\eta}_n(0) \).
    \item \( H_n(x)=x+U_n(x) \).
    \item \( F_{n+1}=H_n^{-1} \circ F_n \circ H_n=\left(H_1 \circ \cdots \circ H_n\right)^{-1} \circ F \circ\left(H_1 \circ \cdots \circ H_n\right) \).
\end{itemize}
\end{frame}
\section{Théorème de Nash Moser}
\begin{frame}
\frametitle{Cadre}
On considère ici deux suites d'espaces de Banach ${E}\sigma, \lVert \cdot \rVert\sigma$ et ${F}\sigma, \lVert \cdot \rVert\sigma$.

De telle sorte qu'il existe une fonction régularisante $S$ telle que 
\[
\forall \theta \in \mathbb{R}, \quad S : E \to F 
\]

\begin{enumerate}
    \item $\left\|S_\theta u\right\|_b \leq C\|u\|_a$, si $b \leq a$
    \item $\left\|S_\theta u\right\|_b \leq C \theta^{b-a}\|u\|_a$, si $a < b$
    \item $\left\|u-S_\theta u\right\|_b \leq C \theta^{b-a}\|u\|_a$, si $a > b$
    \item $\left\|\frac{d}{d \theta} S_\theta u\right\|_b \leq C \theta^{b-a-1}\|u\|_a$.
\end{enumerate}
\end{frame}

\begin{frame}
\frametitle{Enoncé }
Soit $a_2 \in \mathbf{R}$ et soit $\alpha , \beta \in [0;a_2 ]$.
De plus, considérons une application $\Phi : E_\alpha \to F_\beta \: C^2 $ vérifiant :
\[ \left\|\Phi '' (u)(v,w) \right\|{\beta + \delta} \leq C ( 1 +\left\|u \right\|\alpha ) \left\|w \right\|{\alpha - \frac{\epsilon}{2} } \cdot \left\| v \right\|{\alpha - \frac{\epsilon}{2}} \]  
On a de plus l'existence d'une inverse à droite pour $\Phi' $, c'est-à-dire :
$\forall v \in E_\infty$, on a $\Psi (v): f_\infty \to E_\infty $ avec 
\[ \left\| \Psi(v)g \right\|a \leq C \left\| g \right\|{\beta + a - \alpha }+ \left\| g \right\|0 \left\| v \right\|{\alpha + \beta} \]
Alors, $\exists \eta > 0 $ telle que $\forall f \in F_\beta$ vérifiant $ \left\| f \right\|\beta \leq \eta$, alors $\exists u \in E\alpha$ vérifiant $ \Phi(u)-\Phi(0)=f$.
\end{frame}

\begin{frame}
\frametitle{Outil}
On prend $\theta_j$ une suite d'indices divergents et on définit $\Delta_j= \theta_{j+1} - \theta_{j}$ et $R_j u=\left(S_{\theta_{j+1}} u-S_{\theta_j} u\right) / \Delta_j$ si $j>0$, $R_0 u=S_{\theta_1} u / \Delta_0$.

On obtient alors :
\[ u=\sum_{j=0}^{\infty} \Delta_j R_j u \]
Convergente dans $E_a$ si $u \in E_b$ et $a < b$.
\end{frame}

\begin{frame}
\frametitle{Shéma de la preuve:}
On construit les suites suivantes en prenant g $\in F_\beta$
$$
g=\sum \Delta_j g_j ; \quad\left\|g_j\right\|b \leqq C_b \theta_j^{b-\beta-1}\|g\|\beta^{\prime} .
$$
$$
u_{j+1}=u_j+\Delta_j \dot{u}j, \quad \dot{u}_j=\psi\left(v_j\right) g_j, \quad v_j=S{\theta_j} u_j
$$
et on montre que $\Phi(u) -\Phi(0) = T(g) + g$ avec T application continue . 
\end{frame}

\end{document}

