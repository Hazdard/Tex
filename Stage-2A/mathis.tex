\documentclass{article}
\usepackage{amsmath, amsfonts, amssymb}
\usepackage{../jedusor}
\title{Décomposition de Littlewood-Paley}
\author{Mathis Bordet}
\date{Mai 2024}

\begin{document}

\maketitle 

\section{Décomposition de Littlewood-Paley}
Nous allons dans cette section présenter la décomposition de Littlewood-Paley. C'est une décomposition de fonction dans laquelle chaque terme a un support de fréquence (au sens de la transformée de Fourier) localisé. Nous allons également présenter des propriétés sur cette décomposition.

Il nous est cependant indispensable de nous intéresser à l'existence et, plus précisément, à la construction d'un type de fonctions que sont les fonctions $\mathbb{C}^\infty(\mathbb{R}^d)$ avec un support au voisinage de 0 mais également constantes au voisinage de 0.

Intéressons-nous au cas $d=1$ et posons :
\[
g: \mathbb{R}^+ \to \mathbb{R}
\]
\[
g(x) = 
\begin{cases} 
1 & \text{si } 0 \leq x \leq \frac{1}{2} \\
\exp\left(-\frac{1}{\frac{1}{4} - |x - \frac{1}{2}|^2}\right) & \text{si } \frac{1}{2} \leq x \leq 1 \\
0 & \text{sinon}
\end{cases}
\]
On a que $g$ appartient à $\mathbb{C}^\infty(\mathbb{R})$ puisque, en effet, on peut montrer par récurrence que :
\[
\forall n \in \mathbb{N}^*, \ \forall x \in \left[\frac{1}{2}, 1\right], \ g^{(n)}(x) = x Q_n(|x|) \exp\left(-\frac{1}{\frac{1}{4} - |x - \frac{1}{2}|^2}\right) \: (1)
\]
avec $Q_n$ un fraction rationnelle dont le pole ce situe en 1 , ce qui montre la continuité des $(g^{(n)})_{n \in \mathbb{N}}$.

On étend alors cette construction à $\mathbb{R}^d$ en notant $\Psi(x)=g(|x|)$. $\Psi$ est une fonction $\mathbb{C}^\infty(\mathbb{R}^d)$ avec $\text{supp}(\Psi) \subset B(0,1)$ et égale à 1 sur $B(0,\frac{1}{2})$.

On obtient donc, en posant $\chi(x)=\psi(x)-\psi\left(\frac{x}{2}\right)$ :
\[
\forall \xi \in \mathbb{R}^d, \ 1= \psi(\xi) + \sum_{p=0}^\infty \chi(2^{-p} \xi) \: (2)
\]
avec $\text{supp}(\chi(2^{-p} \cdot)) \subset C(0,2^{p-1},2^{p+1})$

\begin{lemma}[Lemme I-?]
Soit $\phi \in \mathbb{S}$ alors :
\[
\hat{\phi} = \psi \hat{\phi} + \sum_{p=0}^\infty \chi(2^{-p} \cdot) \hat{\phi}
\]
\end{lemma}
\begin{proof}[demonstrations:]
Soit g $\in \mathbb{S}$ : \\
Montrons que pout tout $\alpha \ \text{et} \ \beta \in \mathbb{N}\times \mathbb{N}^d $
$$\left\lVert x^\alpha \partial^\beta(g-\psi(2^{-p}x)g) \right\rVert_\infty \to_{p\to \infty} 0$$
On considère alors:
\begin{align*}
    \left\lVert x^\alpha \partial^\beta(g-\psi(2^{-p}x)g)\right\rVert_\infty &= \left\lVert x^\alpha \partial^\beta(g-\psi(2^{-p}x)g)\right\rVert_{[2^{p-1};2^{p+1}]} \\
    &\leq \left\lVert x^\alpha \partial^\beta g\right\rVert_{[2^{p};2^{p+1}]} + \left\lVert x^\alpha \partial^\beta g(1-\psi(2^{-p}x)\right\rVert_{[2^{p-1};2^{p}]}
\end{align*}

De plus puisque g $\in \mathbb{S}$ alors $\lim_{|x| \to \infty} x^\gamma g(x)=0$.On a alors $\left\lVert x^\alpha \partial^\beta g\right\rVert_{[2^{p};2^{p+1}]}$ tend bien vers 0 lorsque p tend vers $+\infty$. En utilisant la formule de derivation de Leibniz sachant que $\partial^k \psi = O(x^{k}$ (en utilisant (1)) on a que $\left\lVert x^\alpha \partial^\beta g\right\rVert_{[2^{p};2^{p+1}]}$ tend bien vers 0 lorsque p tend vers $+\infty$. On utilise ensuive la continuité de la trnasformé de fourrier et de la transformé de Fourrier inverse sur $\mathbb{S}$.


\end{proof}

\begin{defin}[Définition]
On définit les opérateurs de la décomposition de Littlewood-Paley de la manière suivante :
\[
\forall q \in \mathbb{R_+}, \ \forall u \in L^q, \quad \Delta_{-1} u = \mathcal{F}^{-1}(\psi) \ast u, \quad \Delta_{p} u = 2^{pd} \mathcal{F}^{-1}(\chi(2^p \cdot)) \ast u \text{ pour } p \geq 0
\]
\end{defin}

\begin{prop}[Propriété]
Soit $u \in \mathbb{S}'$, en posant :
\[
S_p u = \sum_{k=1}^{p-1} \Delta_k u
\]
On a que :
\[
\lim_{p \to \infty} S_p u = u
\]
\end{prop}

\begin{proof}[démonstration]
Prenons $u \in \mathbb{S}'$ et $v \in \mathbb{S}$.\\

\[
\langle \mathcal{F}(S_n u), v \rangle = \langle \psi(2^{-p}\xi) \mathcal{F}(u), v \rangle = \langle \mathcal{F}(u), \psi(2^{-p}\xi) v \rangle
\]
\end{proof}
Or, $\lim_{p \to +\infty} \psi(2^{-p}\xi)v = v$ dans $\mathbb{S}(\mathbb{R}^d)$ par le lemme ?. On obtient donc :

\[
\mathcal{F}(S_n u) \to \mathcal{F}(u) \quad \text{dans} \ \mathbb{S}
\]

Par continuité de $\mathcal{F}^{-1}$, on a en effet $S_p u = \sum_{k=1}^{p-1} \Delta_k u$.
     
     

\begin{lemma}[Lemme]\label{young}
Il existe $C > 0$ tel que pour tout $1 \leq p \leq +\infty$ et $u \in L^p(\mathbb{R}^d)$,
\begin{align*}
\sup_{k \geq -1} \|S_k u\|_{L^p} &\leq C \|u\|_{L^p}, & \sup_{k \geq -1} \|\Delta_k u\|_{L^p} &\leq C \|u\|_{L^p}
\end{align*}
\end{lemma}
\begin{proof}
On écrit $S_k u = 2^{pd} \mathcal{F}^{-1}(\psi(2^p\cdot))\ast u$. Par inégalité de Young on obtient :
$$\left\lVert S_k u \right\rVert_{L^p} \leq \left\lVert u \right\rVert_{L^p} \left\lVert 2^{pd} \mathcal{F}^{-1}(\psi(2^p\cdot)  \right\rVert_{L^1}$$ 
De même pour $\|\Delta_k u\|_{L^p}$ 

\end{proof}

\begin{lemma}[Presque-orthogonalité]
Pour tout $u \in L^2(\mathbb{R}^d)$,
\begin{equation}\label{qortho}
\sum_{k \geq -1} \|\Delta_k u \|^2_{L^2} \leq \|u\|^2_{L^2} \leq 2 \sum_{k \geq -1} \|\Delta_k u \|^2_{L^2}
\end{equation}
\end{lemma}
\begin{proof}[démonstration]
On pars de $1= \psi(\xi) + \sum_{p=0}^\infty \chi(2^{-p} \xi)$.
Seule deux de ces fonctions on une intersection de support non vide. On utilise alors : $a^2 +b^2 \leq (a+b)^2 \leq 2(a^2 +b^2)$ on obitient alors:
$$ \frac{1}{2} \leq \psi(\xi)^2 + \sum_{p=0}^\infty \chi(2^{-p} \xi)^2 \leq 1$$
On obtient alors al segonde inégalité en multipliant la ligne en aux par û et en utilisantl'identité de  plancherelle.
\end{proof}
\end{document}
