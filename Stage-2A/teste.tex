\documentclass[11pt,a4paper]{article}

\usepackage{../jedusor}
\usepackage[maxalphanames=99, maxnames=99, backend=bibtex, style=alphabetic, sorting=ynt]{biblatex}

\begin{document}

\subsection{Cadre théorique}

\paragraph{Espace :}
On considère ici deux suites d'espaces de Banach $E_\sigma, \lVert \cdot \rVert_\sigma$ et $F_\sigma, \lVert \cdot \rVert_\sigma$.

De telle sorte qu'il existe une fonction régularisante $S$ telle que 
\[
\forall \theta \in \mathbb{R}, \quad S : E_0 \to E_\infty 
\]

\begin{equation}\label{0.1}
\left\|S_\theta u\right\|_b \leq C\|u\|_a \ \text{,si}  \:  b \leq a 
\end{equation}
\begin{equation}\label{0.2}
\left\|S_\theta u\right\|_b \leq C \theta^{b-a}\|u\|_a  \ \text{,si}  \: a < b
\end{equation}
\begin{equation}\label{0.3}
\left\|u-S_\theta u\right\|_b \leq C \theta^{b-a}\|u\|_a \ \text{,si}  \:a > b
\end{equation}
\begin{equation}\label{0.4}
\left\|\frac{d}{d \theta} S_\theta u\right\|_b \leq C \theta^{b-a-1}\|u\|_a 
\end{equation}


\paragraph{Outil :}
On prend $\theta_j$ une suite d'indices divergents et on définit $\Delta_j = \theta_{j+1} - \theta_j$ et $R_j u = \left(S_{\theta_{j+1}} u - S_{\theta_j} u\right) / \Delta_j$ si $j > 0$, $R_0 u = S_{\theta_1} u / \Delta_0$.

On obtient alors :
\[ 
u = \sum_{j=0}^{\infty} \Delta_j R_j u 
\]
Convergente dans $E_a$ si $u \in E_b$ et $a < b$.

Avec de plus :
\begin{equation}\label{5} 
\left\|R_j u\right\|_b \leq C_{a, b} \theta_j^{b-a-1}\|u\|_a \quad (\text{en utilisant } (\ref{0.4}))
\end{equation}
\paragraph{Espace faible :}
On généralise l'approche ci-dessus en définissant l'ensemble $E'_b$ : l'ensemble des sommes $\sum \Delta_j u_j$ convergentes dans $E_b$ et satisfaisant 
\[ 
\left\|u_j \right\|_b \leq M \theta_j^{b-a-1} \quad \forall a \in [0,b] 
\]  
On munit cet espace de la norme $\left\|. \right\|'_b$ l'infimum des constantes $M$ respectant l'inégalité précédente.
On a de manière immédiate que $\left\|. \right\|'_a > \left\|. \right\|_b$ si $b < a$ et par ($\ref{0.2}$) on a que $\left\|. \right\|'_a \leq C \left\|. \right\|_a$.

\subsection*{Théorème de Nash-Moser}

\paragraph{Énoncé :}
Soit $a_2 \in \mathbb{R}$ et soit $\alpha, \beta \in [0, a_2]$.
De plus, considérons une application $\Phi : E_\alpha \to F_\beta \: C^2$ vérifiant :
\begin{equation}\label{phi''} 
\left\|\Phi''(u)(v,w) \right\|_{\beta + \delta} \leq C (1 + \left\|u \right\|_\alpha) \left\|w \right\|_{\alpha - \frac{\epsilon}{2}} \cdot \left\|v \right\|_{\alpha - \frac{\epsilon}{2}} 
\end{equation}  
On a de plus l'existence d'une inverse à droite pour $\Phi'$, c'est-à-dire :
\[
\forall v \in E_\infty, \quad \Psi(v) : F_\infty \to E_{a_2}
\]
avec 
\begin{equation}\label{estimation psi}
\left\| \Psi(v)g \right\|_a \leq C \left\| g \right\|_{\beta + a - \alpha} + \left\| g \right\|_0 \left\| v \right\|_{\alpha + \beta} 
\end{equation}
Alors, $\exists \eta > 0$ telle que $\forall f \in F_\beta'$ vérifiant $ \left\| f \right\|_\beta \leq \eta$, il existe $u \in E'_\alpha$ vérifiant $\Phi(u) - \Phi(0) = f$.

\paragraph{Démonstration :}
Cette démonstration s'appuie sur un schéma de Newton légèrement altéré pour prendre en compte le manque de 'régularité' de l'inverse de la dérivée. En effet, ce schéma calcule uniquement une solution approchée de l'équation :
\[  
\Phi^{\prime}\left(u_n\right)\left(u_{n+1} - u_n\right) + \Phi\left(u_n\right) = 0 
\]
et compte sur la vitesse du schéma de Newton pour combler l'erreur commise.

Prenons $g \in F'_\beta$
\[ 
g = \sum \Delta_j g_j 
\]
On a avec ($\ref{5}$)
\begin{equation}\label{gj}   
\left\|g_j\right\|_b \leq C_b \theta_j^{b-\beta-1}\|g\|_\beta'
\end{equation} 
On construit la suite $u_j$ par récurrence de la manière suivante:
\[ 
u_{j+1} = u_j + \Delta_j \dot{u}_j, \quad \dot{u}_j = \Psi\left(v_j\right) g_j, \quad v_j = S_{\theta_j} u_j
\]
On prendra $u_0 = 0$ et $\theta_i = 2^i$.

\paragraph{Récurrence}
Nous allons démontrer par récurrence les trois inégalités suivantes:
\begin{equation} \label{a}
\; \left\|\dot{u}_j \right\|_a \leq C_1 \theta^{a-\alpha-1}_j \left\|g \right\|_\beta' \; , \; a \leq a_2 
\end{equation} 
\begin{equation} \label{b}
\left\| v_j \right\|_a \leq C_2 \theta^{a-\alpha-}_j \left\|g \right\|_\beta' \; , \; \alpha < a \leq a_2  
\end{equation}
\begin{equation}\label{c} 
\left\|u_j - v_j \right\|_a \leq C_3 \theta^{a-\alpha-}_j \left\|g \right\|_\beta' \; , \; \alpha < a \leq a_2  
\end{equation}
L'initialisation est trivialement vérifiée.

On suppose maintenant les inégalités ($\ref{a}$), ($\ref{c}$) au rang j-1 et l'inégalité ($\ref{c}$) au rang j.
\[
\begin{aligned}
\left\|\dot{u}_j \right\|_a = \left\|\Psi(v_j)g_j \right\|_a &\leq C (\left\|g_j \right\|_{\beta + a - \alpha} + \left\|g_j \right\|_0 \left\|v_j \right\|_{\beta + a}) \: \text{en utilisant}  \ \ref{estimation psi} \\
&\leq C( C_b \left\|g \right\|_\beta' \theta^{a-\alpha-1}_j + \theta^{-\beta-1}_j \theta^{\beta+a-1}_j {\left\|g \right\|_\beta'}^2 \; \text{en utilisant l'inégalité } \ (\ref{gj}) \text{ et} \ (\ref{a}) \ \text{au rang j}) \\
&\leq C (C_b + \left\|g \right\|_\beta' C_2) \theta^{a-\alpha-1}_j \left\|g \right\|_\beta'
\end{aligned}
\]
Si on prend $C_1 > C (C_b + \left\|g \right\|_\beta' C_2)$ on a ($\ref{a}$) au rang j.
On va montrer $\ref{b}$ et $\ref{c}$, pour cela intéressons-nous à la quantité $u_{j+1}$:
\[
\begin{aligned}
\left\|u_{j+1} \right\|_a &\leq \sum^j_{i=1} \left\|\Delta_i \dot{u}_i  \right\|_a \\ 
&\leq \sum^j_{i=1} |\Delta_i| C_1 \left\|g \right\|_\beta' \theta^{a-\alpha-1}_i \; \text{en utilisant} \ (\ref{a}) \ \text{jusqu'au rang j} \\
&\leq C \left\|g \right\|_\beta' \theta^{a-\alpha}_i \; \text{pour } a \geq \alpha
\end{aligned}
\]
On obtient donc avec ($\ref{0.1}$) ,($\ref{b}$) et ($\ref{c}$).

\paragraph{Convergence}

1) Convergence de $u_j$
On a avec $\ref{a}$ que $u = \sum \Delta_j u_j$ appartient à $E_\alpha'$ et que $\left\|u \right\|_\alpha' \leq C \left\|g \right\|_\beta'$

2) Limite de $\Phi(u_j)$
Par continuité on a que 
\[
\Phi(u) - \Phi(0) = \sum^{+\infty}_{j=0} (\Phi(u_{j+1}) - \Phi(u_{j})) 
\]
\[
\begin{aligned}
\Phi(u_{j+1}) - \Phi(u_j) &= (\Phi(u_j + \Delta_j \dot{u}_j) - \Phi(u_j) - \Phi'(u_j)\Delta_j \dot{u}_j) + (\Phi'(u_j) - \Phi'(u_j))\Delta_j \dot{u}_j + \Delta_j g_j \\
&= \Delta_j e_j + \Delta_j e'_j + \Delta_j g_j
\end{aligned}
\]

Par inégalité de Taylor-Lagrange et en utilisant ($\ref{phi''}$)
\[
\begin{aligned}
\left\|e_j \right\|_{\beta + \delta} &\leq C \Delta_j {\left\|\dot{u}_j \right\|_{\alpha - \frac{\epsilon}{2}}}^2 \\
&\leq C \left\|g \right\|_\beta'^2 \Delta_j \theta_j^{-\epsilon-2} \; \text{(en utilisant } (\ref{a}) \\
&\leq C \left\|g \right\|_\beta'^2 \theta_j^{-\epsilon-1}
\end{aligned}
\]

\[
e_j = \int^1_0 \Phi''(v_j + t(u_j - v_j))(\dot{u}_j, u_j - v_j) \, dt
\]

\[
\begin{aligned}
\left\|e'_j \right\|_{\beta + \delta} &\leq C (1 + \left\|v_j + t(u_j - v_j) \right\|_\alpha) \left\|\dot{u}_j \right\|_{\alpha - \frac{\epsilon}{2}} \cdot \left\|u_j - v_j\right\|_{\alpha - \frac{\epsilon}{2}} \\
&\leq C' \left\|\dot{u}_j \right\|_{\alpha - \frac{\epsilon}{2}} \cdot \left\|u_j - v_j\right\|_{\alpha - \frac{\epsilon}{2}} \\
&\leq \left\|g \right\|_\beta' \theta_j^{-\frac{\epsilon}{2}-1} \left\|u_j - v_j\right\|_{\alpha - \frac{\epsilon}{2}} \; \text{(en utilisant)} \ \ref{a}
\end{aligned}
\]
On peut utiliser ($\ref{c}$) pour $|\left\|u_j - v_j\right\|_{\alpha + \eta}$ avec $\eta < \frac{\epsilon}{2}$.

On obtient donc 
\[ 
\sum \Delta_j \left\|e_j'\right\|_\beta' \leq C \left\|g \right\|_\beta'^2 
\]

\paragraph{Conclusion}
En notant $T(g) = \sum \Delta_j e_j + \Delta_j e'_j$ et avec $\left\|T(g) \right\|_\beta' \leq C \left\|g \right\|_\beta'^2$, on peut utiliser le théorème de point fixe de Schauder (pour $T(g) + y$ avec $y \in F_\beta'$) pour montrer la local surjectivité de $T(g) + g$. Puis :
\[ 
\Phi(u) - \Phi(0) = T(g) + g
\]
on conclut bien.
\end{document}