\documentclass{article}
\usepackage{../jedusor}

\title{Preuve du théorème Krylov-Bogoliubov}
\author{Sacha Ben-Arous, Mathis Bordet}
\date{}

\begin{document}

\maketitle

\section{Énoncé du cas particulier}

On s'intéresse ici à la démonstration du théorème de Krylov-Bogoliubov, c'est-à-dire à l'existence d'une mesure invariante pour un homéomorphisme du cercle. On admet deux théorème issus du cours de Patrick Bernard sur les mesures sur un espace métrique compact, que l'on rappelle en fin de document.

\section{Démonstration}

\paragraph{Cadre}
On considère $T$ un homéomorphisme du cercle. Du fait que l'ensemble $\mathbb{S}_1$ est compact, par le théorème ?? on a que $C^0(\mathbb{S}_1,\mathbb{S}_1)$ est séparable, i.e. il existe une sous-famille $F$ dénombrable et dense au sens de la norme uniforme.

\paragraph{Construction de la suite}
Prenons $x \in \mathbb{S}_1$.
Pour tout $f \in C^0(\mathbb{S}_1,\mathbb{S}_1)$, posons :
\[
{S^N}f (x)= \frac{1}{N} \sum_{n=0}^N f(T^n(x))
\]
Par extraction diagonale et par compacité de $\mathbb{S}_1$, il existe une extraction $N_k$ telle que $\forall f \in \mathbb{F}, \: {S^{N_k}}_f (x)$ converge. On note sa limite $S_f (x)$.

\paragraph{Extension de la construction}
Soit $g \in C^0(\mathbb{S}1,\mathbb{S}_1)$. Prenons une suite $(f_n)_{n \in \mathbb{N}} \in F^\mathbb{N}$ convergeant uniformément vers $g$.
Montrons que $S_g (x)$ a un sens. Pour cela, considérons $(p, q) \in \mathbb{N}^2$ :

\[
\begin{aligned}
|{S^{N_p}}g (x) - {S^{N_q}}_g (x)| &= \left| {S^{N_p}}_g (x) - {S^{N_p}}{f_n} (x) + {S^{N_p}}{f_n} (x) - {S^{N_q}}{f_n} (x) + {S^{N_q}}_{f_n} (x) - {S^{N_q}}_g (x) \right| \\
&\leq \left| {S^{N_p}}g (x) - {S^{N_p}}{f_n} (x) \right| + \left| {S^{N_p}}{f_n} (x) - {S^{N_q}}{f_n} (x) \right| + \left| {S^{N_q}}_{f_n} (x) - {S^{N_q}}_g (x) \right| \\
&\leq 2 \left\| g - f_n \right\|_\infty + \left| {S^{N_p}}{f_n} (x) - {S^{N_q}}_{f_n} (x) \right|
\end{aligned}
\]
Donc ${S^{N_p}}_g (x)$ est une suite de Cauchy dans $\mathbb{S}_1$, donc elle converge.

\paragraph{Conclusion}
On pose maintenant 
\[
\begin{aligned}
&L_x: C^0(\mathbb{S}_1,\mathbb{S}_1) \to \mathbb{C} \\
&g \mapsto S_g(x)
\end{aligned}
\]
$L_x$ est 1-lipschitzienne donc continue. D'après le Théorème 2, il existe alors une mesure borélienne $\mu$ vérifiant :
\[
\forall g \in C^0(\mathbb{S}_1), \: \: L_x(g) = \int_{\mathbb{S}_1} g \, d\mu
\]
Or, par construction, $L_x(g \circ T) = L_x(g)$, donc $\mu = \mu(T^{-1}(.))$. Avec $T$ bijectif, on a bien $\mu = \mu(T(.))$.

\section*{Appendice}
\begin{mthm}{1}\label{foo}
Pour tout métrique compacte $X$, l'espace $C(X)$ est séparable pour la norme uniforme.
\end{mthm}

\begin{mthm}{2}\label{foo}
Il y a une bijection entre les mesures boréliennes et les formes linéaires positives sur $C^0(\mathbb{S}_1,\mathbb{S}_1)$.
\end{mthm}

\end{document}