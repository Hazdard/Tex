\documentclass[11pt,a4paper]{article}
\textheight245mm
\textwidth170mm
\hoffset-21mm
\voffset-15mm
\parindent0pt
\usepackage[utf8]{inputenc}
\usepackage{dsfont}
\usepackage{graphicx}
\usepackage{caption}
\usepackage{fancyhdr}
\usepackage{amsmath,amsfonts,amssymb}
\usepackage[french]{babel}
\usepackage[maxalphanames=99, maxnames=99, backend=bibtex, style=alphabetic, sorting=ynt]{biblatex}
\addbibresource{cex-denjoy.bib}
\usepackage[hidelinks]{hyperref} 
\hypersetup{
  colorlinks   = true,    % Colours links instead of ugly boxes
  urlcolor     = blue,    % Colour for external hyperlinks
  linkcolor    = black,    % Colour of internal links
  citecolor    = black      % Colour of citations
}
\usepackage{../zephyr}
\pagestyle{fancy}

\usepackage{array,multirow,makecell}
\setcellgapes{4pt}
\makegapedcells
\newcolumntype{R}[1]{>{\raggedleft\arraybackslash }b{#1}}
\newcolumntype{L}[1]{>{\raggedright\arraybackslash }b{#1}}
\newcolumntype{C}[1]{>{\centering\arraybackslash }b{#1}}

\renewcommand{\headrulewidth}{1pt}
\fancyhead[C]{}
\fancyhead[L]{L3 - 2023/2024}
\fancyhead[R]{D.E.R Mathématiques}

\renewcommand{\footrulewidth}{1pt}
\fancyfoot[C]{\thepage} 
\fancyfoot[L]{Sacha Ben-Arous}
\fancyfoot[R]{E.N.S Paris-Saclay}

\title{\textbf{Contre-exemples au théorème de Denjoy }}
\date{}


\begin{document}
\maketitle

\ \ \ \ \ Ce document est consacré à l'étude de certains contre-exemples du théorème de Denjoy. Dans une première partie nous construirons un contre-exemple $\mathcal{C}^1$ élémentaire proposé par Denjoy lui-même, et qui est développé par H. Rosenberg dans \cite{rosenberg}. Nous présenterons ensuite un résultat sur l'existence et la densité de contre-exemples de régularité $\mathcal{C}^{2-\epsilon}$ proposé par Michael Herman dans sa thèse \cite{herman}, en particulier au chapitre X.

\section{Cas continûment dérivable}

On se propose de montrer le théorème suivant :

\begin{theorem}\textbf{(Denjoy) : }
Pour tout $\alpha \in \mathbb{R} \setminus \mathbb{Q}$, il existe un $\mathcal{C}^1$-difféomorphisme $f$ du tore tel que $\rho(f)=\alpha$ et $f$ n'est pas conjugué à la rotation $R_\alpha$.
\end{theorem}

\ \ \ \ \ Pour faire cela, on s'inspire de la preuve du théorème dans le cas $\mathcal{C}^2$ : on sait déjà par Poincaré qu'un tel $f$ est semi-conjugué à $R_\alpha$ par une fonction continue croissante $h$ du tore. La preuve procède par l'absurde, en montrant que si $h$ n'est pas inversible, alors elle est constante sur un intervalle $I$ non trivial du tore. Les itérés $f^n(I)$ sont alors disjoints (on dit que $I$ est un intervalle errant pour la dynamique de $f$), et on conclut en montrant que la somme de leurs tailles diverge. \\ 
L'idée ici est donc construire une fonction $f$ qui admet un intervalle errant mais dont la suite des tailles des itérés est sommable. \\

On se servira de l'invariant de conjuguaison suivant : 

\begin{defin} : Un homéomorphisme du tore $f$ est dit \textit{minimal} si tout ensemble fermé invariant par  $f$ est vide ou égal au tore tout entier. 
\end{defin}

\begin{proposition} ~ 
\begin{enumerate}
\item Si $f$ est conjugué à $g$ minimal, alors $f$ est minimal.
\item Si $\alpha \in \mathbb{R} \setminus \mathbb{Q}$, alors $R_\alpha$ est minimal
\end{enumerate}
\end{proposition}

\textbf{Preuve :} \\ 
(1) est immédiat en utilisant la bijectivité de la conjuguaison. \\
(2) s'obtient en remarquand que si $\alpha \in \mathbb{R} \setminus \mathbb{Q}$, alors $\alpha\mathbb{Z} + \mathbb{Z}$ est dense dans $\mathbb{R}$, et donc $\alpha\mathbb{Z}/\mathbb{Z}$ est dense dans le tore $\mathbb{R}/\mathbb{Z}$. \\ \qed

La proposition suivante sera utile dans la suite :

\begin{proposition}
Soient $D_1,D_2$ deux ensembles denses dans $[0,1]$, et $f : D_1 \to D_2$ une surjection (strictement) croissante, alors $f$ admet un unique prolongement continu, (strictement) croissant, de $[0,1]$ dans lui-même.
\end{proposition}

\textbf{Preuve :} L'unicité du prolongement est immédiate par densité de $D_1$, car deux fonctions continues sur un sous-ensemble dense sont partout égales. Si $x\in [0,1]\setminus D_1$, il existe $(x_n)_{n\in \mathbb{N}}$ dans $D_1$ qui tend vers $x$, que l'on peut supposer monotone. La suite $(f(x_n))_{n\in\mathbb{N}}$ est alors monotone bornée, donc converge vers une limite qui définit la valeur de $f(x)$.
 % Cette définition ne dépend pas du choix de la suite, car si l'on se donne une autre suite $(\tilde{x}_n)_{n\in\mathbb{N}}$ ayant les mêmes propriétés, on remarque que la suite de terme général $f(x_n)-f(\tilde{x}_n)$ est de Cauchy.% 
Cette construction préserve clairement la monotonie (stricte) de $f$, et est de plus continue, car l'image monotone non continue d'un intervalle évite un intervalle, or l'image de $f$ est dense. \\ \qed
\\

\textbf{Preuve du Théorème 1.1 :} \\

On commence par construire un relèvement de $f$, et plus précisemment la dérivée de ce dernier : on note $(l_n)_{n\in \mathbb{Z}}$ la suite des longueurs des intervalles, telle que $\displaystyle \sum_{n\in \mathbb{Z}} l_n = 1$ et $\displaystyle \lim\limits_{n \to \pm \infty } \frac{l_{n+1}}{l_n} = 1 $. On peut par exemple prendre $l_n = \displaystyle \frac{c}{n^2 + 1}$ où $c$ est bien choisie. \\
On note alors $ \alpha_n := \alpha n \mod 1$ et on pose $I_n := [b_n , c_n]$ où $\begin{cases} b_n := \displaystyle \sum_{\{m, \ \alpha_m < \alpha_n\}} l_m \\ c_n := \displaystyle \sum_{\{m, \ \alpha_m \leq \alpha_n\}} l_m  = b_n + l_n \end{cases}$ \\

Les $(I_n)_{n\in \mathbb{Z}}$ sont alors dans le même ordre que les $(\alpha_n){n\in \mathbb{Z}}$, c'est à dire que $\alpha_{n_1} < \alpha_{n_2} \Leftrightarrow c_{n_1} < b_{n_2}$. On en déduit que les $(I_n)_{n\in \mathbb{Z}}$ sont disjoints, et l'irrationalité de $\alpha$ donne que leur union est dense dans $[0, 1]$. \\

On définit alors $f'$ sur chaque $I_n$ telle que : 
\begin{enumerate}
\item $f'(t) \to 1$ quand $t\to b_n$ ou $t\to c_n$
\item $f'(t) \to 1$ uniformément quand $n\to \infty$
\item $\displaystyle \int_{I_n}f'=l_{n+1}$
\end{enumerate}

On peut par exemple choisir : $f': b_n + x \mapsto \displaystyle e^{\gamma_n x (l_n -x)} $ où $\gamma_n$ est pris de telle sorte à vérifier le 3ème point (l'existence s'obtient par exemple avec le TVI, l'unicité provient de la stricte monotonie). \\ 
On remarquera que la condition $\displaystyle \lim\limits_{n \to \pm \infty } \frac{l_{n+1}}{l_n} = 1 $ permet d'assurer le second point. \\

On prolonge alors $f'$ sur $[0,1]$ par $f'(t)=1$ quand $\displaystyle t\in [0,1] \setminus \bigcup_{n\in \mathbb{Z}} I_n$, puis par $f'(t+1)=f'(t)$ sur $\mathbb{R}$ entier. Cette fonction est continue par ce qui précède, et alors en posant $f(t) := b_1 + \displaystyle \int_0^t f'(s) \mathrm{d}s$, on obtient que $f$ est un $\mathcal{C}^1$ difféomorphisme, et que de plus $f$ descend au quotient sur le tore car : \\
\begin{eqnarray*}
f(t+1) &=& b_1 +  \int_0^{t+1} f'(s) \mathrm{d}s = b_1 + \int_0^1 f'(s) \mathrm{d}s + \int_1^{t+1} f'(s) \mathrm{d}s \\
&=& b_1 + 1 + \int_0^t f'(s) \mathrm{d}s = f(t) + 1
\end{eqnarray*}

Notons $\tilde{f}$ le difféomorphisme induit sur le tore par $f$, et montrons qu'il convient. 

\begin{lemma} $\rho(\tilde{f})=\alpha$
\end{lemma}

\textbf{Preuve :} \\
On rappelle que $\alpha_n = n\alpha - \lfloor n\alpha \rfloor$. La construction de $f$ donne par récurrence que :
\begin{eqnarray*}
f^{n+1}(0) &=& \lfloor n\alpha \rfloor + b_{n+1} \ \ \text{ si } \ \alpha_n < 1-\alpha \\
f^{n+1}(0) &=&  \lfloor n\alpha \rfloor + b_{n+1} + 1 =  \lfloor (n+1)\alpha \rfloor + b_{n+1}\ \ \text{ si } \ \alpha_n \geq 1-\alpha
\end{eqnarray*}

D'où $ \ \displaystyle |\frac{f^n(0)}{n} - \alpha| \leq \frac{1}{n}$ et donc $\rho(\tilde{f})=\alpha$. \\


Maintenant, si l'on arrive à montrer que $\bigcup_{n\in \mathbb{Z}} I_n$ est invariant par $\tilde{f}$, comme cet ensemble est non trivial dans le tore, l'invariant de similitude mentionné plus haut donnera que $\tilde{f}$ ne peut pas être conjugué à la rotation $R_\alpha$ qui est minimale. On va donc montrer le lemme suivant :

\begin{lemma} $\forall n \in \mathbb{N}$, $f(I_n)=I_{n+1} \mod 1$.
\end{lemma}


\textbf{Preuve :} \textit{à faire ...} \\

On en déduit en particulier que, $f$ étant strictement croissante et continue, on a de plus $f(\overset{\circ}I_n)=\overset{\circ}I_{n+1}$.

\begin{lemma} $K := \displaystyle [0,1] \setminus \bigcup_{n\in \mathbb{Z}} \overset{\circ} I_n$ est un fermé non trivial et stable par $f$.
\end{lemma}


\textbf{Preuve :} La stabilité découle de la remarque précédente. Cet ensemble est clairement distinct de $[0,1]$ et il est de plus non vide car sinon, par compacité de $[0,1]$, on pourrait en extraire un recouvrement fini, mais alors la mesure de ce recouvrement serait strictement plus petite que $1$ au vu de la définition des $(I_n)_{n\in\mathbb{Z}}$, ce qui est absurde. \\ \qed  ~\\

\underline{Rq }: $K$ est en fait un ensemble de Cantor, au sens où il est fermé d'intérieur vide et sans point isolé.
\newpage
\printbibliography[heading=bibintoc, title={Références}]

\end{document}
