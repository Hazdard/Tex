\documentclass[11pt,a4paper]{article}

\usepackage{../jedusor}

\title{\textbf{Décomposition de Littlewood-Paley et opérateurs paradifférentiels}}
\date{}
\author{Sacha Ben-Arous, Mathis Bordet}

\begin{document}
\maketitle
ayaya

\section{Test}

\begin{thm}\label{bob}
This is a theorem.
\end{thm}

\begin{lemma}[Plongement lisse]\label{bab}
This is a theorem.This is a theorem.This is a theorem.This is a theorem.This is a theorem.This is a theorem.This is a theorem.This is a theorem.This is a theorem.This is a theorem.This is a theorem.This is a theorem.This is a theorem.This is a theorem.This is a theorem.
\end{lemma}

\begin{mth}{A}[Plongement isométrique]\label{foo}
This is a manual theorem.
\end{mth}

\begin{rmq}
This statement is true, I guess.
\end{rmq}

Here is \cref{bob,,bab} and \fcref{foo}.

\begin{defin}[Fibration]\label{ayaya}
A fibration is a mapping between two topological spaces that has the homotopy lifting property for every space \(X\).
\end{defin}

\begin{proof}
To prove it by contradiction try and assume that \cref{ayaya} the statement is false,
proceed from there and at some point you will arrive to a contradiction.
\end{proof}

\end{document}
	