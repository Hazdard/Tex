\documentclass[11pt,a4paper]{article}

\usepackage{../jedusor}

\usepackage[maxalphanames=99, maxnames=99, backend=bibtex, style=alphabetic, sorting=ynt]{biblatex}
\addbibresource{stage2A.bib}

\title{\textbf{Décomposition de Littlewood-Paley et opérateurs paradifférentiels}}
\date{}
\author{Sacha Ben-Arous, Mathis Bordet}

\begin{document}
\maketitle

% Citer les sources
% Rajouter les explications sur les lemmes de caractérisations
% Remarques sur la décomposition dyadique du bouquin Alinhac-Gérard.
% DEFINIR LES OBJETS DONT ON PARLE : SOBOLEV FOURIER HOLDER PARAPRODUIT
% UTILISER GHYS DANS LINTRO


Ce document a pour objectif d'exposer les principaux théorèmes de paralinéarisation en explicitant tous les outils utilisés, avec pour point de départ la décomposition de Littlewood-Paley. On s'appuiera sur \cite{metivier} et \cite{dgv} comme références, en particulier les chapitres 4 et 5 du livre de Métivier.
\section{Décomposition de Littlewood-Paley}
Nous allons dans cette section présenter la décomposition de Littlewood-Paley. C'est une décomposition de fonction dans laquelle chaque terme a un spectre borné. Nous allons également présenter des propriétés sur cette décomposition. 
\begin{defin}[Transformée de Fourier]
On note $\mathcal{F} : L^1(\R^d) \to C^0_0(\R^d)$,  $\ \mathcal{F}f(\xi) := \displaystyle \int_{\R^d} e^{-2i\pi \xi \cdot x}f(\xi)\mathrm{d}x$, et on pourra utiliser la notation $\hat{f}$ pour désigner $\mathcal{F}f$. On manipulera de plus le prolongement usuel de $\mathcal{F}$ à l'espace des distributions tempérées $\mathcal{S}'(\R^d)$.
\end{defin}
On s'intéresse dans un premier temps à l'existence et, plus précisément, à la construction de fonctions $C^\infty(\mathbb{R}^d)$ avec un support au voisinage de 0 mais également constantes au voisinage de 0.

Considérons le cas $d=1$ et notons  $g: \R^+ \to \R$
\[ g(x) = \begin{cases} 
1 & \text{si } 0 \leq x \leq \frac{1}{2} \\
\exp\left(-\frac{1}{\frac{1}{4} - |x - \frac{1}{2}|^2}\right) & \text{si } \frac{1}{2} \leq x \leq 1 \\
0 & \text{sinon} \end{cases}
\]
On a alors que $g$ appartient à $C^\infty(\mathbb{R})$ puisqu'on peut montrer par récurrence que :
\begin{equation}\label{plateau}
\forall n \in \mathbb{N}^*, \ \forall x \in \left[\frac{1}{2}, 1\right], \ g^{(n)}(x) = x Q_n(|x|) \exp\left(-\frac{1}{\frac{1}{4} - |x - \frac{1}{2}|^2}\right)
\end{equation}
avec $Q_n$ une fraction rationnelle dont le pôle se situe en 1 , ce qui montre la continuité des $(g^{(n)})_{n \in \mathbb{N}}$ par croissances comparées. On étend alors cette construction à $\mathbb{R}^d$ en notant $\psi(x)=g(|x|)$, qui est une fonction $C^\infty(\R^d)$ avec $\text{supp}(\psi) \subset B(0,1)$ et est égale à 1 sur $B(0,\frac{1}{2})$. \\
En posant $\chi(x)=\psi(x)-\psi\left(\frac{x}{2}\right)$, on a ainsi $\text{supp}(\chi(2^{-k} \cdot)) \subset B(0,2^{k+1}) \setminus B(0,2^{k-1})$, et par téléscopage on obtient l'égalité  :
\begin{equation*}
 \forall \xi \in \mathbb{R}^d, \ 1= \psi(\xi) + \sum_{k=0}^\infty \chi(2^{-k} \xi)
\end{equation*}

\begin{lemma}\label{decomp}
Pour tout $u \in \mathcal{S}$ on a :
\[ \hat{u} = \psi \hat{u} + \sum_{k=0}^\infty \chi(2^{-k} \cdot) \hat{u} \]
et la série converge dans l'espace de Schwartz.
\end{lemma}
\begin{proof}
Soit $u \in \mathcal{S}$, montrons que pout tout $\alpha$ et $\beta \in \mathbb{N}\times \mathbb{N}^d $, $\left\lVert x^\alpha \partial^\beta(u-\psi(2^{-k}x)u) \right\rVert_\infty \to_{k\to \infty} 0$
On considère pour cela :
\begin{align*}
    \left\lVert x^\alpha \partial^\beta(g-\psi(2^{-k}x)g)\right\rVert_\infty &= \left\lVert x^\alpha \partial^\beta(g-\psi(2^{-k}x)g)\right\rVert_{\infty,[2^{k-1};2^{k+1}]} \\
    &\leq \left\lVert x^\alpha \partial^\beta g\right\rVert_{\infty,[2^{k};2^{k+1}]} + \left\lVert x^\alpha \partial^\beta g(1-\psi(2^{-k}x)\right\rVert_{\infty,[2^{k-1};2^{k}]}
\end{align*}

Comme $g \in \mathcal{S}$, on a $\left\lVert x^\alpha \partial^\beta g\right\rVert_{\infty,[2^{k};2^{k+1}]}$ qui tend bien vers 0 lorsque $p$ tend vers $+\infty$. En utilisant la formule de derivation de Leibniz sachant que $\partial^j \psi = O(x^{j})$ (en utilisant \eqref{plateau}) on a que $\left\lVert x^\alpha \partial^\beta g\right\rVert_{[2^{k};2^{k+1}]}$ tend bien vers 0 lorsque $p$ tend vers $+\infty$. On utilise ensuite la continuité de la transformé de Fourier et de la transformé de Fourier inverse sur $\mathcal{S}$.
\end{proof}

\begin{defin}
On définit les opérateurs de la décomposition de Littlewood-Paley de la manière suivante :
\[
\text{Pour } u \in \mathcal{S}', \quad \widehat{\Delta_{-1} u} := \psi \cdot \hat{u}, \quad \widehat{\Delta_{k} u} := \chi(2^{-k} \cdot) \cdot \hat{u} \ \text{ si } k \geq 0
\]
\end{defin}

\begin{prop}
Soit $u \in \mathcal{S}'$, en posant :
\[ S_n u := \sum_{k=-1}^{n-1} \Delta_k u \]
On a que :
\[ \lim_{n \to \infty} S_n u = u \]
\end{prop}

\begin{proof}
Prenons $u \in \mathcal{S}'$ et $v \in \mathcal{S}$.
\[ \langle \mathcal{F}(S_n u), v \rangle = \langle \psi(2^{-n}\xi) \mathcal{F}(u), v \rangle = \langle \mathcal{F}(u), \psi(2^{-n}\xi) v \rangle \]

Or, $\lim_{n \to +\infty} \psi(2^{-n}\xi)v = v$ dans $\mathcal{S}(\mathbb{R}^d)$ par le \fcref{decomp}. On obtient donc :

\[ \mathcal{F}(S_n u) \to \mathcal{F}(u) \quad \text{dans} \ \mathcal{S} \]
Par continuité de $\mathcal{F}^{-1}$, on a finalement $S_n u \to u$ quand $n \to +\infty$.
\end{proof}
     
     
\begin{defin}[Espaces de Sobolev]
Pour tout $s\in \R^+$, on définit 
\begin{equation*}
H^s(\R^d) := \left\{ u\in L^2(\R^d), \xi \mapsto (1+|\xi|^2)^{\frac{s}{2}}\hat{u}(\xi) \in L^2(\R^d) \right\}
\end{equation*}
et on admet que c'est un espace de Hilbert muni de la norme $\|u\|_{H^s} := \displaystyle \left(\int_{\R^d} (1+|\xi|^2)^s\hat{u}(\xi)^2\mathrm{d}\xi \right)^\frac{1}{2}$
\end{defin}


\begin{defin}[Espaces de Zygmund]
Pour tout $s\in \R^+$, on définit 
\begin{equation*}
C^\alpha_*(\R^d) := \left\{ u\in L^2(\R^d), \sup_{k \geq -1} 2^{k\alpha}\|u\|_{L^\infty} < \infty \right\}
\end{equation*}
et on admet que c'est un espace de Hilbert muni de la norme $\|u\|_{C^\alpha_*} :=  \sup_{k \geq -1} 2^{k\alpha}\|u\|_{L^\infty} $
\end{defin}

\begin{lemma}[Inégalité de Bernstein]
Soit $B$ une boule, $1\leq p  \leq  q \leq +\infty$, $k\in \N$, et $\lambda >0$. Si $u\in L^p$ est tel que $\text{supp}(\hat{u})\subset \lambda B$, alors
\begin{equation}\label{bernstein}
\max_{|\alpha|=k}{\| \partial^\alpha u\|_{L^q}} \ \lesssim_k \lambda^{|\alpha| +d \left ( \frac{1}{p}- \frac{1}{q} \right )}\|u\|_{L^p}
\end{equation}
\end{lemma}

\begin{proof}
On commence par justifier que $u\in \mathcal{S}$. En effet, $u$ ayant un spectre borné, sa transformée de Fourier est dans l'espace de Schwartz, et l'opérateur transformée de Fourier étant un automorphisme de $\mathcal{S}$ dans lui-même, on en déduit que $u\in \mathcal{S}$. \\
Soit $\varphi \in C^\infty_c(\R^d)$ qui vaut 1 sur un voisinage de $B$, on a $\hat{u}(\xi)=\varphi(\lambda^{-1}\xi)\hat{u}(\xi)$, donc $u=\lambda^d u * g$, avec $g := \mathcal{F}^{-1}(\varphi)(\lambda \cdot)$, et donc $\partial^\alpha u =\lambda^d u * \partial^\alpha g$. L'inégalité de Young donne de plus que : $\|f * g \|_{L^q} \leq \|f\|_{L^p} \|g\|_{L^r}$, où $1\leq p,r\leq q \leq +\infty$, et $\frac{1}{p}+\frac{1}{r}= 1 + \frac{1}{q}$. Or :
\begin{eqnarray*}
\| \partial^\alpha g \|^r_{L^r} &=& \int_{\R^d} \left | \partial^\alpha \left(\mathcal{F}^{-1} \varphi(\lambda x)\right) \right |^r \mathrm{d}x =   \lambda^{|\alpha|r} \int_{\R^d} \left | \partial^\alpha \left(\mathcal{F}^{-1}(\varphi)\right)(\lambda x) \right |^r \mathrm{d}x  \\
&\leq& \lambda^{|\alpha|r-d} \| \partial^\alpha \mathcal{F}^{-1}\varphi \|^r_{L^r}
\end{eqnarray*}
Ce qui donne bien :
\begin{eqnarray*}
\| \partial^\alpha u \|_{L^q} \leq \lambda^{|\alpha| + d(1-\frac{1}{r})} \| \partial^\alpha \mathcal{F}^{-1}\varphi \|_{L^r} \|u\|_{L^p} = C_k\lambda^{|\alpha| + d(\frac{1}{p}-\frac{1}{q})} \|u\|_{L^p}
\end{eqnarray*}
\end{proof}


\begin{lemma}\label{young}
Il existe $C>0$ tel que pour tout $1\leq p \leq +\infty$, $u \in L^p(\R^d)$, 
\begin{align*}
\sup_{n\geq -1}\|S_nu\|_{L^p} &\leq C \|u\|_{L^p} & \sup_{k\geq -1}\|\Delta_ku\|_{L^p} &\leq C \|u\|_{L^p}
\end{align*}
\end{lemma}

\begin{proof}
On écrit $S_n u = 2^{nd} \mathcal{F}^{-1}(\psi(2^n\cdot))\ast u$. Par inégalité de Young on obtient :
$$\left\lVert S_n u \right\rVert_{L^p} \leq \left\lVert u \right\rVert_{L^p} \left\lVert 2^{nd} \mathcal{F}^{-1}(\psi(2^n\cdot)  \right\rVert_{L^1}$$ 
On procède de même pour $\|\Delta_k u\|_{L^p}$.
\end{proof}

\begin{lemma}[Presque-orthogonalité]\label{lqortho}
Pour tout $u\in L^2(\R^d)$, 
\begin{equation}\label{qortho}
\sum_{k \geq -1} \|\Delta_k u \|^2_{L^2} \leq \| u \|^2_{L^2} \leq 2 \sum_{k \geq -1} \|\Delta_k u \|^2_{L^2} 
\end{equation}
\end{lemma}

\begin{proof}
On part de $1= \psi(\xi) + \sum_{p=0}^\infty \chi(2^{-p} \xi)$.
Seule deux de ces fonctions ont une intersection de support non vide. On utilise alors : $a^2 +b^2 \leq (a+b)^2 \leq 2(a^2 +b^2)$ et on obtient:
$$ \frac{1}{2} \leq \psi(\xi)^2 + \sum_{p=0}^\infty \chi(2^{-p} \xi)^2 \leq 1$$
La seconde inégalité de \eqref{qortho} s'en déduit en multipliant l'inégalité ci-dessus par $\hat{u}$ et en utilisant l'identité de Plancherel.
\end{proof}

\begin{prop}[Caractérisation des espaces de Sobolev]
Si $s\in \R^+$, $u\in L^2(\R^d)$, on a alors $u\in H^s(\R^d) \Leftrightarrow \sum_{k\geq -1}2^{2ps}\|\Delta_k u \|_{L^2}^2 < +\infty$. De plus, il existe $C>0$ tel que :
\begin{equation}\label{carac_sobol}
\frac{1}{C}\sum_{k\geq -1}2^{2ks}\|\Delta_k u \|_{L^2}^2 \leq \|u\|_{H^s}^2 \leq C \sum_{k\geq -1}2^{2ks}\|\Delta_k u \|_{L^2}^2 
\end{equation}
\end{prop}

\begin{proof}
En notant $\left\langle \xi \right\rangle = \sqrt{1 + |\xi|^2}$, on a $\|u\|_{H^s} = \|\left\langle D \right\rangle^su\|_{L^2}$, et le \fcref{lqortho} donne 
\[\sum_{k \geq -1} \|\Delta_k \left\langle D \right\rangle^su \|^2_{L^2} \leq \| u \|^2_{H^s} \leq 2 \sum_{k \geq -1} \|\Delta_k \left\langle D \right\rangle^su \|^2_{L^2} \]
La formule de Plancherel et la définition de $\Delta_k$, on obtient l'existence de $C>0$ tel que $\forall k \geq -1$
\[\frac{1}{C}2^{ps}\|\Delta_ku\|_{L^2} \leq  \|\Delta_k\left\langle D \right\rangle^su\|_{L^2} \leq C2^{ps}\|\Delta_ku\|_{L^2} \]
et donc il existe $\tilde{C}$ tel que :
\[
\frac{1}{\tilde{C}}\sum_{k\geq -1}2^{2ks}\|\Delta_k u \|_{L^2}^2 \leq \|u\|_{H^s}^2 \leq \tilde{C} \sum_{k\geq -1}2^{2ks}\|\Delta_k u \|_{L^2}^2 \]
ce qui donne l'équivalence des normes voulue.
\end{proof}

\begin{lemma}[Injection de Sobolev]\label{inject}
Soit $s>\frac{d}{2}$, si $u\in H^s(\R^d)$ alors $u\in L^\infty(\R^d)$, et pour tout $k\in \N$ :
\begin{equation*}
\|\Delta_ku\|_{L^\infty} \lesssim 2^{k(\frac{d}{2}-s)} \|u\|_{H^s}
\end{equation*}
\end{lemma}


\begin{proof}
Soit $u\in H^s(\R^d)$, d'après le résultat précédent on sait que pour tout $k\in \N$, $\Delta_ku\in L^2(\R^d)$, donc sa transformée de Fourier y est de même, et comme elle est à support compact, elle est dans $L^1(\R^d)$. D'après la formule d'inversion, on a ainsi : 
\begin{equation*}
\Delta_ku(x) = \frac{1}{(2\pi)^d}\int_{\R^d} e^{i x\cdot \xi} \widehat{\Delta_ku}(\xi)\mathrm{d}\xi
\end{equation*}
Alors, l'inégalité de Cauchy-Schwarz donne :
\begin{align*}
\|\Delta_ku\|_{L^\infty} \leq \|\Delta_ku\|_{L^2} \left|B(0,C 2^k) \right| ^\frac{1}{2} &\lesssim 2^{k(\frac{d}{2}-s)} 2^{ks}\|\Delta_ku\|_{L^2} \leq 2^{k(\frac{d}{2}-s)} (\sum_{k\geq -1} 2^{ks}\|\Delta_ku\|^2_{L^2})^{\frac{1}{2}} \leq 2^{k(\frac{d}{2}-s)} \|u\|_{H^s}
\end{align*}
Ce qui consitue l'inégalité voulue. De plus, on en déduit que la série de terme général $(\Delta_ku)_{k\in \N}$ est absolument convergente dans $L^\infty$, donc par complétude elle converge simplement vers un $\tilde{u}$. On a déjà vu qu'elle converge dans $\mathcal{S}'$ vers $u$, donc $u=\tilde{u} \in L^\infty(\R^d)$.
\end{proof}


\begin{prop}\label{spec_ball}
Soit $(u_k)_{k\geq -1}$ tel que $\exists R >0, \forall k\geq -1, \text{supp }\hat{u}_k \subset B(0,R2^k)$.
\begin{itemize}
\item[•]Si $\sup $
\end{itemize}
\end{prop}


\begin{prop}[]\label{meyer}
Soit $s>0$, $n\in \N$, $n>s$. Il existe $C$ tel que pour toute famille $(u_k)_{k\in \N}$ dans $H^n(\R^d)$, si pour tout $\alpha \in \N^d$, avec $\ |\alpha| \leq n $ :
\begin{equation*}
\|\partial ^\alpha u_k \|_{L^2} \leq 2^{k(|\alpha|-s)}\varepsilon_k
\end{equation*}
où $(\varepsilon_k)_{k\in \N} \in \ell^2(\N)$, alors $u=\sum_k u_k \in H^s(\R^d) $, et $ \|u\|_{H^s}^2 \leq C \sum_k \varepsilon_k^2 $.
\end{prop}

\section{Estimations douces et paralinéarisation}

\begin{prop}[Estimations douces pour les paraproduits et leur restes] ~\\
\begin{itemize}
\item[•] $\forall s\in \R, u \in L^\infty, v\in H^s,$
\begin{equation*}
\|T_uv\|_{H^s} \leq C_s \|u\|_{L^\infty} \|v\|_{H^s}
\end{equation*}
\item[•] $\forall \alpha \in \R, u \in L^\infty, v\in C^\alpha_*,$
\begin{equation*}
\|T_uv\|_{C^\alpha_*} \leq C_\alpha \|u\|_{L^\infty} \|v\|_{C^\alpha}
\end{equation*}
\item[•] $\forall r,s\in \R,$ tels que $r+s>0, u \in C^r_*, v\in H^s,$
\begin{equation*}
\|R(u,v)\|_{H^{r+s}} \leq C_{r,s} \|u\|_{C^r_*} \|v\|_{H^s}
\end{equation*}
\item[•] $\forall \alpha,\beta\in \R,$ tels que $\alpha + \beta>0, u \in C^\alpha_*, v\in C^\beta_*,$
\begin{equation*}
\|R(u,v)\|_{C^{\alpha + \beta}_*} \leq C_{\alpha,\beta} \|u\|_{C^\alpha_*} \|v\|_{C^\beta_*}
\end{equation*}
\end{itemize}
\end{prop}

\begin{prop}[Estimations douces pour le produit] ~\\
\begin{itemize}
\item[•] $\forall s>0,\ u,v \in L^\infty \cap H^s(\R^d)$,
\begin{equation*}
\|uv\|_{H^s} \leq C (\|u\|_{L^\infty}\|v\|_{H^s} + \|v\|_{L^\infty}\|u\|_{H^s})
\end{equation*}
\item[•] $\forall \alpha \in \R^+ \setminus \N,\ u,v \in C^\alpha_*(\R^d)$,
\begin{equation*}
\|uv\|_{C^\alpha_*} \leq C (\|u\|_{L^\infty}\|v\|_{C^\alpha_*} + \|v\|_{L^\infty}\|u\|_{C^\alpha_*})
\end{equation*}
\end{itemize}
\end{prop}


\begin{thm}\label{paralin}
Soit $F$ une fonction $C^{\infty}$ de $\R$ telle que $F(0)=0$. Si $u\in H^s(\R^d)$, avec $\rho := s - \frac{d}{2}>0 $, alors 
\begin{equation}
F(u) - T_{F'(u)}u \in H^{s+\rho}(\R^d).
\end{equation}
\end{thm}
Pour prouver ce théorème, on commence par montrer le lemme suivant : 


\begin{lemma}\label{lem_paralin}
Soit $F$ une fonction $C^{\infty}$ de $\R$ telle que $F(0)=0$. Si $u\in H^s(\R^d)\cap L^{\infty}(\R^d) $, avec $s\geq 0 $, alors $F(u) \in H^s(\R^d)$ et 
\begin{equation}
\|F(u)\|_{H^s} \leq C_s \|u\|_{L^{\infty}} \|u\|_{H^s}
\end{equation}
\end{lemma}


\begin{proof}
Si $s=0$, le résultat se déduit de l'existence d'une fonction $G$ continue telle que $F(u)=uG(u)$. Alors, comme $u\in L^2$ et $G(u) \in L^{\infty}$ car $u$ est bornée, on obtient bien $F\in L^2$. \\
Quand $s>0$, on remarque qu'il existe $C_\alpha$  indépendant de $u$ et $k$ telle que  : 
\begin{equation}\label{ineq_l2}
\|\partial^\alpha \Delta_k u \|_{L^2} \leq C_\alpha 2^{(|\alpha|-s)k} \varepsilon_k
\end{equation}
avec $\sum \varepsilon_k^2 =  \|u\|_{H^s}^2$. En effet, l'inégalité de Bernstein (\ref{bernstein}) puis la caractérisation des espaces de Sobolev (\ref{carac_sobol}) donnent le résultat voulu. On a de plus,
\begin{equation}\label{ineq_unif}
\|\partial^\alpha \Delta_k u \|_{L^{\infty}} \leq C_\alpha 2^{|\alpha|k} \|u\|_{L^{\infty}}
\end{equation}
toujours par l'inégalité de Bernstein. Le lemme de presque-orthogonalité (\ref{qortho}) donnant que $(\Delta_k u)_{k\in \N}$ est terme général d'une série absolument convergente, la complétude de $L^2$ fournit alors la convergence simple de cette série, i.e. $S_nu \to u$ dans $L^2$. De plus, d'après le \fcref{young},  $\|S_pu\|_{L^\infty} \leq C \|u\|_{L^\infty}$. On en déduit que $F(S_nu) \to F(u)$ dans $L^2$ car :
\[ \| F(S_nu) - F(u)\|_{L^2} \leq C \sup_{t\in[0,1]} \|F'(tS_nu- (1-t)u \|_{L^\infty}\|S_nu-u\|_{L^2} \to 0 \]
Un argument télescopique donne alors : 
\begin{equation}\label{telesc}
F(u) = F(S_0 u) + \sum_{k=0}^{+\infty} F(S_{k+1} u) - F(S_k u) = F(S_0 u) + \sum_{k=0}^{+\infty} m_k \Delta_k u
\end{equation}
où \[m_k := \int_0^1 F'(S_ku + t\Delta_ku)\mathrm{d}t\]
Alors, on obtient dans un premier temps que : 
\begin{equation*}
\| \partial^\alpha F'(S_ku + t\Delta_ku)\|_{L^{\infty}} \leq C_{\alpha,F} 2^{|\alpha|k} \|u\|_{L^{\infty}}
\end{equation*}
Pour cela on utilise la règle de la chaine (plus précisement la formule de Faà di Bruno), on majore uniformément les termes en $F'$ avec le \fcref{young} et on utilise (\ref{ineq_unif}) pour les termes en $u$.
En intégrant, on obtient alors :
\begin{equation}\label{majint}
\|\partial^\alpha m_k \|_{L^\infty} \leq C_{\alpha,F} 2^{|\alpha|k} \|u\|_{L^{\infty}}
\end{equation}
Donc par la formule de Leibniz et l'inégalité (\ref{ineq_l2}), on obtient :
\begin{equation*}
\|\partial^\alpha(m_k \Delta_k u) \|_{L^2} \leq C_{\alpha,F} 2^{(|\alpha|-s)k} \|u\|_{L^{\infty}} \varepsilon_k
\end{equation*}
On peut donc conclure par la \fcref{meyer}.
\end{proof}


\begin{proof}[Preuve du \fcref{paralin}]On commence par remarquer que quitte à soustraire un terme linéaire $au$ à $F(u)$, on peut supposer que $F'(0)=0$. Cela ne change rien à la preuve car :
\[F(u)+au - T_{F'(u) + a}u = F(u) + au - T_{F'(u)}u  -T_{a}u =  F(u) + au - T_{F'(u)}u  -au =  F(u) - T_{F'(u)}u \]
Ensuite, comme $\rho >0$, le \fcref{inject} donne $u\in L^\infty(\R^d)$. Par définition, on a \[T_{F'(u)}u = S_{-3}F'(u)\cdot u_0 + \sum_{k=0}^\infty S_{k-2} F'(u)\cdot \Delta_k u \]
En utilisant (\ref{telesc}), comme $F(S_0u)$ et $S_{-3}F'(u) \cdot u_0$ sont dans $H^\infty$, il suffit de prouver que :
\[\sum_{k=0}^\infty(m_k-S_{k-2}g)\Delta_k u \ \in H^{s+\rho}\]
Cela découle de la \fcref{meyer}, que l'on peut appliquer d'une part grâce à (\ref{ineq_l2}), et d'autre part car on a l'inégalité :
\begin{equation*}
\| \partial^\alpha (m_k - S_{k-2}F'(u))\|_{L^\infty} \leq C_\alpha 2^{(|\alpha|-\rho)k}
\end{equation*} 
Pour obtenir cette dernière, on va montrer séparément :
\begin{align}
\| \partial^\alpha (m_k - F'(S_{k-2}u))\|_{L^\infty} &\leq C_\alpha 2^{(|\alpha|-\rho)k} \label{eq1} \\
\| \partial^\alpha (F'(S_{k}u) - S_kF'(u))\|_{L^\infty} &\leq C_\alpha 2^{(|\alpha|-\rho)k} \label{eq2}
\end{align}
On commence par écrire la formule de Taylor avec reste intégral, qui donne 
\[F'(S_ku+t\Delta_ku)-F'(S_{k-2}u)=\mu_kw_k\]
avec 
\begin{equation*}
w_k=(\Delta_{k-2}u + \Delta_{k-1}u + t\Delta_ku) \quad \text{ et } \mu_k = \int_0^1F''(S_{k-2}u + \tau w_k)\mathrm{d}\tau.
\end{equation*}
De manière analogue à \eqref{majint}, on a 
\begin{equation*}
\|\partial^\alpha \mu_k \|_{L^\infty} \leq C_{\alpha,F} 2^{|\alpha|k} \|u\|_{L^\infty}
\end{equation*}
Tandis que $w_k$ vérifie
\begin{equation*}
\| \partial^\alpha w_k\|_{L^\infty} \leq C_{\alpha} 2^{\frac{d}{2}k} \| \partial^\alpha w_k\|_{L^2} \leq C_{\alpha} 2^{\frac{d}{2}k} 2^{(|\alpha|-s)k} \varepsilon_k  \leq \tilde{C}_{\alpha} 2^{(|\alpha|-\rho)k}
\end{equation*}
où l'on a utilisé l'inégalité de Bernstein \eqref{bernstein}, puis \eqref{ineq_l2}. On en déduit donc que 
\begin{equation*}
\|\partial^\alpha (\mu_kw_k)\|_{L^\infty} \leq C_\alpha 2^{(|\alpha|-\rho)k} 
\end{equation*}
Or 
\begin{equation*}
m_k - F'(S_{k-2}u) = \int_0^1 \mu_k w_k\mathrm{d}t
\end{equation*}
Ce qui donne \eqref{eq1}.
Pour montrer la seconde inégalité, on commence par décomposer en deux membres le terme à majorer :
\begin{equation*}
\left[ F'(S_{k}u) - S_k F'(S_{k}u) \right] + \left[ S_k F'(S_{k}u) - S_kF'(u) \right] = (\rom{1}) + (\rom{2})
\end{equation*}
L'inégalité de Bernstein \eqref{bernstein} donne alors 
\begin{equation*}
\|\partial^\alpha S_k (F'(u) - F'(S_ku) )\|_{L^\infty} \lesssim_\alpha 2^{(|\alpha| + \frac{d}{2})k } \|S_k (F'(u) - F'(S_ku) )\|_{L^2}.
\end{equation*}
De plus :
\begin{equation*}
\|S_k (F'(u) - F'(S_ku) )\|_{L^2} \lesssim \|F'(u) - F'(S_ku) )\|_{L^2} \lesssim \|u-S_ku\|_{L^2} \lesssim 2^{-ks}\|u\|_{H^s}
\end{equation*}
grâce au \fcref{young}, puis aux accroissements finis, et finalement avec une majoration du reste géométrique dans la caractérisation des espaces de Sobolev \eqref{carac_sobol}. Ainsi $(\rom{2})$ vérifie l'inégalité \eqref{eq2}. \\
Il reste maintenant à étudier $(\rom{1})$. Pour cela, on remarque que $S_ku \in H^\infty$ car sa transformée de Fourier est à support compact. Alors, d'une part $S_ku \in L^\infty	$ par le \fcref{young} car $u\in L^\infty$, et sa norme est bornée indépendamment de $k$. D'autre part, en écrivant la norme usuelle de Sobolev et en utilisant l'inégalité de Bernstein, on a que pour tout $N \in \N$ :
\begin{align*}
 \|S_ku\|_{H^{s+N}} &\leq C\|S_ku\|_{H^s} + C\sum_{|\alpha|=s+N} \|\partial^\alpha S_k u \|_{L^2} \\
& \leq \|S_ku\|_{H^s} + C_{\alpha,N} 2^{kN} \|S_ku\|_{H^s} \\
& \leq C_{\alpha,N} 2^{kN}\|u\|_{H^s} .
\end{align*}
où l'on a observé que $\|S_ku\|_{H^s} \leq \|u\|_{H^s}$ à partir de l'écriture utilisant les multiplicateurs de Fourier, et avec la norme de Sobolev adaptée. Alors, le \fcref{lem_paralin} donne que $F'(S_ku)\in H^{s+N}$, et 
\begin{equation}\label{utillem}
\|F'(S_ku)\|_{H^{s+N}} \leq C_{\alpha,N}2^{kN}\|u\|_{H^s}\|u\|_{L^\infty}
\end{equation}
En utilisant l'inégalité de Bernstein puis le \fcref{inject}, on remarque que pour $\sigma > |\alpha| + \frac{d}{2}$ et $a\in H^\sigma(\R^d)$,
\begin{equation*}
\|\partial^\alpha \Delta_j a\|_{L^\infty} \leq C 2^{j(\frac{d}{2}-\sigma+|\alpha|)} \|a\|_{H^\sigma}
\end{equation*}
Et alors, comme $a-S_ka=\sum_{j\geq k} \Delta_j a$, par majoration d'un reste géométrique, on a :
\begin{equation}\label{aled}
\|\partial^\alpha (a-S_k a)\|_{L^\infty} \leq C 2^{k(\frac{d}{2}-\sigma+|\alpha|)} \|a\|_{H^\sigma}
\end{equation}

En appliquant \eqref{aled} avec $a=F'(S_ku)$ et $\sigma = s + N$ où $N$ est suffisamment grand pour que $s+N > \frac{d}{2} + |\alpha|$, on a 
\begin{align*}
\|\partial^\alpha (F'(S_ku)-S_k F'(S_ku))\|_{L^\infty} &\leq C 2^{k(\frac{d}{2}-s-N+|\alpha|)} \|F'(S_ku)\|_{H^{s+N}} \\
& \leq  C_{\alpha,N} 2^{k(\frac{d}{2}-s+|\alpha|)} \|u\|_{H^s}
\end{align*}
où l'on a utilisé \eqref{utillem}, ce qui donne finalement la majoration attendue pour $(\rom{1})$, conclut la preuve de l'inégalité \eqref{eq2} et achève donc la preuve du \fcref{paralin}.
\end{proof}



















\newpage
\printbibliography[heading=bibintoc, title={Références}]
\end{document}
	