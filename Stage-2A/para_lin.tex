\documentclass[11pt,a4paper]{article}

\usepackage{../jedusor}
\usepackage[maxalphanames=99, maxnames=99, backend=bibtex, style=alphabetic, sorting=ynt]{biblatex}
\addbibresource{stage2A.bib}
\title{\textbf{Décomposition de Littlewood-Paley et opérateurs paradifférentiels}}
\date{}
\author{Sacha Ben-Arous, Mathis Bordet}

\begin{document}
\maketitle

Ce document a pour objectif d'exposer les principaux théorèmes de paralinéarisation en explicitant tous les outils utilisés, avec pour point de départ la décomposition de Littlewood-Paley. On s'appuiera sur \cite{metivier} et \cite{dgv} comme références, en particulier les chapitres 4 et 5 du livre de Métivier.
\section{Décomposition de Littlewood-Paley}


\begin{thm}[Inégalité de Bernstein]
Soit $r_1,r_2 >0$. Il existe $C$ tel que pour tout $k \geq 1$, $\lambda >0$, $u\in L^p(\R^d)$ $(1\leq p \leq +\infty)$, on a :
\begin{equation}\label{bernstein}
\text{supp } \hat{u} \subset B(0,r_1\lambda) \Rightarrow \sup_{|\alpha|=k} \|\partial^\alpha u\|_{L^p} \leq C^k \lambda^k \|u\|_{L^p}
\end{equation}
\begin{equation}
\text{supp } \hat{u} \subset C(0,r_1\lambda,r_2\lambda) \Rightarrow  C^{-k} \lambda^k \|u\|_{L^p} \leq \sup_{|\alpha|=k} \|\partial^\alpha u\|_{L^p} \leq C^k \lambda^k \|u\|_{L^p}
\end{equation}
\end{thm}


\begin{lemma}\label{young}
Il existe $C>0$ tel que pour tout $1\leq p \leq +\infty$, $u \in L^p(\R^d)$, 
\begin{align*}
\sup_{p\geq -1}\|S_pu\|_{L^p} &\leq C \|u\|_{L^p} & \sup_{p\geq -1}\|\Delta_pu\|_{L^p} &\leq C \|u\|_{L^p}
\end{align*}
\end{lemma}


\begin{lemma}[Quasi-orthogonalité]
Pour tout $u\in L^2(\R^d)$, 
\begin{equation}\label{qortho}
\sum_{p \geq -1} \|\Delta_p u \|^2_{L^2} \leq \| u \|^2_{L^2} \leq 2 \sum_{p \geq -1} \|\Delta_p u \|^2_{L^2} 
\end{equation}
\end{lemma}


\begin{thm}[Caractérisation des espaces de Sobolev]
Si $s\in \R$, $u\in L^2(\R^d)$, on a alors $u\in H^s(\R^d) \Leftrightarrow \sum_{p\geq -1}2^{2ps}\|\Delta_p u \|_{L^2}^2 < +\infty$. De plus, il existe $C>0$ tel que :
\begin{equation}\label{carac_sobol}
\frac{1}{C}\sum_{p\geq -1}2^{2ps}\|\Delta_p u \|_{L^2}^2 \leq \|u\|_{H^s} \leq C \sum_{p\geq -1}2^{2ps}\|\Delta_p u \|_{L^2}^2 
\end{equation}
\end{thm}

\section{Théorèmes de paralinéarisation}

\begin{thm}\label{paralin}
Soit $F$ une fonction $C^{\infty}$ de $\R$ telle que $F(0)=0$. Si $u\in H^s(\R^d)$, avec $\rho := s - \frac{d}{2}>0 $, alors 
\begin{equation}
F(u) - T_{F'(u)}u \in H^{s+\rho}(\R^d).
\end{equation}
\end{thm}
Pour prouver ce théorème, on commence par montrer le lemme suivant : 

\begin{lemma}
Soit $F$ une fonction $C^{\infty}$ de $\R$ telle que $F(0)=0$. Si $u\in H^s(\R^d)\cap L^{\infty}(\R^d) $, avec $s>0 $, alors $F(u) \in H^s(\R^d)$ et 
\begin{equation}
\|F(u)\| \leq C_s \|u\|_{L^{\infty}} \|u\|_{H^s}
\end{equation}
\end{lemma}

\begin{proof}
Si $s=0$, le résultat se déduit de l'existence d'une fonction $G$ continue telle que $F(u)=uG(u)$. Alors, comme $u\in L^2$ et $G(u) \in L^{\infty}$ car $u$ est bornée, on obtient bien $F\in L^2$. \\
Quand $s>0$, on remarque qu'il existe $C_\alpha$  indépendant de $u$ et $k$ telle que  : 
\begin{equation}
\|\partial^\alpha \Delta_k u \|_{L^2} \leq C_\alpha 2^{(|\alpha|-s)k} \varepsilon_k
\end{equation}
avec $\sum \varepsilon_k^2 =  \|u\|_{H^s}^2$. En effet, l'inégalité de Bernstein (\ref{bernstein}) puis la caractérisation des espaces de Sobolev (\ref{carac_sobol}) donnent le résultat voulu. On a de plus,
\begin{equation}
\|\partial^\alpha \Delta_k u \|_{L^{\infty}} \leq C_\alpha 2^{|\alpha|k} \|u\|_{L^{\infty}}
\end{equation}
toujours par l'inégalité de Bernstein. Le lemme de quasi-orthogonalité (\ref{qortho}) donnant que $(\Delta_k u)_{k\in \N}$ est terme général d'une série absolument convergente, la complétude de $L^2$ fournit alors la convergence simple de cette série, i.e. $S_n \to u$ dans $L^2$. De plus, d'après le \fcref{young},  $\|S_pu\|_{L^\infty} \leq C \|u\|_{L^\infty}$. On en déduit que $F(S_nu) \to F(u)$ dans $L^2$ car :
\[ \| F(S_nu) - F(u)\|_{L^2} \leq C \sup_{t\in[0,1]} \|F'(tS_nu- (1-t)u \|_{L^\infty}\|S_nu-u\|_{L^2} \to 0 \]

Un argument télescopique donne alors : 
\begin{equation*}
F(u) = F(S_0 u) + \sum_{k=0}^{+\infty} F(S_{p+1} u) - F(S_p u) = F(S_0 u) + \sum_{k=0}^{+\infty} m_p \Delta_p u
\end{equation*}

où \[m_p := \int_0^1 F'(S_pu + t\Delta_pu)\mathrm{d}t\]




\end{proof}


\newpage
\printbibliography[heading=bibintoc, title={Références}]
\end{document}
	