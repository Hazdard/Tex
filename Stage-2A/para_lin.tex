\documentclass[11pt,a4paper]{article}

\usepackage{../jedusor}
\usepackage[maxalphanames=99, maxnames=99, backend=bibtex, style=alphabetic, sorting=ynt]{biblatex}
\addbibresource{stage2A.bib}
\title{\textbf{Décomposition de Littlewood-Paley et opérateurs paradifférentiels}}
\date{}
\author{Sacha Ben-Arous, Mathis Bordet}

\begin{document}
\maketitle

Ce document a pour objectif d'exposer les principaux théorèmes de paralinéarisation en explicitant tous les outils utilisés, avec pour point de départ la décomposition de Littlewood-Paley. On s'appuiera sur \cite{metivier} et comme \cite{dgv} références, en particulier les chapitres 4 et 5 du livre de Métivier.

\section{Théorèmes de paralinéarisation}

\begin{thm}\label{paralin}
Soit $F$ une fonction $C^{\infty}$ de $\R$ telle que $F(0)=0$. Si $u\in H^s(\R^d)$, avec $\rho := s - \frac{d}{2}>0 $, alors 
\begin{equation}
F(u) - T_{F'(u)}u \in H^{s+\rho}(\R^d).
\end{equation}
\end{thm}
Pour prouver ce théorème, on commence par montrer le lemme suivant : 

\begin{lemma}
Soit $F$ une fonction $C^{\infty}$ de $\R$ telle que $F(0)=0$. Si $u\in H^s(\R^d)\cap L^{\infty}(\R^d) $, avec $s>0 $, alors $F(u) \in H^s(\R^d)$ et 
\begin{equation}
\|F(u)\| \leq C_s \|u\|_{L^{\infty}} \|u\|_{H^s}
\end{equation}
\end{lemma}

\begin{proof}
Si $s=0$, le résultat se déduit de l'existence d'une fonction $G$ continue telle que $F(u)=uG(u)$. Or $u\in L^2$, et $G(u) \in L^{\infty}$ car $u$ est bornée, ce qui donne le résultat.
\par Quand $s>0$, on remarque qu'il existe $C_\alpha$  indépendant de $u$ et $k$ telle que  : 
\begin{equation}
\|\partial^\alpha \Delta_k u \|_{L^2} \leq C_\alpha 2^{(|\alpha|-s)k} \varepsilon_k
\end{equation}
\end{proof}


\newpage
\printbibliography[heading=bibintoc, title={Références}]
\end{document}
	