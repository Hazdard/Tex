\documentclass[11pt,a4paper]{article}

\usepackage{../jedusor}
\usepackage[maxalphanames=99, maxnames=99, backend=bibtex, style=alphabetic, sorting=ynt]{biblatex}
\addbibresource{stage2A.bib}
\title{\textbf{Difféomorphisme du Cercle}}
\date{}
\author{Sacha Ben-Arous, Mathis Bordet}

\begin{document}
\maketitle
Nous allons dans cette section présenter rapidement les homéomorphismes du cercle et énoncer quelques propriétés fondamentales puisque ces objets nous offrent un cadre d'application des techniques de résolution d'équations que nous verrons dans les sections suivantes.
\section{Application du cercle}
On utilise les notations suivantes :
On note $\mathbb{S}^1$ le cercle de dimension 1 : $\left\{ z \in \mathbb{C} \mid |z|=1 \right\}$.
\\
On note également $\Pi$ l'application $ t \to \exp (2i\pi t) $ (la projection de $\mathbb{R}$ sur $\mathbb{S}^1$).
\begin{defin}
Soit $f \ \text{:} \ \mathbb{S}^1 \to \mathbb{S}^1$. On appelle relèvement de $f$, une application $F \ \text{:} \ \mathbb{R} \to \mathbb{R}$ vérifiant :
\begin{equation*}
f \circ \Pi = \Pi \circ F
\end{equation*}
\end{defin}
\begin{rmq}
Les relèvements permettent de faire le parallèle entre les fonctions du cercle et les fonctions réelles.
\end{rmq}
\begin{thm}
Toute application continue du cercle possède un relèvement continu. De plus, tous ses relèvements continus diffèrent d'une constante entière.
\end{thm}
Dans la suite, toute propriété qui sera énoncée pour une application continue du cercle concernant ses relèvements, ce dernier sera pris continu.
\begin{defin}
Pour une application continue du cercle $f$, on définit son degré $deg(f)= F(1)-F(0)$ avec $F$ un relèvement de $f$.
\end{defin}
\begin{rmq}
On peut démontrer que ce nombre est un entier relatif et qu'il ne diffère pas selon les relèvements (continus).
\end{rmq}
\begin{defin}
On dit que $f$ préserve l'orientation si pour tout triangle avec ses sommets sur $\mathbb{S}^1$, l'image de ses sommets par $f$ n'inverse pas l'ordre de ses sommets.
\end{defin}
Ce qui nous intéressera dans la suite de cette étude sont surtout les homéomorphismes du cercle, qui sont les applications continues, bijectives et de réciproques continues (la dernière propriété est redondante du fait de la compacité de $\mathbb{S}^1$).

\begin{prop}
Prenons $f$ un homéomorphisme préservant l'orientation alors, $deg(f)=1$ et de plus ses relèvements sont croissants.
\end{prop}

\begin{prop}
Soit $f$ un homéomorphisme préservant l'orientation et $F$ un relèvement de $f$. Alors le nombre $\rho (F)= \frac{F^n(x)}{n}$ existe et ne dépend pas de $x$ et ne diffère que d'un entier relatif entre chaque relèvement.
\end{prop}

\begin{defin}
On définit alors le nombre de rotation de $f$ $\rho_0(f) = \rho (F) \mod[1]$
\end{defin}
Ce nombre est primordial puisque celui-ci permet de classifier les homéomorphismes préservant l'orientation puisqu'en effet on a le Théorème de Poincaré :
\begin{thm}\label{Théorème de Poincaré}
Soit $f$ un homéomorphisme préservant l'orientation, si son nombre de rotation est un irrationnel alors $f$ est semi-conjugué à la rotation d'angle $\rho_0(f)$.
\end{thm}
On peut même dire un peu plus si on a des informations sur la régularité de $f$, c'est l'essence du théorème de Denjoy :
\begin{thm}\label{Théorème Denjoy}
En plus des hypothèses du théorème de Poincaré, si $f$ est $C^2$ alors il est cette fois-ci conjugué à la rotation d'angle $\rho_0(f)$.
\end{thm}

\end{document}
