\documentclass[10pt]{beamer}
\usetheme{metropolis}           % Use metropolis theme
\usepackage[utf8]{inputenc}
\usepackage{amsmath,amsfonts,amssymb}
\usepackage{dsfont}
\usepackage{graphicx}
\usepackage{caption}
\usefonttheme[onlymath]{serif}


\title{Difféomorphismes du cercle et théorème de Nash-Moser}
\date{14 Mai 2024}
\author{Sacha Ben-Arous, Mathis Bordet}
\institute{ENS Paris-Saclay}
\begin{document}
  \maketitle
\begin{frame}
\tableofcontents
\end{frame}  

\section{Calcul paradifférentiel}

\begin{frame}{Multiplicateurs de Fourier}

On définit les opérateurs de la décomposition de Littlewood-Paley de la manière suivante :
\[
\text{Pour } u \in \mathcal{S}', \quad \widehat{\Delta_{-1} u} := \psi \cdot \hat{u}, \quad \widehat{\Delta_{k} u} := \chi(2^{-k} \cdot) \cdot \hat{u} \ \text{ si } k \geq 0
\]

\end{frame}

\begin{frame}{Décomposition régularisante}

On définit les opérateurs de la décomposition de Littlewood-Paley de la manière suivante :
\[
\text{Pour } u \in \mathcal{S}', \quad \widehat{\Delta_{-1} u} := \psi \cdot \hat{u}, \quad \widehat{\Delta_{k} u} := \chi(2^{-k} \cdot) \cdot \hat{u} \ \text{ si } k \geq 0
\]

\end{frame}


\begin{frame}{Caractérisations d'espaces fonctionnels}

On définit les opérateurs de la décomposition de Littlewood-Paley de la manière suivante :
\[
\text{Pour } u \in \mathcal{S}', \quad \widehat{\Delta_{-1} u} := \psi \cdot \hat{u}, \quad \widehat{\Delta_{k} u} := \chi(2^{-k} \cdot) \cdot \hat{u} \ \text{ si } k \geq 0
\]

\end{frame}

\begin{frame}{Paraproduits}

On définit les opérateurs de la décomposition de Littlewood-Paley de la manière suivante :
\[
\text{Pour } u \in \mathcal{S}', \quad \widehat{\Delta_{-1} u} := \psi \cdot \hat{u}, \quad \widehat{\Delta_{k} u} := \chi(2^{-k} \cdot) \cdot \hat{u} \ \text{ si } k \geq 0
\]

\end{frame}

\begin{frame}{Paralinéarisation}

On définit les opérateurs de la décomposition de Littlewood-Paley de la manière suivante :
\[
\text{Pour } u \in \mathcal{S}', \quad \widehat{\Delta_{-1} u} := \psi \cdot \hat{u}, \quad \widehat{\Delta_{k} u} := \chi(2^{-k} \cdot) \cdot \hat{u} \ \text{ si } k \geq 0
\]

\end{frame}

\section{Difféomorphismes du cercle}
\begin{frame}{Dynamique en dimension 1}
    On considère le cercle $\mathbb{S}^1 := \left\{z, |z|=1 \right\}$, ainsi que le plongement $\Pi : \mathbb{R} \mapsto \mathbb{S}^1, t \mapsto e^{2i\pi t}$. \\
    
Pour $f:\mathbb{S}^1 \mapsto \mathbb{S}^1$, on dit que $F: \mathbb{R} \mapsto \mathbb{R}$ est un relèvement si $\Pi \circ F = f \circ \Pi$, i.e $f(e^{2i\pi t}) = e^{2i\pi F(t)}$

\textbf{Lemme :} Si $f$ est un $\mathcal{C}^k$-difféomorphisme du cercle, alors il existe un relèvement de $f$ qui est un $\mathcal{C}^k$-difféomorphisme (de $\mathbb{R}$). \\~\\

Dans la suite, on considèrera a minima des homéomorphismes, ainsi que leur relèvements réguliers associés.
\end{frame}

\begin{frame}{Nombre de rotation et Théorème Poincaré }
    Si $f$ est un homéomorphisme, $F$ un relèvement, et $x\in \mathbb{R}$, la suite $\displaystyle (\frac{F^n(x)}{n})_{n\in \mathbb{N}}$ converge vers une limite indépendante de $x$, notée $\rho(F)$. On définit alors $\rho(f):= \rho(F) \mod 1$. \\~\\
   

\textbf{Théorème (Poincaré) :} Si $f$ est un homéomorphisme de nombre de rotation $\alpha$ irrationnel, alors $f$ est semi-conjugué à $R_\alpha$, i.e il existe $h$ continue telle que $h \circ f = R_\alpha \circ h$. \\~\\

\end{frame}





\section{Théorème d'Arnold}
\begin{frame}
\frametitle{But}
Le But du théorème d'Arnold est déterminé la régularité des conjugaison d'un Difféomorphisme g d'angle de rotation $\alpha$ irrationnelle proche de la rotation $R_\alpha$. c'est a dire résoudre :
$$\eta(x + \alpha)= g \cdot \eta(x)$$
\\
d'inconnue $\eta$.
\end{frame}
\begin{frame}
\frametitle{Cadre}
On écrit alors :
\[ g(x) = R_\alpha + f(x) \]
avec $f$ "assez petit". \\
On cherche alors $\eta= id + u $. 
L'équation a résoudre:
$$\eta(x + \alpha)= g \cdot \eta(x)$$
devient alors : 
$$ \Delta_\alpha u= f \cdot (id +u)  \; \text{avec  } \Delta_\alpha u(x)= u(x+\alpha) - u(x)$$

\end{frame}

\begin{frame}
\frametitle{Perte de régularité}
Simplement en voulant résoudre
$$ U(x+ \alpha) - U(x) = \eta (x) $$
On a que 
\[ (\exp(2 i \pi n \alpha)-1)\widehat{u} (n) = \widehat{f} (n) \]
Avec la condition diophansiènne :
$$ \left\| \alpha - \frac{m}{n} \right\| \geq \frac{k}{n^\sigma}$$
cela induit la perte de régularité suivante, si $f \in H^{s+\sigma}$ alors $\Delta_\alpha^{-1}f \in H^{s}$


\end{frame}


\begin{frame}
\frametitle{Enoncé et Preuve}
Le théorème d'Arnold affirme 
\end{frame}


\end{document}

