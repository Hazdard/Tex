\documentclass[11pt,a4paper]{article}
\textheight245mm
\textwidth170mm
\hoffset-21mm
\voffset-15mm
\parindent0pt
\usepackage[utf8]{inputenc}
\usepackage{dsfont}
\usepackage{graphicx}
\usepackage{caption}
\usepackage{fancyhdr}
\usepackage{amsmath,amsfonts,amssymb}
\usepackage[french]{babel}
\usepackage[hidelinks]{hyperref} 
\hypersetup{
  colorlinks   = true,    % Colours links instead of ugly boxes
  urlcolor     = blue,    % Colour for external hyperlinks
  linkcolor    = black,    % Colour of internal links
  citecolor    = black      % Colour of citations
}
\usepackage{../../zephyr}
\pagestyle{fancy}

\usepackage{array,multirow,makecell}
\setcellgapes{4pt}
\makegapedcells
\newcolumntype{R}[1]{>{\raggedleft\arraybackslash }b{#1}}
\newcolumntype{L}[1]{>{\raggedright\arraybackslash }b{#1}}
\newcolumntype{C}[1]{>{\centering\arraybackslash }b{#1}}

\renewcommand{\headrulewidth}{1pt}
\fancyhead[C]{}
\fancyhead[L]{L3 - 2022/2023}
\fancyhead[R]{D.E.R Informatique}

\renewcommand{\footrulewidth}{1pt}
\fancyfoot[C]{\thepage} 
\fancyfoot[L]{Sacha Ben-Arous}
\fancyfoot[R]{E.N.S Paris-Saclay}

\begin{document}
\begin{itemize}
\item[•] Résidus quadratiques
\item[•] Schwartz–Zippel lemma
\item[•] Suites de Cauchy + CV ssi existe extraite CV + complétude ssi cv abs $\Rightarrow$ cv simple 
\item[•] Sommes exponentielles (théorie analytique des nombres) (cf maths D 2022).
\item[•] Dunford Newton
\item[•] Convergence des séries aléatoires
\item[•] Riemann intégrable ssi discontinuités négligeables
\item[•] Intérieur et adhérence des diagonalisables
\item[•] Fractales/Cantor
\item[•] Complétude des parties compacts d'un complet pour distance de Hausdorff
\item[•] Ascoli-Arzela ; Dini ; Sard (en dim 1) ; Brouwer ; Principe du maximum
\item[•] Cv abs donne cvs ssi Banach
\item[•] Dénombrement Wigner
\item[•] Fourier (par ex fonctions de Weierstrass dérivable nul part avec TF)
\item[•] Lemme de Serre
\item[•] Decomp polaire exo max trace de O*M
\item[•] Gronwall normal puis intégral puis application edo
\item[•] Rayon spectral et normes subordonnées
\item[•] Tout hyperplan rencontre $\text{GL}_n$
\item[•] Vandermonde sans une colonne
\item[•] Fonctions monotones discontinuités au plus dénombrables + application TMI
\item[•] Polnyomes cyclotomiques, Eisenstein
\item[•] Somme des 1/p diverge
\item[•] Méthode de la phase stationnaire
\item[•] Sous-groupe des inversibles d'un corps est cyclique
\item[•] Relèvement et systèmes dynamiques ?
\item[•] Second concours facile ?
\end{itemize}

\end{document}