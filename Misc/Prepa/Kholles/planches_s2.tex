\documentclass[11pt,a4paper]{article}
\textheight245mm
\textwidth170mm
\hoffset-21mm
\voffset-15mm
\parindent0pt
\usepackage[utf8]{inputenc}
\usepackage{dsfont}
\usepackage{graphicx}
\usepackage{caption}
\usepackage{fancyhdr}
\usepackage{amsmath,amsfonts,amssymb}
\usepackage[french]{babel}
\usepackage[hidelinks]{hyperref} 
\hypersetup{
  colorlinks   = true,    % Colours links instead of ugly boxes
  urlcolor     = blue,    % Colour for external hyperlinks
  linkcolor    = black,   % Colour of internal links
  citecolor    = black    % Colour of citations
}
\usepackage{../../zephyr}
\pagestyle{fancy}

\usepackage{array,multirow,makecell}
\setcellgapes{4pt}
\makegapedcells
\newcolumntype{R}[1]{>{\raggedleft\arraybackslash }b{#1}}
\newcolumntype{L}[1]{>{\raggedright\arraybackslash }b{#1}}
\newcolumntype{C}[1]{>{\centering\arraybackslash }b{#1}}

\renewcommand{\headrulewidth}{1pt}
\fancyhead[C]{Oraux blancs}
\fancyhead[L]{}
\fancyhead[R]{}

\renewcommand{\footrulewidth}{1pt}
\fancyfoot[C]{\thepage} 
\fancyfoot[L]{Sacha Ben-Arous}
\fancyfoot[R]{E.N.S Paris-Saclay}

\begin{document}


\textbf{Exercice 1 :} 
\begin{enumerate}
\item Soit $(u_n)_{n \in \mathbb{N}}$ telle que $u_n = \displaystyle O(1/n^2)$. Que dire de $\displaystyle \sum_{k=n}^{+\infty} u_k$ ?
\item Montrer que \[\sum_{k=1}^n \frac{1}{k} = \ln{n} + \gamma + O(\frac{1}{n}) \] où $\gamma$ est un réel que l'on ne cherchera pas à calculer. \\
\item On note pour $n \geq 2$,  $u_n := \displaystyle \frac{(-1)^n}{n} \left \lfloor \frac{\ln{n}}{\ln{2}} \right \rfloor $. Justifier la convergence de $\displaystyle \sum_{n \geq 2} u_n$ et calculer sa valeur.\\
\end{enumerate} 


\textbf{Exercice 2 (avec préparation) :} Soient $E$ un e.v.n de dimension quelconque, et $u \in \mathcal{L}(E)$ tel que \[\forall x\in E, \ \| u(x) \| \leq \|x\| \]
Pour tout $n \in \mathbb{N}$, on note \[v_n := \frac{1}{n+1} \sum_{k=0}^n u^k\]
\begin{enumerate}
\item Simplifier $v_n \circ (u - \text{Id})$.
\item Montrer que :  \[\text{Im}(u-\text{Id}) \cap \text{Ker}(u-\text{Id}) = \{0\} \]
\item On suppose que $E$ est de dimension finie, montrer que : \[\text{Im}(u-\text{Id}) \oplus \text{Ker}(u-\text{Id}) = E \]
\item On suppose à nouveau que $E$ est de dimension quelconque. Montrer que si  \[\text{Im}(u-\text{Id}) \oplus \text{Ker}(u-\text{Id}) = E \] alors la suite $(v_n)_{n \in \mathbb{N}}$ converge simplement et l'espace $\text{Im}(u-\text{Id})$ est une partie fermée de $E$.
\item Étudier la réciproque. \\
\end{enumerate}


\textbf{Exercice 3 :} \\
On répète successivement et indépendamment une expérience qui a la même probabilité de réussir que d'échouer. Pour $n \geq 2$, on introduit les événements : \\
$A_n = $ ``On obtient deux succès consécutifs lors des $n$ premières expériences",\\
$B_n = $ ``On obtient le premier couple de succès consécutifs aux rangs $n - 1$ et $n$". \\
On pose $p_n = \mathbb{P}(B_n)$, et $p_1=0$.
\begin{enumerate}
\item Calculer $p_2$, $p_3$ et $p_4$.
\item Pour $n\geq 2$, montrer que : \[\mathbb{P}(A_n) = \sum_{k=1}^n p_k \ \ \ et \ \ \  p_{n+3} = \frac{1}{8}\left ( 1-\sum_{k=1}^n p_k \right ) \]
\item En déduire une relation entre $p_{n+2}$, $p_{n+3}$ et $p_n$ valable pour tout $n \geq 1$.
\item Exprimer le terme général de la suite $(p_n)_{n \in \mathbb{N}}$ \\
\end{enumerate}


\textbf{Exercice 4 (avec préparation) :} \\
On étudie l'équation fonctionnelle : \[(E) : f(2x) = 2f(x) - 2f(x)^2\]
\begin{enumerate}
\item Quelles sont les solutions constantes sur $\mathbb{R}$ ?
\item Soit $h: \mathbb{R} \to \mathbb{R}$. On note $f(x)=xh(x)$ pour tout $x\in \mathbb{R}$. À quelle condition sur $h$ la fonction $f$ est-elle solution de $(E)$ ?
\item On définit par récurrence une suite de fonctions avec $h_0 : x \mapsto 1$ et, pour tout $n \in \mathbb{N}$ : 
\[h_{n+1}(x) = h_n\left (\frac{x}{2} \right ) - \frac{x}{2}\left ( h_n \left ( \frac{x}{2} \right ) \right )^2\]
Pour $x\in [0;1]$, soit $ \displaystyle T_x : y \mapsto y - \frac{xy^2}{2}$. Montrer que $T_x$ est 1-lipschitzienne sur [0;1] et que $T_x([0;1]) \subset [0;1]$. En déduire que $(h_n)$ converge uniformément sur $[0;1]$.
\item Montrer que l'équation $(E)$ admet une solution continue et non constante sur [0;1].
\item Montrer que l'équation $(E)$ admet une solution continue et non constante sur $\mathbb{R}_+$ \\
\end{enumerate}


\textbf{Exercice 5 :} \\
Soient $E$ un $\mathbb{C}$-ev de dimension finie, et $u \in \mathcal{L}(E)$. On note $\lambda_1, \dots, \lambda_q$ les valeurs propres de $u$, et $n_1,\dots, n_q$ leurs multiplicités. On suppose que tous les espaces propres de $u$ sont de dimension $1$.
\begin{enumerate}
\item Si $1 \leq i \leq q$ et $0 \leq m \leq n_i$, montrer que le noyau de $\left ( u - \lambda	_i\text{Id} \right)^m$ est de dimension $m$.
\item Soit $F$ un sous-espace vectoriel de $E$ stable par $u$. Montrer qu'il existe un polynôme unitaire $Q$ de $\mathbb{C}[X]$ tel que \[F=\text{Ker}\left(Q(u)\right )\]
\item Montrer que le nombre de sous-espaces de $E$ stables par $u$ est le nombre de diviseurs unitaires de $\chi_u $ dans $\mathbb{C}[X]$. \\
\end{enumerate}


\textbf{Exercice 6 :} \\
Soit $\mathcal{P}$ l'ensemble des nombres premiers et, pour $s>1$, \[\zeta(s) := \sum_{n=1}^{+\infty} n^{-s}\]
\begin{enumerate}
\item Pour quels $\lambda \in \mathbb{R}$, la famile $(\lambda n^{-s})_{n \in \mathbb{N}^*}$ définit-elle une loi de probabilité sur $\mathbb{N^*}$ ?
\item Pour $p \in \mathcal{P}$, on note $A_p := p\mathbb{N}^*$. Montrer que les $\displaystyle (A_p)_{p \in \mathcal{P}}$ sont mutuellement indépendants pour la loi de probabilité précédente.
\item Montrer que \[\zeta(s) = \prod_{p\in \mathcal{P}}\frac{1}{1-p^{-s}}\]
\item Que dire de la sommabilité de la famille $\displaystyle (1/p)_{p \in \mathcal{P}}$ ?
\end{enumerate}
~ \\
\textbf{Exercice bonus 1 :} \\
Montrer que tout sous-groupe fini des inversibles d'un corps commutatif est cyclique. \\


\textbf{Exercice bonus 2 :} \\
Montrer qu'une matrice $M \in \mathcal{M}_n(\mathbb{K})$ est nilpotente si et seulement si $ \forall k \in \mathbb{N}^*, \ \text{Tr}(M^k)=0$.
\end{document}
