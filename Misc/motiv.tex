\documentclass[12pt,a4paper]{article}

\usepackage{../jedusor}
	
\renewcommand{\headrulewidth}{1pt} 
\renewcommand{\footrulewidth}{1pt}	
\fancyhead[C]{}
\fancyhead[L]{}
\fancyhead[R]{}
\fancyfoot[C]{\thepage} 
\fancyfoot[L]{Sacha Ben-Arous}
\fancyfoot[R]{E.N.S Paris-Saclay}

%je veux faire de la recherche et enchainer sur une thèse
%D'une part sys dyn (burguet) et les cours qu'ils proposent
%D'autre part, et surtout, EDP, analyse, etc ... citer les cours qui m'intéressent

\begin{document}

Madame, Monsieur, \\

Ma passion pour les mathématiques a débuté dès mon plus jeune âge. Elle s’est concrétisée durant mes années de collège et de lycée par la participation à des concours comme le concours Kangourou au collège, ou bien les olympiades académiques de mathématiques en classe de première, pour lesquelles j’ai été classé premier.  Cette passion m’a naturellement conduit à poursuivre mon parcours en classe préparatoire dans l’objectif d’intégrer une école d’ingénieurs.  \\

J'ai alors eu le plaisir de découvrir des mathématiques beaucoup plus profondes que celles de lycée, ainsi que l’imagination, la curiosité, la persévérance, et l’abnégation qui caractérisent le travail acharné nécessaire pour maitriser un sujet et s'approprier un thème. J'ai donc décidé de tenter le concours des Écoles Normales car la recherche me semblait être la voie idéale pour continuer à faire des mathématiques intéressantes. \\

N’ayant été admis sur dossier qu’au département d’informatique de l’ENS Paris-Saclay, j’ai souhaité découvrir si cette discipline pouvait m’apporter une satisfaction comparable à celle que je trouvais dans les mathématiques. Ce ne fut pas le cas. En parallèle, j’ai suivi des cours de mathématiques particulièrement stimulants, ce qui m’a conduit à présenter — avec succès — le second concours du département de mathématiques, afin d’y poursuivre mes études en tant que normalien fonctionnaire.\\

Ce parcours m’a permis d’élargir mon horizon mathématique, notamment au cours de la première année de master, où j’ai suivi des enseignements remarquables en systèmes dynamiques, équations aux dérivées partielles d’évolution, ou encore en théorie spectrale. Grâce à leurs interactions profondes avec notre compréhension du monde physique, ces domaines m’ont définitivement convaincu de poursuivre mon parcours mathématique dans cette direction.\\

Ainsi, je souhaiterais faire une thèse et poursuivre une carrière dans la recherche mathématique, et c'est dans cette optique que je postule à votre master. D'une part, les cours de systèmes dynamiques que vous proposez me semblent dans la continuité du cours que j'ai suivi cette année, et d'autre part je suis particulièrement intéressé par les cours d'analyse microlocale et d'étude des singularités des équations aux dérivées partielles. J'ai par ailleurs déjà pu explorer ces domaines durant mes stages de recherche : le premier, dirigé par Thomas Alazard, portait sur l'utilisation des opérateurs paradifférentiels appliqués aux systèmes dynamiques, en particulier à la théorie KAM, et le second, que je mène actuellement sous la direction de Tristan Buckmaster, concerne la formation de singularités auto-similaires en mécanique des fluides. \\

Ainsi, intégrer votre master serait un tremplin idéal vers le monde de la recherche en me permettant de suivre des cours qui me passionnent, et d'obtenir une thèse avec un(e) chercheur de haut niveau dans l'un de ces domaines.

\end{document}
