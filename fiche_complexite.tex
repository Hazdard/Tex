\documentclass[10pt,a4paper]{article}
\textheight245mm
\textwidth170mm
\hoffset-21mm
\voffset-15mm
\parindent0pt
\usepackage[utf8]{inputenc}
\usepackage{amsmath,amsfonts,amssymb}
\usepackage{dsfont}
\usepackage{graphicx}
\usepackage{caption}
\usepackage{fancyhdr}
\pagestyle{fancy}

\renewcommand{\headrulewidth}{1pt}
\fancyhead[C]{Fiche Complexité}
\fancyhead[L]{L3 - 2022/2023}
\fancyhead[R]{D.E.R Informatique}

\renewcommand{\footrulewidth}{1pt}
\fancyfoot[C]{\thepage} 
\fancyfoot[L]{Sacha Ben-Arous}
\fancyfoot[R]{E.N.S Paris-Saclay}

\begin{document} 



\section{Mise en place de la théorie :}

\ \ \ - Un \textit{problème} est une fonction $P$ de $A$ dans $B$. Un élément de $A$ est appelée
une \textit{instance} du problème. Lorsque $B$ est le domaine des booléens, on parle de
problème de décision. \\

\ \ \ - Les complexités que l'on va calculer dépendent de la représentation des entrées. Par exemple, si l'on représente les entiers en binaire, l'espace est en $log(n)$ pour une entrée $n$. \\

\ \ \ - Dans tout ce qui suit, on travaillera avec des machines de Turing I/O à $k$-bandes. \\

\ \ \ - Un language $L$ appartient à : 
\begin{enumerate}
\item[--] TIME($f$) (resp. NTIME($f$)), s'il existe une machine de Turing déterministe (resp. non déterministe) opérant en temps au plus $f(n)$ sur tout mot de longueur $n$ dont le language est $L$.
\item[--] SPACE($f$) (resp. NSPACE($f$)), s'il existe une machine de Turing déterministe (resp. non déterministe) opérant en temps au plus $f(n)$ sur tout mot de longueur $n$ dont le language est $L$.
\item[--] LOGSPACE = SPACE(log$_2$($n$)) ; NLOGSPACE = NSPACE(log$_2$($n$)) 
\item[--] P = PTIME = $\bigcup_{k \in \mathbb{N}}$TIME($n^k$) ; NP = NPTIME = $\bigcup_{k \in \mathbb{N}}$NTIME($n^k$)  
\item[--] PSPACE =  $\bigcup_{k \in \mathbb{N}}$SPACE($n^k$) ; NPSPACE = $\bigcup_{k \in \mathbb{N}}$NSPACE($n^k$)
\item[--] EXPTIME = $\bigcup_{k \in \mathbb{N}}$TIME($2^{n^k}$) = EXP ; NEXPTIME = $\bigcup_{k \in \mathbb{N}}$NTIME($2^{n^k}$) = NEXP 
\end{enumerate}



\section{Théorèmes :}
\ \ \ \textbf{Théorème : }\textit{Pour toute fonction $f$, on a les inclusions suivantes :}
\begin{enumerate}
\item[-] TIME($f$) $\subseteq$ NTIME($f$) ; SPACE($f$) $\subseteq$ NSPACE($f$)
\item[-] TIME($f$) $\subseteq$ SPACE($f$) ; NTIME($f$) $\subseteq$ NSPACE($f$)
\item[-] NSPACE($f$) $\subseteq$ TIME($2^{O(f+log)}$) ; NLOGSPACE($f$) $\subseteq$ PTIME($f$)
\end{enumerate}

\ \ \ \textbf{Théorème de compression linéaire:} \textit{Pour toute fonction $f$ et pour tout entier $k$, on a les inclusions suivantes :}
\begin{enumerate}
\item[-] SPACE($kf$) $\subseteq$ SPACE($f$)
\item[-] NSPACE($kf$) $\subseteq$ NSPACE($f$)
\end{enumerate}
\ \ \ \textbf{Théorème d'accélération linéaire : } \textit{Pour toute fonction $f$ telle que $\forall n \in \mathbb{N}, \ f(n) \geq n$, on a l'inclusion suivante : }
\[\text{NTIME}(f) \subseteq \text{SPACE}(f)\] 
\ \ \ \textbf{Théorème de Savitch : }
\[\text{NPSPACE} \subseteq \text{PSPACE}\] 
\ \ \ \textbf{Théorème d'Immerman-Szelepscényi : }
\[\text{NLOGSPACE} = \text{coNLOGSPACE}\] 
\ \ \ \textbf{Théorème de hiérarchie stricte : }
\[\text{NLOGSPACE} \subsetneq \text{PSPACE}\] 
\section{Réductions :}
\ \ \ - Une \textit{réduction} d'un problème de décision $P : A \to \{ \textbf{false}, \textbf{true} \}$ à un problème $P' : A' \to \{ \textbf{false}, \textbf{true} \}$ est une fonction calculable $r : A \to A'$ telle que : $\forall a \in A, P'(r(a)) = P(a)$ \\

\ \ \ - Une réduction est dite LOGSPACE (resp. PTIME) si la procédure associée à $r$ est en espace logarithmique (resp. en temps polynomial). \\

\ \ \ - Supposons que $P$ se réduise en temps polynomial (resp. en espace logarithmique) à $P'$ et que $P'$ se réduise en temps polynomial (resp. en espace logarithmique) à $P''$ alors $P$ se réduit en temps polynomial (resp. en espace logarithmique) à $P''$.


\end{document}





   















