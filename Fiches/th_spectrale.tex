\documentclass[11pt,a4paper]{article}

\usepackage{../jedusor}
	
\renewcommand{\headrulewidth}{1pt} 
\renewcommand{\footrulewidth}{1pt}	
\fancyhead[C]{}
\fancyhead[L]{}
\fancyhead[R]{}
\fancyfoot[C]{\thepage} 
\fancyfoot[L]{Sacha Ben-Arous}
\fancyfoot[R]{E.N.S Paris-Saclay}

\begin{document}


\begin{center}  
\section*{Théorie spectrale} 
\end{center}
ayaya
\subsection*{Contexte}
% Motivations : extension morale de la theorie de Fourier. Origine dans la meca q.

\subsection*{Méthode}

\subsection*{Définitions et propriétés élémentaires}

%def opérateur, domaine tjrs dense, prop pour etre un domaine, def spectre, def etre fermé.
%contre exemple : le décalage à droite sur l^2 est injectif mais pas surjectif, 0 est une valeur spectrale mais pas valeur propre.
%Le spectre est fermé.
%si pas fermé alors spectre vaut C
%Impulsion sur L^2(R) et fonctions tests est pas fermé.
%fermeture
%Le spectre n'est jamais vide !
%f(1/2) non fermable
%dans preuve th 2.16, dévisser l'argument de c'est égal au sens des distrib, (donc on problonge car D est dense dans L2, puis riesz donc unique représentation qui est dans L2, donc ok).
%def adjoint avec besoin de fermable (et bien sur domaine dense). Lien EDP ?
% Def symétrique plus inclusion dans l'adjoint.
%Remarque que les extensions symétriques (fermées) sont entre A (\bar{A}) et A^*.
%Lemme : tout opérateur symétrique est fermable, et sa fermeture est symétrique.
% <Au|u> est toujours réel quand A est symétrique car égal à son conjugué.
%Lemme 2.22
%def auto-adjoint, et si symétrique alors condition "faible" pour être auto adjoint.
% Un op auto adjoint n'a aucune extension ou restriction stricte qui est encore auto adjointe.
%PERRON-FROBENIUS.
% L’inégalité de Poincaré c’est relié à la première v.p; du Laplacien sur le domaine.

\subsection*{Résultats principaux}

%Spectre des opérateurs symétriques, et bien dire en remarque que A-z est tjrs injectif quand im z non nul, i.e pas de valeur propre complexe. Au pire, ce n'est pas surjectif, mais si ça l'est, alors l'inverse exist et est borné.

\subsection*{Outils importants}


\subsection*{Autres résultats}
\end{document}
