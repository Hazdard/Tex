\documentclass[12pt,a4paper]{article}

\usepackage{../jedusor}	

\renewcommand{\headrulewidth}{0pt} 
\renewcommand{\footrulewidth}{1pt}
\fancyhead[C]{}
\fancyhead[L]{}
\fancyhead[R]{}
\fancyfoot[C]{\thepage} 
\fancyfoot[L]{Sacha Ben-Arous}
\fancyfoot[R]{E.N.S Paris-Saclay}
	
\begin{document}
\newpage
\begin{center}  
\section*{Topologie} 
\end{center}


%Topologie : complet ssi cv abs donne cvs ;  Urysohn ; bif continue sur compact = homéo ; Riesz boules compactes ; BAIRE et rmq sur R[X];  Brouwer ; Invariance du domaine ; Ascoli + Fréchet Kolmogorov ;  projection sur un convexe fermé, représentation de Riesz ; compact ssi B-L ssi B-W ssi precompact complet ; topologie faible, hahn banach, banach alaoglu.
%Krein–Milman theorem (analyse 3 page 100) ou preuve one shot avec Hahn-Banach à mettre dans bestiaire.
% Dire que liminf limsup est la plus petite/plus grand valeur d'adhérence.
% Dire que la complétude n'est pas une notion topologique mais une notion métrique. Mettre l'extraction diagonale en méthode avec des exemples.
% Unique valeur d'adhérence dans un compact => cv vers cette v.a. ; L'ensemble des v.a. est un compact.
%Preuve Brouwer par Stokes (et du coup preuve formule de Stokes/Green).
%On peut montrer la continuité en regardant composante par composante car la topologie produit est engendrée par les pavés.


\subsection*{Contexte et méthode}

%Se rammener a prouver les props sur des trucs engendrant tout le monde ; un pi-syst ; indicatrices ; base de topo/voisinage ; partie dense ; etc ...

\subsection*{Définitions et propriétés élémentaires}
\begin{definstar}
Soit $E$ un ensemble, et $\mathcal{T} \subset \mathcal{P}(E)$.  On dit que $(E,\mathcal{T})$ est un espace topologique si :
\begin{enumerate}
\item $\emptyset \in \mathcal{T}$, $E \in \mathcal{T}$,
\item $\mathcal{T}$ est stable par union quelconque,
\item $\mathcal{T}$ est stable par intersection finie.
\end{enumerate}
Les éléments de $\mathcal{T}$ sont appelés des ouverts, et leurs complémentaires sont des fermés.
\end{definstar}
\begin{rmq}
Une intersection de topologies étant encore une topologie, on en déduit la validité de la définition suivante.
\end{rmq}
\begin{definstar}
Soit $A \subset \mathcal{P}(E)$, on note $\mathcal{T}(A)$ la plus petite topologie contenant $A$, dite \textit{topologie engendrée}. Cette topologie est obtenue en prenant les unions quelconques d'intersections finies
d'éléments de A.
\end{definstar}

\begin{definstar}
Soit $(E,\mathcal{T})$ un espace topologique, on dit que $\mathcal{B}\subset \mathcal{T}$ est une base si tout élément de $\mathcal{T}$ s'écrit comme union d'éléments de $\mathcal{B}$.
\end{definstar}
\begin{rmq}
Attention, en général $A$ n'est pas une base de $\mathcal{T}(A)$ !
\end{rmq}

\subsection*{Résultats principaux}

\subsection*{Outils importants}

\subsection*{Autres résultats}


\newpage
\begin{center}  
\section*{Topologie algébrique} 
\end{center}



\subsection*{Contexte}

\subsection*{Méthode}

\subsection*{Définitions et propriétés élémentaires}

\subsection*{Résultats principaux}

\subsection*{Outils importants}


\subsection*{Autres résultats}


\newpage
\begin{center}  
\section*{Géométrie différentielle} 
\end{center}

%REVOIR TRUC DIFFÉRENTIELLE COMPOSÉE OU PERSONNE PEUT JUSTIFIER
%Extremas liés cf L3 
%Curl = 0 implique que t’es un gradient
\subsection*{Contexte}

\subsection*{Méthode}

\subsection*{Définitions et propriétés élémentaires}

\subsection*{Résultats principaux}

\subsection*{Outils importants}


\subsection*{Autres résultats}
\end{document}
