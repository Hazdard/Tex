\documentclass[11pt,a4paper]{article}

\usepackage{../../jedusor}	

\renewcommand{\headrulewidth}{0pt} 
\renewcommand{\footrulewidth}{1pt}
\fancyhead[C]{}
\fancyhead[L]{}
\fancyhead[R]{}
\fancyfoot[C]{\thepage} 
\fancyfoot[L]{Sacha Ben-Arous}
\fancyfoot[R]{E.N.S Paris-Saclay}

\parindent0pt

%Penser à : enveloppe convexe ; dualité ; entropie ; minimisation de l'énergie \\
%Ajoute : Gronwall \\
%Faire un chapitre topologie et y mettre complet ssi cv abs donne cvs ;  Urysohn ; Riesz boules compactes ; BAIRE et rmq sur R[X];  Brouwer ; Ascoli ; Dini
%Faire un chapitre algèbre linéaire, et y mettre Cauchy-Schwarz, décomposition polaire, min-max Courant-Fischer, Gershgorin, rayon spectral 
%Mettre les contre-exemples classiques

% VÉRIFIER ORDRE TMI ENTRE SHORT ET LONG
%
%Probas : 
%Chap 6 : Jensen ;  Injectivité de Fourier ; (Bonus : les classiques Markov, Bienaymé-Chebyshev )
%
%Chap 7 : Coalitions ; WLLN ; 
%
%Chap 8 : Borel-Cantelli ; Loi du 0-1 ; SLLN ; TCL (et Levy par la même occasion !)

\begin{document}
\begin{center}
\section*{Théorie de la mesure} 
\end{center}
~\\
\begin{enumerate}
\item Théorème de la classe monotone (Dynkin).
\item Théorèmes de convergence monotone, convergence dominée.
\item Régularité des mesures.
\item Représentation de Riesz-Markov.
\item Inégalité de Hölder, inégalité de Young.
\item Complétude des $L^p$.
\item Densité dans les $L^p$.
\item Dérivée de Radon-Nikodym et décomposition de Lebesgue.
\item Fubini, Changement de variable.
\end{enumerate}


\subsection*{Outils :}
\begin{itemize}
\item[•] Lemme de Fatou.
\item[•] Régularité des intégrales à paramètre.
\item[•] Mesure de Stieltjes.
\item[•] Mesures extérieures, critère de Carathéodory.
\item[•] Points de Lebesgue.
\item[•] Recouvrement de Vitali.
\item[•] Dualité $L^p$-$L^q$.
\end{itemize}



\newpage\begin{center}
\section*{Probabilités} 
\end{center}
~\\
\begin{enumerate}
\item Inégalité de Jensen
\item Inégalité de Markov
\end{enumerate}


\subsection*{Outils :}
\begin{itemize}
\item[•] Lemme de Fekete
\end{itemize}



\newpage
\begin{center}
\section*{Algèbre} 
\end{center}

\subsection*{Outils :}
- $\mathbb{Z}/p\mathbb{Z}$ espace vectoriel quand $p$ premier et abélien. \\
- Faire agir le centre, ou par conjuguaison




\newpage
\begin{center}
\section*{Analyse complexe} 
\end{center}

1- Prolongement analytique \\

2- Formules et inégalités de Cauchy \\

3- Résidus \\

4- Représentation conforme de Riemann

~\\

\subsection*{Outils :}
- Cauchy-Riemann \\

- Formule d'homologie \\

- Principe du maximum \\

- Morera \\

- Holo sur une couronne $\Leftrightarrow$ Dev en Fourier sur une bande \\

- Lemme de Schwarz \\


\end{document}
