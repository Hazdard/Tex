\documentclass[11pt,a4paper]{article}

\usepackage{../../jedusor}	

\renewcommand{\headrulewidth}{0pt} 
\renewcommand{\footrulewidth}{1pt}
\fancyhead[C]{}
\fancyhead[L]{}
\fancyhead[R]{}
\fancyfoot[C]{\thepage} 
\fancyfoot[L]{Sacha Ben-Arous}
\fancyfoot[R]{E.N.S Paris-Saclay}

\parindent0pt

\begin{document}
Penser à : enveloppe convexe ; dualité \\
Ajoute : Cauchy-Schwarz ; min-max Courant-Fischer ; Gershgorin ; rayon spectral ; Gronwall \\
Faire un chapitre topologie et y mettre Urysohn.


\newpage
\begin{center}
\section*{Théorie de la mesure} 
\end{center}
~\\
\begin{enumerate}
\item Théorème de la classe monotone (Dynkin).
\item Théorèmes de convergence monotone, convergence dominée.
\item Représentation de Riesz-Markov
\item Régularité des mesures
\item Dérivée de Radon-Nikodym
\item Fubini, Changement de variable
\item Complétude des $\displaystyle L^p$
\end{enumerate}


\subsection*{Outils :}
\begin{itemize}
\item[•] Lemme de Fatou
\item[•] Points de Lebesgue
\item[•] Recouvrement de Vitali
\end{itemize}



\newpage
\begin{center}
\section*{Algèbre} 
\end{center}

\subsection*{Outils :}
- $\mathbb{Z}/p\mathbb{Z}$ espace vectoriel quand $p$ premier et abélien. \\
- Faire agir le centre, ou par conjuguaison




\newpage
\begin{center}
\section*{Analyse complexe} 
\end{center}

1- Prolongement analytique \\

2- Formules et inégalités de Cauchy \\

3- Résidus \\

4- Représentation conforme de Riemann

~\\

\subsection*{Outils :}
- Cauchy-Riemann \\

- Formule d'homologie \\

- Principe du maximum \\

- Morera \\

- Holo sur une couronne $\Leftrightarrow$ Dev en Fourier sur une bande \\

- Lemme de Schwarz \\


\end{document}
