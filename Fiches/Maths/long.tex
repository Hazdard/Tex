\documentclass[11pt,a4paper]{article}

\usepackage{../../jedusor}	

\renewcommand{\headrulewidth}{0pt} 
\renewcommand{\footrulewidth}{1pt}
\fancyhead[C]{}
\fancyhead[L]{}
\fancyhead[R]{}
\fancyfoot[C]{\thepage} 
\fancyfoot[L]{Sacha Ben-Arous}
\fancyfoot[R]{E.N.S Paris-Saclay}
	
\begin{document}
\newpage
\begin{center}
\section{Théorie de la mesure} 
\end{center}


\subsection*{Méthode}

\begin{itemize}
\item[-] Toujours se demander : dans quel ensemble je travaille, quelle est la tribu, quelle est la mesure ? 
\item[-] Utiliser les indicatrices, raisonner par densité, par régularité.
\end{itemize}

\subsection*{Théorèmes}
Dans tout ce qui suit, on travaille dans un espace mesuré $(E,\mathcal{A},\mu)$, et on pourra utiliser un espace métrique $(T,d)$.


\begin{thm}[Dynkin] 
Le $\sigma$-système engendré par un $\pi$-système est égal à la tribu engendrée par ce dernier.
\end{thm}


\begin{thm}[Convergence monotone]
Soit $(f_n)_{n\in\N}$ une suite de fonctions mesurables de $E$ dans $[0,+\infty]$, telle que pour tout $n\geq 0$, $f_n \leq f_{n+1}  $. Alors, en notant $f$ la limite simple de cette suite, on a que $f$ est mesurable, et : 
\[\int f = \lim\limits_{n \to +\infty} \int f_n \]
\end{thm}


\begin{thm}[Convergence dominée]
Soit $(f_n)_{n\in\N}$ une suite de fonctions mesurables qui converge simplement vers $f$. Si il existe $g$ intégrable telle que pour tout $n \geq 0$, $\left| f_n \right| \leq g$, alors $f$ est intégrable et :
\[\int f = \lim\limits_{n \to +\infty} \int f_n \]
\end{thm}


\begin{rmq}
On peut seulement supposer les conditions ci-dessus vraies presque partout, mais dans ce cas il faut imposer la mesurabilité de la limite simple, ou bien travailler dans la tribu complétée, comme l'explique la remarque suivante.
\end{rmq}


\begin{rmq}
Si $\tilde{f} : E \mapsto F$ est presque partout égale à une fonction mesurable $f : (E,\mathcal{A}) \mapsto (F,\mathcal{B})$, alors $\tilde{f}$ est mesurable pour la tribu complétée $\overline{\mathcal{A}}$ sur $E$.
\end{rmq}

\begin{thm}[Régularité des mesures boréliennes finies]
Si $\mu$ est une mesure borélienne finie sur un espace métrique $(T,d)$, alors $\mu$ est régulière, i.e pour tout $A \in \mathcal{B}(T)$:
\begin{align*}
\mu(A) &= \inf \left\{ \mu(U) \ \setbar \ U \text{ ouvert},\  A \subset U \right\} \\
&=\sup \left\{ \mu(F) \ \setbar \ F \text{ fermé}, \ F \subset A \right\}
\end{align*}
\end{thm}


\begin{rmq}
Dans un espace polonais (métrique séparable complet), on a de plus que la régularité intérieure est sur les compacts.
\end{rmq}

\begin{thm}[Régularité des mesures de Radon]
Si $\mu$ est une mesure de Radon sur un espace métrique localement compact séparable (ex. $\R^d$), alors $\mu$ est régulière, et  la régularité intérieure est sur les compacts. 
\end{thm}

\begin{thm}[Représentation de Riesz-Markov]
Si $E$ est un espace métrique localement compact séparable, et $\Lambda : C_c(E) \mapsto \R$ une forme linéaire positive, alors il existe une unique mesure de Radon $\mu$ sur $E$ telle que, pour tout $f\in C_c(E)$ : 
\[\Lambda(f) = \int f \mathrm{d}\mu\]
\end{thm}

\begin{rmq}
Avec ce qui précède, on obtient de plus que la mesure $\mu$ est régulière.
\end{rmq}

\begin{thm}[Inégalités d'Hölder et de Young]
Soient $ 1 \leq p,q,r \leq +\infty$, $f\in L^p$, $g \in L^q$, on a les inégalités :
\begin{itemize}
\item[•] Si $p$ et $q$ sont des exposants conjugués, i.e : $\frac{1}{p} + \frac{1}{q} = 1$, alors $\|fg\|_{L^1} \leq \|f\|_{L^p}\|g\|_{L^q}$.
\item[•] Si $\frac{1}{p} + \frac{1}{q} = 1 + \frac{1}{r}$, alors $\|f*g\|_{L^r} \leq \|f\|_{L^p}\|g\|_{L^q}$.
\end{itemize}
\end{thm}

\begin{thm}[Riesz-Fischer]
Si $p \in [1; +\infty]$, alors $L^p(E,\mathcal{A},\mu)$ est un espace de Banach.
\end{thm}


\begin{rmq}
La preuve du théorème ci-dessus donne en particulier que la convergence en norme $L^p$ implique l'existence d'une extraite qui converge presque partout. Le théorème de convergence dominée donne une sorte de réciproque, i.e que la convergence presque partout fournit la convergence $L^p$, sous réserve d'avoir une domination intégrable.
\end{rmq}

\begin{thm}[Densité]
Soit $p \in [1;+\infty[$.
\begin{enumerate}
\item Les fonctions étagées sont denses dans $L^p$. De plus, si $\mathcal{A}$ est généré par une famille dénombrable dense et si $\mu$ est $\sigma$-finie, alors $L^p(E,\mathcal{A}, \mu)$ est séparable.
\item 
\begin{enumerate}
\item Si $(E,d)$ est un espace métrique et $\mu$ une mesure borélienne finie, alors les fonctions lipschitziennes sont denses dans $L^p$.
\item Si de plus $(E,d)$ est localement compact séparable et $\mu$ est une mesure de Radon, alors les fonctions lipschitziennes à support compact sont denses dans $L^p$.
\end{enumerate}
\end{enumerate}
\end{thm}

\begin{rmq}
Si  $(E,d)$ est une métrique séparable, alors $\mathcal{B}(E)$ est dénombrablement généré.
\end{rmq}

\begin{rmq}
Pour le point \textit{2.(a)}, il suffit en fait que la mesure soit extérieurement régulière sur les ouverts, et pour le point \textit{2.(b)}, il suffit qu'elle soit de plus intérieurement régulière sur les compacts.
\end{rmq}


\begin{thm}[Dérivée de Radon-Nikodym]
Soient $\mu$ et $\nu$ deux mesures $\sigma$-finies sur $(E,\mathcal{A})$. Si $\mu$ est absolument continue par rapport à $\nu$, alors il existe $f : E \mapsto \overline{\R}_+$ mesurable tel que $\mu=f\nu$.
\end{thm}

\subsection*{Outils :}


\begin{cor}[Unicité des mesures] Soient $\mu$ et $\nu$ deux mesures sur $(E,\mathcal{A})$ qui coïncident sur un $\pi$-système $\mathcal{C}$ tel que $\mathcal{A}=\sigma(\mathcal{C})$. Alors :
\begin{enumerate}
\item Si $\mu(E)=\nu(E) < +\infty$, alors $\mu = \nu$.
\item Si il existe une suite croissante $(A_n)_{n\in\N}$ d'éléments de $\mathcal{C}$ tels que $\displaystyle \bigcup_n A_n = E$, et que pour tout $n\in\N$, $\mu(A_n)=\nu(A_n)<+\infty$, alors $\mu = \nu$.
\end{enumerate}
\end{cor}


\begin{lemma}[Fatou]
Soit $(f_n)_{n\in\N}$ une suite de fonctions mesurables positives. Alors :
\[\int \varliminf_n f_n \leq \varliminf_n \int f_n\]
\end{lemma}


\begin{prop}[Continuité des intégrales]
Soit $t_0 \in T$  et $f : T \times E \mapsto \overline{\R} $. Si :
\begin{itemize}
\item[•] Pour tout $t\in T$, $x\mapsto f(t,x)$ est mesurable.
\item[•] Pour presque tout $x \in E$, $t\mapsto f(t,x)$ est continue en $t_0$.
\item[•] Il existe $g$ intégrable telle que pour tout $t\in T$, pour presque tout $x \in E$, on a $\left| f(t,x) \right| \leq g(x)$
\end{itemize}
Alors, $t\mapsto \displaystyle \int f(t,x)\mathrm{d}\mu(x)$ est continue en $t_0$.
\end{prop}


\begin{prop}[Dérivation des intégrales]
Soit $I$ un intervalle ouvert de $\R$, $t_0 \in I$, et $f:I \times E \mapsto \overline{\R}$. Si :
\begin{itemize}
\item[•] Pour tout $t\in I$, $x\mapsto f(t,x)$ est mesurable et intégrable.
\item[•] Pour presque tout $x \in E$, $t\mapsto f(t,x)$ est dérivable en $t_0$.
\item[•] Il existe $g$ intégrable telle que pour tout $t\in I$, pour presque tout $x \in E$, on a \[\left| f(t,x) - f(t_0,x) \right| \leq g(x) \left|t-t_0\right|\]
\end{itemize}
Alors, $t\mapsto \displaystyle \int f(t,x)\mathrm{d}\mu(x)$ est dérivable en $t_0$, de dérivée $\displaystyle \int \frac{\partial f}{\partial t} (t_0,x)\mathrm{d}\mu(x)$.
\end{prop}


\begin{rmq}
On peut remplacer les deux dernières conditions par les suivantes, plus fortes, qui assurent alors la dérivabilité en tout point de l'intervalle :
\begin{itemize}
\item[•] Pour presque tout $x \in E$, $t\mapsto f(t,x)$ est dérivable sur tout $I$.
\item[•] Il existe $g$ intégrable tel que pour presque tout $x\in E$, pour tout $t\in I$, on a $\displaystyle \left| \frac{\partial f}{\partial t}(t,x) \right| \leq g(x)$.
\end{itemize}
\end{rmq}


\begin{prop}[Mesure de Stieltjes]
Il y a une correspondance entre les fonctions croissantes continues à droite, et les mesures de Radon. Plus précisemment :
\begin{itemize}
\item[•] Si $\mu$ est une mesure de Radon sur $\R$, alors la fonction $F_{\mu}$ suivante est croissante, continue à droite.
$\displaystyle F_{\mu}(x) := \begin{cases} \mu(]0;x]) & \text{ si } x \geq 0  \\ -\mu(]x;0]) & \text{ si } x < 0 \end{cases}$
\item[•] Si $F : \R \mapsto \R$ est croissante, continue à droite, alors il existe une unique mesure de Radon $\mu$ telle que, pour tout $a<b \in \R$, $\mu(]a;b])=F(b)-F(a)$.
\end{itemize}
\end{prop}

\begin{cor}[Décomposition de Lebesgue]
Si $\mu$ et $\nu$ sont deux mesures $\sigma$-finies sur $(E,\mathcal{A})$, alors il existe une unique décomposition $\nu = \nu_a + \nu_\bot$ en deux mesures telles que $\nu_a$ est absolument continue par rapport à $\mu$, et $\nu_\bot$ est étrangère à $\mu$. 

\end{cor}

\end{document}
