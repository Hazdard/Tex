\documentclass[11pt,a4paper]{article}

\usepackage{../../jedusor}	

\renewcommand{\headrulewidth}{0pt} 
\renewcommand{\footrulewidth}{1pt}
\fancyhead[C]{}
\fancyhead[L]{}
\fancyhead[R]{}
\fancyfoot[C]{\thepage} 
\fancyfoot[L]{Sacha Ben-Arous}
\fancyfoot[R]{E.N.S Paris-Saclay}
	
\begin{document}
\newpage
\begin{center}
\section{Théorie de la mesure} 
\end{center}

\begin{thm}[Dynkin] Le $\pi$-système engendré par un $\sigma$-système est égal à la tribu engendrée par ce dernier.
\end{thm}

\begin{cor}[Unicité des mesures] Soient $\mu$ et $\nu$ deux mesures sur $(E,\mathcal{A})$ qui coïncident sur un $\pi$-système $\mathcal{C}$ tel que $\mathcal{A}=\sigma(\mathcal{C})$. Alors :
\begin{enumerate}
\item Si $\mu(E)=\nu(E) < +\infty$, alors $\mu = \nu$.
\item Si il existe une suite croissante $(A_n)_{n\in\N}$ d'éléments de $\mathcal{C}$ tels que $\displaystyle \bigcup_n A_n = E$, et pour tout $n\in\N$, $\mu(A_n)=\nu(A_n)<+\infty$, alors $\mu = \nu$.
\end{enumerate}
\end{cor}

%\begin{rmq}
%$\mathcal{B}(\R^2) = \mathcal{B}(\R) \otimes \mathcal{B}(\R)$
%\end{rmq}

\subsection*{Outils :}

\end{document}
