\documentclass[11pt,a4paper]{article}

\usepackage{../../jedusor}	

\renewcommand{\headrulewidth}{0pt} 
\renewcommand{\footrulewidth}{1pt}
\fancyhead[C]{}
\fancyhead[L]{}
\fancyhead[R]{}
\fancyfoot[C]{\thepage} 
\fancyfoot[L]{Sacha Ben-Arous}
\fancyfoot[R]{E.N.S Paris-Saclay}
	
\begin{document}
\newpage
\begin{center}
\section{Théorie de la mesure} 
\end{center}
Dans tout ce qui suit, on travaille dans un espace mesuré $(E,\mathcal{A},\mu)$, et on pourra utiliser un espace métrique $(T,d)$.\\ 

\begin{thm}[Dynkin] Le $\pi$-système engendré par un $\sigma$-système est égal à la tribu engendrée par ce dernier.
\end{thm}


\begin{thm}[Convergence monotone]
Soit $(f_n)_{n\in\N}$ une suite de fonctions mesurables de $E$ dans $[0,+\infty]$, telle que pour tout $n\geq 0$, $f_{n+1} \geq f_n$. Alors, en notant $f$ la limite simple de cette suite, on a que $f$ est mesurable, et : 
\[\int f = \lim\limits_{n \to +\infty} \int f_n \]
\end{thm}


\begin{thm}[Convergence dominée]
Soit $(f_n)_{n\in\N}$ une suite de fonctions mesurables qui converge simplement vers $f$. Si il existe $g$ mesurable telle que pour tout $n \geq 0$, $\left| f_n \right| \leq g$, alors $f$ est intégrable et :
\[\int f = \lim\limits_{n \to +\infty} \int f_n \]
\end{thm}


\begin{rmq}
On peut seulement supposer les conditions ci-dessus vraies presque partout, mais dans ce cas il faut imposer la mesurabilité de la limite simple, ou bien travailler dans la tribu complétée.
\end{rmq}

\subsection*{Méthode :}

Toujours se demander : dans quel ensemble je travaille, quelle est la tribu, quelle est la mesure ?

\subsection*{Outils :}


\begin{cor}[Unicité des mesures] Soient $\mu$ et $\nu$ deux mesures sur $(E,\mathcal{A})$ qui coïncident sur un $\pi$-système $\mathcal{C}$ tel que $\mathcal{A}=\sigma(\mathcal{C})$. Alors :
\begin{enumerate}
\item Si $\mu(E)=\nu(E) < +\infty$, alors $\mu = \nu$.
\item Si il existe une suite croissante $(A_n)_{n\in\N}$ d'éléments de $\mathcal{C}$ tels que $\displaystyle \bigcup_n A_n = E$, et que pour tout $n\in\N$, $\mu(A_n)=\nu(A_n)<+\infty$, alors $\mu = \nu$.
\end{enumerate}
\end{cor}


\begin{lemma}[Fatou]
Soit $(f_n)_{n\in\N}$ une suite de fonctions mesurables positives. Alors :
\[\int \varliminf_n f_n \leq \varliminf_n \int f_n\]
\end{lemma}


\begin{prop}[Continuité des intégrales]
Soit $t_0 \in T$  et $f : T \times E \mapsto \overline{\R} $. Si :
\begin{itemize}
\item[•] Pour tout $t\in T$, $x\mapsto f(t,x)$ est mesurable.
\item[•] Pour presque tout $x \in E$, $t\mapsto f(t,x)$ est continue en $t_0$.
\item[•] Il existe $g$ intégrable telle que pour tout $t\in T$, pour presque tout $x \in E$, on a $\left| f(t,x) \right| \leq g(x)$
\end{itemize}
Alors, $t\mapsto \displaystyle \int f(t,x)\mathrm{d}\mu(x)$ est continue en $t_0$.
\end{prop}


\begin{prop}[Dérivation des intégrales]
Soit $I$ un intervalle ouvert de $\R$, $t_0 \in I$, et $f:I \times E \mapsto \overline{\R}$. Si :
\begin{itemize}
\item[•] Pour tout $t\in T$, $x\mapsto f(t,x)$ est mesurable et intégrable.
\item[•] Pour presque tout $x \in E$, $t\mapsto f(t,x)$ est continue en $t_0$.
\item[•] Il existe $g$ intégrable telle que pour tout $t\in T$, pour presque tout $x \in E$, on a $\left| f(t,x) \right| \leq g(x)$
\end{itemize}
Alors, $t\mapsto \displaystyle \int f(t,x)\mathrm{d}\mu(x)$ est continue en $t_0$.
\end{prop}


\end{document}
