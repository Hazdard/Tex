\documentclass[11pt,a4paper]{article}

\usepackage{../jedusor}
	
\renewcommand{\headrulewidth}{1pt} 
\renewcommand{\footrulewidth}{1pt}	
\fancyhead[C]{}
\fancyhead[L]{}
\fancyhead[R]{}
\fancyfoot[C]{\thepage} 
\fancyfoot[L]{Sacha Ben-Arous}
\fancyfoot[R]{E.N.S Paris-Saclay}

\begin{document}

\newpage
\begin{center}  
\section*{Équations différentielles} 
\end{center}

ayaya

%EDP :
%Preuve Feller-Miyadera-Phillips
%Comprendre exemples du cours (voir poly pour peut-être plus de détails).
% Mettre qqpart avec preuve le lemme sur la dérivée d'une distrib nulle => distrib cste

% Analyse (Hilbertienne) et EDOs/EDPs :  Résultats prépa sur EDO linéaire avec wronskien, variation de la constante, formule de Duhamel !!!!!!, etc ; Cauchy-Lipschitz (voir Brézis p199 pour version générale), Gronwall, Lax-Milgram + Stampaccia, Fourier, Cours d'EDP, Céa, densité C infini à support compact dans Lp et Sobolev.  
% Lemme d'explosion (cf Majda Bertozzi p103 Theoreme 3.3 avec ref Hartman).
%Variation de la constante en EDO
%Formule de Duhamel
%Les distributions en analyse/EDP jouent le même rôle que les indicatrices en théorie de la mesure.
%Le Laplacien apparait souvent dans les EDP car on est au voisinage d'un point d'équilibre, donc la dérivée première s'annule et en première approximation on est en équilibre dans un potentiel harmonique.
% Lire le bouquin de Tao "Dispersive equations" en particulier la partie 1, Cauchy-Kowalevski, Boostrap arguments, Noether invariance, Duhamel formula, monotonocity formula, etc ... 
% According to Tristan, le bootstrap trivialise plein de trucs ! Mettre les bons exemples physiques issus du bouquin de Tao.

\subsection*{Contexte}

\subsection*{Méthode}

\subsection*{Définitions et propriétés élémentaires}

\subsection*{Résultats principaux}

\subsection*{Outils importants}


\subsection*{Autres résultats}

\end{document}
