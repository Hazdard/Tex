\documentclass[11pt,a4paper]{article}

\usepackage{../jedusor}	

\renewcommand{\headrulewidth}{0pt} 
\renewcommand{\footrulewidth}{1pt}
\fancyhead[C]{}
\fancyhead[L]{}
\fancyhead[R]{}
\fancyfoot[C]{\thepage} 
\fancyfoot[L]{Sacha Ben-Arous}
\fancyfoot[R]{E.N.S Paris-Saclay}
	
\begin{document}
\newpage
\begin{center}  
\section*{Analyse réelle} 
\end{center}


%Analyse : Théorème fondamental de l'analyse, formules de Taylor, TAF, TVI, Heine, Dini, bij continue sur R homéo, théorèmes de limite de prépa (série de fonctions normalement cv on peut mettre la limite de continuité dans la somme)(et autres résultats sur les séries et les modes de convergence)(critère séries alternées, Stirling (preuve ?, dualité suite-série), Stone-Weierstrass ; double limite, cv uniforme de la suite des dérivées implique limite dérivable, comparaison série-intégrale, transformée d'Abel et critère de CV qui va avec, équivalent logarithmique cf exo sinus,  etc ...), Picard, Inversion locale, fonctions implicites, rang constant et la remarque qui va avec, bij continue sur un compact est un homéo (Poincaré), série de Neumann (= ouverture de GL(E) et dire que |T-Id|<1 donne T inversible et son inverse en série), C^k homéo est un C^k difféo, théorèmes d'analyse fonctionnelle (peut-être dans une autre section) ; Les théorèmes de prolongement (dense + complet ; prépa prolongement de la dérivée) ; Principe du maximum (laplacien).
% Dire que le seul but de l'analyse c'est de faire converger des suites, de trouver des points fixes, et d'utiliser des transformées et avoir des inégalités. En général on doit deviner la limite, si complet alors Cauchy miracle, etc ...
% Analyse (dans les calculs) = cauchy-schwarz ; gronwall ; holder ; Analyse (dans les idées) : point fixe, transformées ; inégalités (convexité ou autre) ;
%Dire les 3 grands usages de la complétude (cf Villani), rajouter sommes de Neumann
%Moralement l'analyse c'est toujours pareil : on fais les cas cools régulier, puis on tronque et on régularise pour s'y rammener par densité.


\subsection*{Contexte}

\subsection*{Méthode}

\subsection*{Définitions et propriétés élémentaires}

\subsection*{Résultats principaux}

\subsection*{Outils importants}


\subsection*{Autres résultats}

\end{document}