\documentclass[11pt,a4paper]{article}

\usepackage{../jedusor}	

\renewcommand{\headrulewidth}{0pt} 
\renewcommand{\footrulewidth}{1pt}
\fancyhead[C]{}
\fancyhead[L]{}
\fancyhead[R]{}
\fancyfoot[C]{\thepage} 
\fancyfoot[L]{Sacha Ben-Arous}
\fancyfoot[R]{E.N.S Paris-Saclay}
	
\begin{document}
\newpage
\begin{center}  
\section*{Topologie algébrique} 
\end{center}



\subsection*{Contexte}

\subsection*{Méthode}

\subsection*{Définitions et propriétés élémentaires}

Un espace topologique est simplement connexe s’il est connexe par arcs et si toute
application continue du cercle S1 dans X se prolonge (continuement) en une application
continue du disque B2 dans X, ou, de manière équivalente si toute application continue de
S1 dans X est homotope à une application constante de S1 dans X

Soient X et B deux espaces topologiques. Une application continue
f ∶ X →B est un revêtement si et seulement si tout point y de B admet un voisinage
V tel que f−1(V)= ⋃i∈IUi soit une union disjointe non vide d’ouverts Ui de X tels que
f∣Ui
∶Ui →V soit un homéomorphisme.

\subsection*{Résultats principaux}

\subsection*{Outils importants}


\subsection*{Autres résultats}

\end{document}