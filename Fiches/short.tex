\documentclass[12pt,a4paper]{article}

\usepackage{../jedusor}	

\renewcommand{\headrulewidth}{0pt} 
\renewcommand{\footrulewidth}{1pt}
\fancyhead[C]{}
\fancyhead[L]{}
\fancyhead[R]{}
\fancyfoot[C]{\thepage} 
\fancyfoot[L]{Sacha Ben-Arous}
\fancyfoot[R]{E.N.S Paris-Saclay}

\parindent0pt

%Penser à : enveloppe convexe ; dualité ; entropie ; minimisation de l'énergie 
%Théorème fondamental de l'analyse, de l'algèbre ?
%Mettre des contre-exemples quand c'est intéressant et utile.
%Faire une fiche histoire des maths (et mettre Bourbaki, Villani, Ghys, etc ...)

%Classification groupes R
%Se rammener a prouver les props sur des trucs engendrant un espace dense ; un pi-syst ; etc …
%Normes equiv en dim finie
%Vrai Ascoli



%Outils généraux : encadrement integral, equivalent log, transfo d’abel, regle de bioche, methode de la phase stationnaire (Laplace) ; Zorn ; Vitali ; Cantor-Bernstein, Césaro, interpolation dans les espaces fonctionnels.

% Analyse réelle, analyse fonctionnelle, analyse hilbertienne


%Rajouter que la mesure image par une matrice inversible de la mesure de Lebesgue revient juste à multiplier par le jacobien. Théorème de Haar à rajouter dans TMI ? Mettre une première partie de définitions et propriétés élémentaires ?


%Topologie : complet ssi cv abs donne cvs ;  Urysohn ; bif continue sur compact = homéo ; Riesz boules compactes ; BAIRE et rmq sur R[X];  Brouwer ; Ascoli ; Dini ; Krein-Milman, projection sur un convexe fermé, représentation de Riesz ; compact ssi B-L ssi B-W ssi precompact complet ;
% Dire que liminf limsup est la plus petite/plus grand valeur d'adhérence.
% Dire que la complétude n'est pas une notion topologique mais une notion métrique. Mettre l'extraction diagonale en méthode avec des exemples.
% Unique valeur d'adhérence dans un compact => cv vers cette v.a. ; L'ensemble des v.a. est un compact.


%Analyse : formules de Taylor, TAF, TVI, théorèmes de limite de prépa (critère séries alternées, Stirling (preuve ?, dualité suite-série), Stone-Weierstrass ; double limite, cv uniforme de la suite des dérivées implique limite dérivable, comparaison série-intégrale, transformée d'Abel, équivalent logarithmique cf exo sinus,  etc ...), Picard, Inversion locale, fonctions implicites, rang constant et la remarque qui va avec, bij continue sur un compact est un homéo (Poincaré), série de Neumann (= ouverture de GL(E)), C^k homéo est un C^k difféo, théorèmes d'analyse fonctionnelle (peut-être dans une autre section).
% Dire que le seul but de l'analyse c'est de faire converger des suites, de trouver des points fixes, et d'utiliser des transformées et avoir des inégalités. En général on doit deviner la limite, si complet alors Cauchy miracle, etc ...
% Analyse (dans les calculs) = cauchy-schwarz ; gronwall ; holder ; Analyse (dans les idées) : point fixe, transformées ; inégalités (convexité ou autre)


% Analyse (Hilbertienne) et EDPs : Cauchy-Lipschitz, Gronwall, Lax-Milgram + Stampaccia, Fourier, Cours d'EDP, Céa, densité C infini à support compact dans Lp. dans autres trucs mettre norme découlant d'un p.s. ssi parallélogramme.


% FOURIER


% Systèmes dynamiques : faire un graphe des implications des propriétés comme en probas : ergodique, mélangeant, minimal, transitif, uniquement ergodique, etc ...


%Algèbre linéaire : Cauchy-Schwarz, inégalité triangulaire, thm spectral, Cayley-Hamilton, décomposition polaire, min-max Courant-Fischer, Gershgorin, rayon spectral, Perron-Frobenius. Mettre les décomposition de Jordan, Dunford, et tout le bazard vu en prépa/L3. Dire que toute forme n-linéaire alternée est proportionnelle au det. Densité de Gl_n, équivalence de matrices, continuité d'une app linéaire ssi norme d'opérateur finie (i.e lipschitz "joli").


%Théorie de l'information : loi uniforme maximise H ; on peut pas désinformer ni détruire de l'information, seulement la spoiler/voler, i.e conditionner réduit forcémment l'entropie, mais peut augmenter ou réduire l'information mutuelle. Exo convexité/concavité pour l'optimisation canal/source.

%Tools/Bestiaire : Formule de Vandermonde avec preuve polynomes.




\begin{document}
\begin{center}
\section*{Théorie de la mesure} 
\end{center}
~\\
\begin{enumerate}
\item Théorème de la classe monotone (Dynkin). \\
\item Théorèmes de convergence monotone, convergence dominée. \\
\item Régularité des mesures. \\
\item Représentation de Riesz-Markov. \\
\item Inégalité de Hölder, inégalité de Young.  \\
\item Complétude des $L^p$. \\
\item Densité dans les $L^p$. \\
\item Dérivée de Radon-Nikodym et décomposition de Lebesgue. \\
\item Fubini, Intégration par parties, Changement de variable. \\
\end{enumerate}


\subsection*{Outils :} ~
\begin{itemize}
\item[-] Lemme de Fatou. \\
\item[-] Régularité des intégrales à paramètre. \\
\item[-] Mesure de Stieltjes. \\
\item[-] Points de Lebesgue. \\
\item[-] Dualité $L^p$-$L^q$. \\
\end{itemize}



\newpage\begin{center}
\section*{Probabilités 1} 
\end{center}
~\\
\begin{enumerate}
\item Inégalité de Markov. \\
\item Injectivité de la transformée de Fourier.\\
\item Lemme des coalitions.\\
\item Loi faible des grands nombres.\\
\item Lemme de Borel-Cantelli. \\
\item Loi du 0-1 de Kolmogorov. \\
\item Loi forte des grands nombres. \\
\item Théorème central limite. \\
\end{enumerate}


\subsection*{Outils :} ~
\begin{itemize}
\item[-] Lemme de Fekete. \\
\item[-] Formule de transfert. \\
\item[-] Inégalité de Jensen. \\
\item[-] Inégalité de Bienaymé-Tchebychef. \\
\item[-] Inégalité de Hoeffding. \\
\item[-] Théorème de Lévy.
\end{itemize}



\newpage
\begin{center}
\section*{Algèbre} 
\end{center}

\begin{enumerate}
\item Théorème de Lagrange
\item Équation aux classes
\item Sylow
\end{enumerate}
\subsection*{Outils :}
- Burnside. \\
- Sous-groupe fini des inversibles d'un corps commutatif est cyclique. \\
- $\mathbb{Z}/p\mathbb{Z}$ espace vectoriel quand $p$ premier et abélien. \\
- Faire agir le centre par multiplication, ou un sous-groupe sympa par conjuguaison.




\newpage
\begin{center}
\section*{Analyse complexe} 
\end{center}

1- Prolongement analytique \\

2- Formules et inégalités de Cauchy \\

3- Résidus \\

4- Représentation conforme de Riemann

~\\

\subsection*{Outils :}
- Cauchy-Riemann \\

- Formule d'homologie \\

- Principe du maximum \\

- Morera \\

- Holo sur une couronne $\Leftrightarrow$ Dev en Fourier sur une bande \\

- Lemme de Schwarz \\


\end{document}
