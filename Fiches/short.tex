\documentclass[12pt,a4paper]{article}

\usepackage{../jedusor}	

\renewcommand{\headrulewidth}{0pt} 
\renewcommand{\footrulewidth}{1pt}
\fancyhead[C]{}
\fancyhead[L]{}
\fancyhead[R]{}
\fancyfoot[C]{\thepage} 
\fancyfoot[L]{Sacha Ben-Arous}
\fancyfoot[R]{E.N.S Paris-Saclay}

\parindent0pt

%Penser à : enveloppe convexe ; dualité ; entropie ; minimisation de l'énergie 
%Théorème fondamental de l'analyse, de l'algèbre ?


%Théorème de Haar à rajouter dans TMI ? Mettre une première partie de définitions et propriétés élémentaires ?


%Faire un chapitre topologie et y mettre complet ssi cv abs donne cvs ;  Urysohn ; Riesz boules compactes ; BAIRE et rmq sur R[X];  Brouwer ; Ascoli ; Dini ; Krein-Milman, projection sur un convexe fermé, représentation de Riesz
% Dire que la complétude n'est pas une notion topologique mais une notion métrique. Mettre l'extraction diagonale en méthode avec des exemples.


%Faire un chapitre analyse et y mettre formules de Taylor, TAF, TVI, théorèmes de limite de prépa (critère séries alternées, Stirling (preuve ?, dualité suite-série), Stone-Weierstrass ; double limite, cv uniforme de la suite des dérivées implique limite dérivable, transformée d'Abel, équivalent logarithmique cf exo sinus,  etc ...), Picard, Inversion locale, fonctions implicites, rang constant et la remarque qui va avec, bij continue sur un compact est un homéo (Poincaré), série de Neumann (= ouverture de GL(E)), C^k homéo est un C^k difféo, théorèmes d'analyse fonctionnelle (peut-être dans une autre section).
% Dire que le seul but de l'analyse c'est de faire converger des suites, de trouver des points fixes, et d'utiliser des transformées et avoir des inégalités. En général on doit deviner la limite, si complet alors Cauchy miracle, etc ...
% Analyse (dans les calculs) = cauchy-schwarz ; gronwall ; holder ; Analyse (dans les idées) : point fixe, transformées ; inégalités (convexité ou autre)


% Analyse (Hilbertienne) et EDPs : Cauchy-Lipschitz, Gronwall, Lax-Milgram, Fourier, Cours d'EDP, Céa, densité C infini à support compact dans Lp.


%Faire un chapitre algèbre linéaire, et y mettre Cauchy-Schwarz, inégalité triangulaire, thm spectral, Cayley-Hamilton, décomposition polaire, min-max Courant-Fischer, Gershgorin, rayon spectral, Perron-Frobenius. Mettre les décomposition de Jordan, Dunford, et tout le bazard vu en prépa/L3. Dire que toute forme n-linéaire alternée est proportionnelle au det. Densité de Gl_n, équivalence de matrices, continuité d'une app linéaire ssi norme d'opérateur finie (i.e lipschitz "joli").


%Mettre des contre-exemples quand c'est intéressant et utile

% VÉRIFIER ORDRE ENTRE SHORT ET LONG



\begin{document}
\begin{center}
\section*{Théorie de la mesure} 
\end{center}
~\\
\begin{enumerate}
\item Théorème de la classe monotone (Dynkin).
\item Théorèmes de convergence monotone, convergence dominée.
\item Régularité des mesures.
\item Représentation de Riesz-Markov.
\item Inégalité de Hölder, inégalité de Young.
\item Complétude des $L^p$.
\item Densité dans les $L^p$.
\item Dérivée de Radon-Nikodym et décomposition de Lebesgue.
\item Fubini, Intégration par parties, Changement de variable.
\end{enumerate}


\subsection*{Outils :}
\begin{itemize}
\item[-] Lemme de Fatou.
\item[-] Régularité des intégrales à paramètre.
\item[-] Mesure de Stieltjes.
\item[-] Points de Lebesgue.
\item[-] Dualité $L^p$-$L^q$.
\end{itemize}



\newpage\begin{center}
\section*{Probabilités} 
\end{center}
~\\
\begin{enumerate}
\item Inégalité de Jensen
\item Inégalité de Markov
\end{enumerate}


\subsection*{Outils :}
\begin{itemize}
\item[•] Lemme de Fekete
\end{itemize}



\newpage
\begin{center}
\section*{Algèbre} 
\end{center}

\begin{enumerate}
\item Théorème de Lagrange
\item Équation aux classes
\item Sylow
\end{enumerate}
\subsection*{Outils :}
- Burnside. \\
- Sous-groupe fini des inversibles d'un corps commutatif est cyclique. \\
- $\mathbb{Z}/p\mathbb{Z}$ espace vectoriel quand $p$ premier et abélien. \\
- Faire agir le centre par multiplication, ou un sous-groupe sympa par conjuguaison.




\newpage
\begin{center}
\section*{Analyse complexe} 
\end{center}

1- Prolongement analytique \\

2- Formules et inégalités de Cauchy \\

3- Résidus \\

4- Représentation conforme de Riemann

~\\

\subsection*{Outils :}
- Cauchy-Riemann \\

- Formule d'homologie \\

- Principe du maximum \\

- Morera \\

- Holo sur une couronne $\Leftrightarrow$ Dev en Fourier sur une bande \\

- Lemme de Schwarz \\


\end{document}
