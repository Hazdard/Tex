\documentclass[12pt,a4paper]{article}

\usepackage{../jedusor}	

\renewcommand{\headrulewidth}{0pt} 
\renewcommand{\footrulewidth}{1pt}
\fancyhead[C]{}
\fancyhead[L]{}
\fancyhead[R]{}
\fancyfoot[C]{\thepage} 
\fancyfoot[L]{Sacha Ben-Arous}
\fancyfoot[R]{E.N.S Paris-Saclay}

\parindent0pt


%Penser à : enveloppe convexe ; dualité ; entropie ; minimisation de l'énergie 
%Se rammener a prouver les props sur des trucs engendrant tout le monde ; un pi-syst ; indicatrices ; base de topo/voisinage ; partie dense ; etc ...
% Adopte la structure minimale. Faire des exemples et les dévisser. Partir d’un problème central et développer des outils pour le résoudre, plutôt que l’inverse : se borner à utiliser un outil que l’on trouve cool pour résoudre des trucs randoms.
%Commencer par la fin. Essayer des cas particuliers, des cas triviaux, des problèmes proches qui ont la même nature. Comprendre à fond ces situations simples. Pour une conjecture, alterner entre les tentatives de preuve et de réfutation.
%Faire une fiche histoire des maths (et mettre Bourbaki, livre Villani, Ghys, etc ...)
%Pour la suite : en fait, l'analyse microlocale (les opérateurs pseudo-diff) et les ondelettes c'est hyper important !


%Mettre des contre-exemples quand c'est intéressant et utile.
%Mettre les résultats de TD dans les fiches.
%RAJOUTER LES BONNES REF AU DEBUT DE CHAQUE FICHES
% Topo ; ana_r ; ana_fonc ; geom ; hilbert ? + Bestiaire


%-------------------------------------------------SOURCES-----------------------------------------------%

%http://www-math.univ-poitiers.fr/~jlehec/
%https://www.math.ucla.edu/~tao/
%https://terrytao.wordpress.com/category/teaching/254a-ergodic-theory/page/2/
%https://terrytao.wordpress.com/category/teaching/255b-incompressible-euler-equations/
%https://terrytao.wordpress.com/tag/finite-time-blowup/
%https://djalil.chafai.net/blog/?s=julia

% Probabilités : LeGall
% Topo + Analyse : L3 --> Ulm --> Début Rudin --> M1 caca --> Villani IAF --> Alazard --> PGerard
% Fourier : utiliser JB et JBB et Alazard et Rudin et PbsD'Evol et Antoine Levitt et Villani

%-------------------------------------------------------------------------------------------------------%






% Analyse (Hilbertienne) et EDOs/EDPs : Lemmes techniques Hilbert sys dyn et EDP et th spectrale ; Résultats prépa sur EDO linéaire avec wronskien, variation de la constante, formule de Duhamel, etc ; Cauchy-Lipschitz (voir Brézis p199 pour version générale), Gronwall, Lax-Milgram + Stampaccia, Fourier, Cours d'EDP, Céa, densité C infini à support compact dans Lp. dans autres trucs mettre norme découlant d'un p.s. ssi parallélogramme.


%Tools/Bestiaire : Formule de Vandermonde avec preuve polynomes. Th de Weierstrass avec probas. Thm de Helly. Encadrement integral, equivalent log, transfo d’abel, regle de bioche, methode de la phase stationnaire (Laplace) ; Zorn ; Vitali ; Cantor-Bernstein, Césaro, interpolation dans les espaces fonctionnels. Kirszbraun theorem. Banach–Mazur theorem. CV ps n'est pas topologisable. Preuve krein-millman villani. Monotone au plus discontinuités dénombrables avec preuve TMI.
% Gronwall avec variantes et preuves.
%Classification groupes R
%Normes equiv en dim finie
%Vrai Ascoli
%Ptite partie théorie des ensembles ?
%Critère pour passer d’une norme à un ps dans un pré hilbert


%Algèbre linéaire : Cauchy-Schwarz, inégalité triangulaire, thm spectral, Cayley-Hamilton, décomposition polaire, min-max Courant-Fischer, Gershgorin, rayon spectral, Perron-Frobenius. Mettre les décomposition de Jordan, Dunford, et tout le bazard vu en prépa/L3. Dire que toute forme n-linéaire alternée est proportionnelle au det. Densité de Gl_n, équivalence de matrices, continuité d'une app linéaire ssi norme d'opérateur finie (i.e lipschitz "joli").






\begin{document}
\begin{center}
\section*{Théorie de la mesure} 
\end{center}
~\\
\begin{enumerate}
\item Théorème de la classe monotone (Dynkin). \\
\item Théorèmes de convergence monotone, convergence dominée. \\
\item Régularité des mesures. \\
\item Représentation de Riesz-Markov. \\
\item Inégalité de Hölder, inégalité de Young.  \\
\item Complétude des $L^p$. \\
\item Densité dans les $L^p$. \\
\item Dérivée de Radon-Nikodym et décomposition de Lebesgue. \\
\item Fubini, Intégration par parties, Changement de variable. \\
\end{enumerate}


\subsection*{Outils :} ~
\begin{itemize}
\item[-] Lemme de Fatou. \\
\item[-] Régularité des intégrales à paramètre. \\
\item[-] Mesure de Stieltjes. \\
\item[-] Points de Lebesgue. \\
\item[-] Dualité $L^p$-$L^q$. \\
\end{itemize}



\newpage\begin{center}
\section*{Probabilités 1} 
\end{center}
~\\
\begin{enumerate}
\item Inégalité de Markov. \\
\item Injectivité de la transformée de Fourier.\\
\item Lemme des coalitions.\\
\item Loi faible des grands nombres.\\
\item Lemme de Borel-Cantelli. \\
\item Loi du 0-1 de Kolmogorov. \\
\item Loi forte des grands nombres. \\
\item Théorème central limite. \\
\end{enumerate}


\subsection*{Outils :} ~
\begin{itemize}
\item[-] Lemme de Fekete. \\
\item[-] Formule de transfert. \\
\item[-] Inégalité de Jensen. \\
\item[-] Inégalité de Bienaymé-Tchebychef. \\
\item[-] Inégalité de Hoeffding. \\
\item[-] Théorème de Lévy.
\end{itemize}



\newpage
\begin{center}
\section*{Algèbre} 
\end{center}

\begin{enumerate}
\item Théorème de Lagrange
\item Équation aux classes
\item Sylow
\end{enumerate}
\subsection*{Outils :}
- Burnside. \\
- Sous-groupe fini des inversibles d'un corps commutatif est cyclique. \\
- $\mathbb{Z}/p\mathbb{Z}$ espace vectoriel quand $p$ premier et abélien. \\
- Faire agir le centre par multiplication, ou un sous-groupe sympa par conjuguaison.




\newpage
\begin{center}
\section*{Analyse complexe} 
\end{center}

1- Prolongement analytique \\

2- Formules et inégalités de Cauchy \\

3- Résidus \\

4- Représentation conforme de Riemann

~\\

\subsection*{Outils :}
- Cauchy-Riemann \\

- Formule d'homologie \\

- Principe du maximum \\

- Morera \\

- Holo sur une couronne $\Leftrightarrow$ Dev en Fourier sur une bande \\

- Lemme de Schwarz \\


\end{document}
