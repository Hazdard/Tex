\documentclass[12pt,a4paper]{article}

\usepackage{../jedusor}	
\renewcommand{\headrulewidth}{0pt} 
\renewcommand{\footrulewidth}{1pt}
\fancyhead[C]{}
\fancyhead[L]{}
\fancyhead[R]{}
\fancyfoot[C]{\thepage} 
\fancyfoot[L]{Sacha Ben-Arous}
\fancyfoot[R]{E.N.S Paris-Saclay}

\parindent0pt


\begin{document}
\begin{center}
\section*{Théorie de la mesure} 
\end{center}
~\\
\begin{enumerate}
\item Théorème de la classe monotone (Dynkin). \\
\item Théorèmes de convergence monotone, convergence dominée. \\
\item Régularité des mesures. \\
\item Représentation de Riesz-Markov. \\
\item Inégalité de Hölder, inégalité de Young.  \\
\item Complétude des $L^p$. \\
\item Densité dans les $L^p$. \\
\item Dérivée de Radon-Nikodym et décomposition de Lebesgue. \\
\item Fubini, Intégration par parties, Changement de variable. \\
\end{enumerate}


\subsection*{Outils :} ~
\begin{itemize}
\item[-] Lemme de Fatou. \\
\item[-] Régularité des intégrales à paramètre. \\
\item[-] Mesure de Stieltjes. \\
\item[-] Points de Lebesgue. \\
\item[-] Dualité $L^p$-$L^q$. \\
\end{itemize}



\newpage\begin{center}
\section*{Probabilités 1} 
\end{center}
~\\
\begin{enumerate}
\item Inégalité de Markov. \\
\item Injectivité de la transformée de Fourier.\\
\item Lemme des coalitions.\\
\item Loi faible des grands nombres.\\
\item Lemme de Borel-Cantelli. \\
\item Loi du 0-1 de Kolmogorov. \\
\item Loi forte des grands nombres. \\
\item Théorème central limite. \\
\end{enumerate}


\subsection*{Outils :} ~
\begin{itemize}
\item[-] Lemme de Fekete. \\
\item[-] Formule de transfert. \\
\item[-] Inégalité de Jensen. \\
\item[-] Inégalité de Bienaymé-Tchebychef. \\
\item[-] Inégalité de Hoeffding. \\
\item[-] Théorème de Lévy.
\end{itemize}



\newpage
\begin{center}
\section*{Algèbre} 
\end{center}

\begin{enumerate}
\item Théorème de Lagrange
\item Équation aux classes
\item Sylow
\end{enumerate}
\subsection*{Outils :}
- Burnside. \\
- Sous-groupe fini des inversibles d'un corps commutatif est cyclique. \\
- $\mathbb{Z}/p\mathbb{Z}$ espace vectoriel quand $p$ premier et abélien. \\
- Faire agir le centre par multiplication, ou un sous-groupe sympa par conjuguaison.




\newpage
\begin{center}
\section*{Analyse complexe} 
\end{center}

1- Prolongement analytique \\

2- Formules et inégalités de Cauchy \\

3- Résidus \\

4- Représentation conforme de Riemann

~\\

\subsection*{Outils :}
- Cauchy-Riemann \\

- Formule d'homologie \\

- Principe du maximum \\

- Morera \\

- Holo sur une couronne $\Leftrightarrow$ Dev en Fourier sur une bande \\

- Lemme de Schwarz \\


\end{document}
