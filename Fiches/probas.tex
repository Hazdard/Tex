\documentclass[11pt,a4paper]{article}

\usepackage{../jedusor}	

\renewcommand{\headrulewidth}{0pt} 
\renewcommand{\footrulewidth}{1pt}
\fancyhead[C]{}
\fancyhead[L]{}
\fancyhead[R]{}
\fancyfoot[C]{\thepage} 
\fancyfoot[L]{Sacha Ben-Arous}
\fancyfoot[R]{E.N.S Paris-Saclay}
	
\begin{document}
\newpage
\begin{center}
\section*{Probabilités} 
\end{center}

%Fekete
%Chap 6 : Jensen ;  Injectivité de Fourier ; (Bonus : les classiques Markov, Bienaymé-Chebyshev )
%
%Chap 7 : Coalitions ; WLLN ; 
%
%Chap 8 : Borel-Cantelli et notion de lim sup/inf; Loi du 0-1 ; SLLN ; TCL (et Levy par la même occasion !)

% Parler des liens entre les différents modes de convergence


\subsection*{Méthode et contexte}
-LES LP SONT EMBOITES 

\subsection*{Résultats principaux}
Dans tout ce qui suit, on travaille dans un espace probabilisé $(\Omega, \mathcal{F}, \P)$, et on pourra utiliser un espace mesurable $(E,\mathcal{E})$.

\begin{definstar} ~
\begin{enumerate}
\item Si $X: \Omega \mapsto E$ est mesurable, alors $X$ est appelée \textit{variable aléatoire} (v.a) à valeurs dans $E$.
\item Si $X$ est une v.a à valeurs dans E, on appelle loi de $X$ la mesure image de $\P$ par $X$, notée $\P_X$ et vérifiant \[\P_X(A) = \P\left(X^{-1}(A)\right)=\P\left(\left\{ \omega\in \Omega \setbar X(\omega)\in A \right\}\right) =\P(X\in A).\]
\end{enumerate}
\end{definstar}


\subsection*{Outils importants}



%\subsection*{Autres théorèmes}

\end{document}
