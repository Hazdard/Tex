\documentclass[11pt,a4paper]{article}

\usepackage{../jedusor}	

\renewcommand{\headrulewidth}{0pt} 
\renewcommand{\footrulewidth}{1pt}
\fancyhead[C]{}
\fancyhead[L]{}
\fancyhead[R]{}
\fancyfoot[C]{\thepage} 
\fancyfoot[L]{Sacha Ben-Arous}
\fancyfoot[R]{E.N.S Paris-Saclay}
	
\begin{document}
\newpage
\begin{center}
\section*{Probabilités} 
\end{center}

%Fekete
%Chap 6 : Jensen ;  Injectivité de Fourier ; (Bonus : les classiques Markov, Bienaymé-Tchebychef )
%
%Chap 7 : Coalitions ; WLLN ; 
%
%Chap 8 : Borel-Cantelli et notion de lim sup/inf; Loi du 0-1 ; SLLN ; TCL (et Levy par la même occasion !)

% Parler des liens entre les différents modes de convergence


\subsection*{Méthode et contexte}
\begin{itemize}
\item[-] Une mesure de probabilité étant en particulier finie, on a dans ce cadre que les espaces $L^p$ sont emboités, i.e : $L^\infty \subseteq \dots \subseteq L^1$. Cela se traduit par le fait que si une variable aléatoire possède un moment d'ordre $k$, tous ses moments d'ordre inférieur sont également finis.
\end{itemize}

\subsection*{Définitions et propriétés élémentaires}
Dans tout ce qui suit, on travaille dans un espace probabilisé $(\Omega, \mathcal{F}, P)$, et on pourra utiliser un espace mesurable $(E,\mathcal{E})$.

\begin{definstar} ~
\begin{enumerate}
\item Si $X: \Omega \mapsto E$ est mesurable, alors $X$ est appelée \textit{variable aléatoire} (v.a.) à valeurs dans $E$.
\item Si $X$ est une v.a. à valeurs dans E, on appelle loi de $X$ la mesure image de $P$ par $X$, notée $P_X$ et vérifiant \[P_X(A) = P\left(X^{-1}(A)\right)=P\left(\left\{ \omega\in \Omega \setbar X(\omega)\in A \right\}\right) =P(X\in A).\]
\end{enumerate}
\end{definstar}

\begin{definstar}
Pour toute v.a.r $X$, on appelle \textit{fonction de répartition} de $X$ la donnée de $F_X :\R \mapsto [0,1]$ définie par $F_X(t) = P(X\leq t) = P_X(]-\infty,t])$.
\end{definstar}

\begin{rmq}
$F_X$ est continue à droite, limitée à gauche (càdlàg), croissante, tend vers $0$ en $-\infty$, $1$ en $+\infty$, et caractérise $P_X$.
\end{rmq}

\begin{definstar}
Soit $X$ une v.a. à valeurs dans $\R^d$. On appelle fonction caractéristique de $X$, notée $\Phi_X$, la fonction de $\R^d$ dans $\C$ définie par \[\Phi_X(\xi) := \int_{\R^d} e^{i\left<x,\xi\right>}\mathrm{d}P_X(x) = E\left(e^{i\left<X,\xi\right>}\right).\]
\end{definstar}

\begin{rmq}
$\Phi_X$ est en fait la transformée de Fourier de la loi $P_X$. C'est une fonction uniformément continue, dont le module est borné par $1$. $\Phi_X$ a autant de dérivées que $X$ a de moments finis.
\end{rmq}

\begin{rmq}
Si $X\sim \mathcal{N}(\mu,\sigma^2)$, alors $\Phi_X(\xi)=exp(i\xi\mu - \frac{\xi^2\sigma^2}{2})$.
\end{rmq}

\begin{definstar}
Si $X$ est une v.a. à valeurs dans $\N$, on appelle fonction génératrice de $X$, la fonction $G_X : [0,1] \mapsto \R^+$ définie par : \[G_X(t) := E(t^X) = \sum_{n=0}^{+\infty} t^nP(X=n).\]
\end{definstar}

\begin{rmq}
$G_X$ caractérise la loi de $X$, et détermine tous les moments de $X$ comme l'explique la proposition suivante.
\end{rmq}

\begin{propstar}
Soit $X$ une v.a. à valeurs dans $\N$, alors pour tout $k\geq 1$ :\[E\left(\prod_{i=0}^{k-1}(X-i)\right) = \lim_{t\to 1^-}G_X^{(k)}(t).\]
\end{propstar}

\newpage
\subsection*{Résultats principaux}

\begin{thmstar}
[Inégalité de Markov] Soit $X$ une v.a.r presque surement positive, alors pour $\alpha>0$ : \[P(X \geq \alpha) \leq \frac{E(X)}{\alpha}\]
\end{thmstar}

\begin{thmstar}
[Inégalité de Jensen] Soient $X\in L^1$, et $\Phi$ une fonction convexe sur un intervalle $I$ tel que $P(X\in I)=1$ et $E(|\Phi(X)|)<\infty$. Alors $\Phi(E(X)) \leq E(\Phi(X))$. Si $\Phi$ est de plus strictement convexe, alors il y a égalité si et seulement si $X$ est p.s constante.
\end{thmstar}

\begin{thmstar}[Injectivité de la transformée de Fourier]
Soient $X_1$ et $X_2$ des v.a. à valeurs dans $\R^d$. Si $\Phi_{X_1}=\Phi_{X_2}$, alors $P_{X_1}=P_{X_2}$.
\end{thmstar}

\subsection*{Outils importants}

\begin{propstar}
[Changement de variable] Soit $X$ une v.a. à valeurs dans $(E,\mathcal{E})$, et $f :E\mapsto \overline{\R}$ une fonction mesurable telle que $f\geq0$ p.p. ou $E\big( \left|f(X)\right| \big) <\infty$, alors :
\[E(f(X))=\int_E f(x)\mathrm{d}P_X(x).\]
\end{propstar}

\begin{corstar}
[Inégalité de Bienaymé-Tchebychef] Si $X\in L^2$ est une v.a.r, alors pour tout $\varepsilon >0$ :
\[P(|X-E(X)| \geq \varepsilon) \leq \frac{V(X)}{\varepsilon^2}\]
\end{corstar}

\subsection*{Autres résultats}

\begin{lemmastar}
Soit $I$ un intervalle de $\R$. Si $ \Phi :I \mapsto \R$ est une fonction convexe, alors pour tout $x\in \mathring{I}$ :
\[\Phi(x) = \sup_{a,b \ | \ l_{a,b} \leq \Phi } l_{a,b}(x)\]
\end{lemmastar}

%\subsection*{Autres théorèmes}

\end{document}
