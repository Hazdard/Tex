\documentclass[11pt,a4paper]{article}

\usepackage{../jedusor}	

\renewcommand{\headrulewidth}{0pt} 
\renewcommand{\footrulewidth}{1pt}
\fancyhead[C]{}
\fancyhead[L]{}
\fancyhead[R]{}
\fancyfoot[C]{\thepage} 
\fancyfoot[L]{Sacha Ben-Arous}
\fancyfoot[R]{E.N.S Paris-Saclay}
	
\begin{document}
\newpage
\begin{center}
\section*{Probabilités} 
\end{center}

%Chap 8 : Borel-Cantelli et notion de lim sup/inf; Loi du 0-1 ; SLLN ; TCL (et Levy par la même occasion !)

% Parler des liens entre les différents modes de convergence

%Rajouter TDs de 1A et 2A why not

% Cv dominée marche aussi en remplcant l'hyp de cvs par une cv en proba (cf début chap 10) (pas dans le cours mais utiliser la distance).

% Schéma des implications de cv (version Trouvé et LeGall) + commentaire que cv L1 et p.s sont de natures différentes, et p.s est plus naturelle pour les probas.

% Il y a aussi cv dans L^1 dans la loi forte des grands nombres, avec 12.5 du poly.

\subsection*{Contexte}
\begin{itemize}
\item[-] Une mesure de probabilité étant en particulier finie, on a dans ce cadre que les espaces $L^p$ sont emboités, i.e : $L^\infty \subseteq \dots \subseteq L^1$. Cela se traduit par le fait que si une variable aléatoire possède un moment d'ordre $k$, tous ses moments d'ordre inférieur sont également finis.
\item[-] Des variables indépendantes sont de covariance nulle, mais la réciproque est très fausse ! Par exemple en considérant une loi gaussienne et son produit par une v.a. de Rademacher, leur covariance est nulle mais elles ne sont pas indépendantes, sinon leurs valeurs absolues le seraient, et donc la gaussienne serait indépendante d'elle même, i.e. constante.
\item[-] À l'inverse, des variables a priori corrélées peuvent êtres indépendantes : si $U$ est une loi exponentielle et $V$ une loi uniforme sur $[0,1]$, alors $\sqrt{U}\cos(2\pi V)$ et $\sqrt{U}\sin(2\pi V)$ sont indépendantes et suivent chacune la loi $\mathcal{N}(0,1/2)$.
\end{itemize}

\subsection*{Méthode}
\begin{itemize}
\item[-] Utiliser les outils adaptés : pour étudier une somme de v.a. indépendantes on utilise la transformée de Fourier, pour étudier leur $\min$ on utilise la fonction de répartition, etc $\dots$
\item[-] Pour calculer la loi d'un couple $(X,Y)$ de v.a., on prend $f$ mesurable positive et on essaye d'écrire $\displaystyle E\left(f(X,Y)\right) = \int f(x,y) \mathrm{d}\mu(x,y)$, et alors le couple est de loi $\mu$.
\end{itemize}

\subsection*{Définitions et propriétés élémentaires}

\begin{definstar} Soit  $(\Omega, \mathcal{F}, P)$ un espace probabilisé, et $(E,\mathcal{E})$ un espace mesurable.
\begin{enumerate}
\item Si $X: \Omega \to E$ est mesurable, alors $X$ est appelée \textit{variable aléatoire} (v.a.) à valeurs dans $E$.
\item Si $X$ est une v.a. à valeurs dans E, on appelle loi de $X$ la mesure image de $P$ par $X$, notée $P_X$ et vérifiant : 
\[P_X(A) = P\left(X^{-1}(A)\right)=P\left(\left\{ \omega\in \Omega \setbar X(\omega)\in A \right\}\right) =P(X\in A).\]
\end{enumerate}
\end{definstar}

\begin{definstar}
Pour toute v.a.r $X$, on appelle \textit{fonction de répartition} de $X$ la donnée de $F_X :\R \to [0,1]$ définie par $F_X(t) = P(X\leq t) = P_X(]-\infty,t])$.
\end{definstar}

\begin{rmq}
$F_X$ est continue à droite, limitée à gauche (càdlàg), croissante, tend vers $0$ en $-\infty$, $1$ en $+\infty$, et caractérise $P_X$.
\end{rmq}

\begin{definstar}
Soit $X$ une v.a. à valeurs dans $\R^d$. On appelle \textit{fonction caractéristique} de $X$, notée $\Phi_X$, la fonction de $\R^d$ dans $\C$ définie par \[\Phi_X(\xi) := \int_{\R^d} e^{i\left<x,\xi\right>}\mathrm{d}P_X(x) = E\left(e^{i\left<X,\xi\right>}\right).\]
\end{definstar}

\begin{rmq}
$\Phi_X$ est en fait la transformée de Fourier de la loi $P_X$. C'est une fonction uniformément continue, dont le module est borné par $1$. $\Phi_X$ a autant de dérivées que $X$ a de moments finis.
\end{rmq}

\begin{rmq}
Si $X\sim \mathcal{N}(\mu,\sigma^2)$, alors $\Phi_X(\xi)=exp(i\xi\mu - \frac{\xi^2\sigma^2}{2})$.
\end{rmq}

\begin{definstar}
Si $X$ est une v.a. à valeurs dans $\N$, on appelle \textit{fonction génératrice} de $X$, la fonction $G_X : [0,1] \to \R^+$ définie par : \[G_X(t) := E(t^X) = \sum_{n=0}^{+\infty} t^nP(X=n).\]
\end{definstar}

\begin{rmq}
$G_X$ caractérise la loi de $X$, et détermine tous les moments de $X$ comme l'explique la proposition suivante.
\end{rmq}

\begin{propstar}
Soit $X$ une v.a. à valeurs dans $\N$, alors pour tout $k\geq 1$ :\[E\left(\prod_{i=0}^{k-1}(X-i)\right) = \lim_{t\to 1^-}G_X^{(k)}(t).\]
\end{propstar}

\begin{definstar}[Indépendance]~
\begin{itemize}
\item[-] Des événements $(A_i)_{i\in I}$  sont dits indépendants si pour toute partie finie $J$ de $I$, on a : \[P\Big(\bigcap_{j\in J}A_j\Big)=\prod_{j\in J}A_j\]
\item[-] Des tribus  $(\mathcal{A}_i)_{i\in I}$  sont dites indépendantes si pour toute famille $(A_i)_{i\in I}$ telle que $A_i\in \mathcal{A}_i$, les événements sont indépendants.
\item[-] Des variables aléatoires $(X_i)_{i\in I}$  à valeurs dans des espaces mesurables $(E_i,\mathcal{E}_i)$ sont dites indépendantes si la famille de tribus $(\sigma(X_i))_{i\in I}$ l'est.
\end{itemize}
\end{definstar}

\begin{rmq}
L'indépendance des $(X_i)_{i\in I}$ porte sur les tribus engendrées (sur $\Omega$) et non sur les
valeurs proprement dites de ces variables aléatoires. Par suite, si des $\Phi_i : (E_i,\mathcal{E}_i) \to (E'_i,\mathcal{E}'_i)$ sont mesurables, l'indépendance des $(X_i)_{i\in I}$ entraine celle des $(\Phi_i(X_i))_{i\in I}$.
\end{rmq}

\begin{rmq}
La vérification de l'indépendance des $(X_i)_{i\in I}$ se rammène à montrer que pour tout $J\subset I$ fini, pour toute famille $(A_j)_{j\in J}$ telle que $A_j\in \mathcal{E}_j$, on a $\displaystyle P\Big(\bigcap_{j\in J}(X_j\in A_j )\Big)=\prod_{j\in J}(X_j \in A_j)$.
\end{rmq}

\begin{propstar}
[Caractérisations de l'indépendance] Soit $(X_i)_{ 1 \leq i \leq n}$ une famille de variables aléatoires réelles, et $X:=(X_1,\dots,X_n)$. 
\begin{enumerate}
\item Si les $(X_i)_{ 1 \leq i \leq n}$ sont indépendants, et ont $(f_{X_i})_{1\leq i \leq n}$ comme densités respectives par rapport à la mesure de Lebesgue, alors $P_X \ll \lambda_n$ et a pour densité $f_X(x_1,\dots,x_n)=f_{X_1}(x_1)\dots f_{X_n}(x_n)$.
\item Réciproquement, si $P_X \ll \lambda_n$, de densité s'écrivant $f_X(x_1,\dots,x_n)=f_{X_1}(x_1)\dots f_{X_n}(x_n)$ où les $(f_i)_{1\leq i \leq n}$ sont des densités de probabilité (i.e. positives, d'intégrale valant $1$), alors les $(X_i)_{ 1 \leq i \leq n}$ sont indépendants, de densités respectives $(f_{X_i})_{1\leq i \leq n}$.
\end{enumerate}
\end{propstar}

\begin{corstar} Soit $(X_i)_{ 1 \leq i \leq n}$ des variables aléatoires réelles, les propriétés suivantes sont équivalentes :
\begin{enumerate}
\item  $(X_i)_{ 1 \leq i \leq n}$ est une famille de variables aléatoires indépendantes ; 
\item $P_X=\otimes_{i=1}^n P_{X_i}$
\item Pour toute famille $(f_i)_{1\leq i \leq n}$ de fonctions boréliennes positives, $\displaystyle E\left(\prod_{i=1}^n f_i(X_i)\right) = \prod_{i=1}^n E\left(f_i(X_i)\right)$
\item $\Phi_X=\otimes_{i=1}^n \Phi_{X_i}$
\end{enumerate}
\end{corstar}


\subsection*{Résultats principaux}

\begin{thmstar}
[Inégalité de Markov] Soit $X$ une variable aléatoire réelle presque surement positive, alors pour $\alpha>0$ : \[P(X \geq \alpha) \leq \frac{E(X)}{\alpha}\]
\end{thmstar}

\begin{thmstar}
[Inégalité de Jensen] Soient $X\in L^1$, et $\Phi$ une fonction convexe sur un intervalle $I$ tel que $P(X\in I)=1$ et $E(|\Phi(X)|)<\infty$. Alors $\Phi(E(X)) \leq E(\Phi(X))$. Si $\Phi$ est de plus strictement convexe, alors il y a égalité si et seulement si $X$ est p.s constante.
\end{thmstar}

\begin{thmstar}[Injectivité de la transformée de Fourier]
Soient $X_1$ et $X_2$ des variables aléatoires à valeurs dans $\R^d$. Si $\Phi_{X_1}=\Phi_{X_2}$, alors $P_{X_1}=P_{X_2}$.
\end{thmstar}

\begin{thmstar}
[Coalitions] Soit $(\mathcal{A}_i)_{i\in I}$ une famille de tribus engendrées par les $\pi$-systèmes $(C_i)_{i\in I}$. Alors ces tribus sont indépendantes si et seulement si les $\pi$-systèmes générateurs le sont. En particulier, si $(X_i)_{i\in I}$ est une famille de variables aléatoires indépendantes, et $(I_k)_{k\in K}$ une partition de $I$, alors les tribus $\big(\sigma\left(X_i,\ i\in I_k\right)\big)_{k\in K}$ sont indépendantes.
\end{thmstar}

\begin{thmstar}
[Loi faible des grands nombres] Soit $(X_i)_{i\in \N^*}$ une famille de variables aléatoires indépendantes de $L^2$, telles que $\lim\limits_{n\to +\infty} \frac{1}{n}\sum_{i=1}^n E(X_i) = \mu$ et $\sup_i V(X_i) = \sigma^2$. Alors : 
\begin{enumerate}
\item La moyenne empirique $\displaystyle \overline{X}_n := \frac{1}{n}\sum_{i=1}^n X_i$ converge dans $L^2$ vers la moyenne théorique $\mu$.
\item Pour $\varepsilon >0$, on a l'estimation suivante : \[P(|\overline{X}_n - E(\overline{X}_n)| \geq \varepsilon) \leq \frac{\sigma^2}{n\varepsilon^2}.\]
\end{enumerate}
\end{thmstar}

\begin{lemmastar}
[Borel-Cantelli] Soit $(A_n)_{n\geq 0}$ une suite d'évènements. On note $\limsup A_n := \displaystyle \bigcap_{n\geq 0} \bigcup_{k\geq n} A_k$, qui correspond à l'évènement ``être dans une infinité de $A_n$''. On a alors la dichotomie suivante :
\begin{enumerate}
\item Si $\displaystyle \sum_n P(A_n) < \infty$, alors $P(\limsup A_n)=0$, i.e. $\sum_n \mathds{1}_{A_n} < +\infty$ presque sûrement.
\item Si $\displaystyle \sum_n P(A_n) = \infty$ et que les $(A_n)_{n\geq 0}$ sont indépendants, alors $P(\limsup A_n)=1$.
\end{enumerate}
\end{lemmastar}

\subsection*{Outils importants}

\begin{lemmastar}
[Fekete] Si $(u_n)_{n\in\N}$ est une suite sous additive, i.e. $\forall n,m\in \N, \ u_{n+m} \leq u_n + u_m$, alors $\displaystyle (\frac{u_n}{n})_{n\in\N}$ converge, et on a l'égalité:  $\displaystyle \lim\limits_{n\to \infty} \frac{u_n}{n}=\inf_{n\geq 1} \frac{u_n}{n} \in \R \cup \{ -\infty\}$.
\end{lemmastar}

\begin{propstar}
[Changement de variable] Soit $X$ une v.a. à valeurs dans $(E,\mathcal{E})$, et $f :E\to \overline{\R}$ une fonction mesurable telle que $f\geq0$ p.p. ou $E\big( \left|f(X)\right| \big) <\infty$, alors :
\[E(f(X))=\int_E f(x)\mathrm{d}P_X(x).\]
\end{propstar}

\begin{corstar}
[Inégalité de Bienaymé-Tchebychef] Si $X\in L^2$ est une v.a.r., alors pour tout $\varepsilon >0$ :
\[P(|X-E(X)| \geq \varepsilon) \leq \frac{V(X)}{\varepsilon^2}\]
\end{corstar}

\begin{propstar}
[Inégalité de Hoeffding]
Soient $(X_i)_{i\in \N^*}$ une famille de v.a. indépendantes à valeurs dans $[a,b]$. Alors pour tout $\epsilon > 0$ :
\[P( |\overline{X}_n - E(\overline{X}_n) | \geq \epsilon) \leq 2 \exp(-2\frac{n\varepsilon^2}{(b-a)^2})\]
\end{propstar}

\subsection*{Autres résultats}

\begin{lemmastar}
Soit $I$ un intervalle de $\R$. Si $ \Phi :I \to \R$ est une fonction convexe, alors pour tout $x\in \mathring{I}$ :
\[\Phi(x) = \sup_{a,b \ | \ l_{a,b} \leq \Phi } l_{a,b}(x)\]
\end{lemmastar}

\begin{lemmastar}
Pour une v.a. à valeurs dans $\R^+$, $\displaystyle E(X)=\int_0^\infty P(X \geq x) \mathrm{d}x$, car $\displaystyle X=\int_0^\infty 1_{x\leq X} \mathrm{d}x$
\end{lemmastar} 
\end{document}
