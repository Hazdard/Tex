\documentclass[11pt,a4paper]{article}

\usepackage{../jedusor}
	
\renewcommand{\headrulewidth}{1pt} 
\renewcommand{\footrulewidth}{1pt}	
\fancyhead[C]{}
\fancyhead[L]{}
\fancyhead[R]{}
\fancyfoot[C]{\thepage} 
\fancyfoot[L]{Sacha Ben-Arous}
\fancyfoot[R]{E.N.S Paris-Saclay}

\begin{document}
ayaya

\begin{center}  
\section*{Analyse harmonique} 
\end{center}
% Motivations : diagonaliser les op differentiels. Lien avec la physique au vois d'un equilibre (i.e les cas faciles de la vie de tous les jours) le gradient est nul, il reste le laplacien. On s'interesse en particulier à cet op, est-il sym ? auto adj ? diag ?

% Wikipedia : Bochner's theorem ; Positive harmonic function
% théorème de représentation spectrale d’Herglotz (poly salez séries temporelles chap4)
% En plus de Bochner, regarder et essayer de parler du thm de Schoenberg (cf papier dont Gérard m'a parlé).

% Sources : utiliser JB et JBB et Alazard et Rudin et PbsD'Evol et Antoine Levitt et Villani ; et surtout Trigonometric series de Zygmund

% Regarder la théorie des fonction harmoniques du coup ...

\subsection*{Contexte}

\subsection*{Méthode}

\subsection*{Définitions et propriétés élémentaires}

\subsection*{Résultats principaux}

\subsection*{Outils importants}


\subsection*{Autres résultats}

\end{document}
