\documentclass[11pt,a4paper]{article}

\usepackage{../jedusor}	

\renewcommand{\headrulewidth}{0pt} 
\renewcommand{\footrulewidth}{1pt}
\fancyhead[C]{}
\fancyhead[L]{}
\fancyhead[R]{}
\fancyfoot[C]{\thepage} 
\fancyfoot[L]{Sacha Ben-Arous}
\fancyfoot[R]{E.N.S Paris-Saclay}
	
\begin{document}

\begin{center}  
\section*{Analyse réelle} 
\end{center}

%Analyse : Théorème fondamental de l'analyse, formules de Taylor, TAF, TVI, Heine, Dini, bij continue sur R homéo, théorèmes de limite de prépa (série de fonctions normalement cv on peut mettre la limite de continuité dans la somme)(et autres résultats sur les séries et les modes de convergence)(critère séries alternées, Stirling (preuve ?, dualité suite-série), Stone-Weierstrass ; double limite, cv uniforme de la suite des dérivées implique limite dérivable, comparaison série-intégrale, CESARO, transformée d'Abel et critère de CV qui va avec, équivalent logarithmique cf exo sinus,  etc ...), Picard, Inversion locale, fonctions implicites, rang constant et la remarque qui va avec, bij continue sur un compact est un homéo (Poincaré), série de Neumann (= ouverture de GL(E) et dire que |T-Id|<1 donne T inversible et son inverse en série), C^k homéo est un C^k difféo, théorèmes d'analyse fonctionnelle (peut-être dans une autre section) ; Les théorèmes de prolongement (dense + complet ; prépa prolongement de la dérivée) ; Principe du maximum (laplacien). Double limite (vérifier pour Banach) + version suite.
% Dire que le seul but de l'analyse c'est de faire converger des suites, de trouver des points fixes, et d'utiliser des transformées et avoir des inégalités. En général on doit deviner la limite, si complet alors Cauchy miracle, etc ...
% Analyse (dans les calculs) = cauchy-schwarz ; gronwall ; holder ; Analyse (dans les idées) : point fixe, transformées ; inégalités (convexité ou autre) ;
%Dire les 3 grands usages de la complétude (cf Villani), rajouter sommes de Neumann
%Moralement l'analyse c'est toujours pareil : on fais les cas cools régulier, puis on tronque et on régularise pour s'y rammener par densité.
%Regle d’hadamard, critere d’abel pour le rayon des series ent, regle de d’alembert


\subsection*{Contexte}

\subsection*{Méthode}

\subsection*{Définitions et propriétés élémentaires}

\subsection*{Résultats principaux}

\subsection*{Outils importants}

\subsection*{Autres résultats}


\newpage
\begin{center}  
\section*{Analyse de Banach et Hilbert} 
\end{center}

% Topologie des espaces de Banach ; puis Géométrie des espaces de Hilbert

%Formule de polarisation (R et C) et suffisant d’avoir le parallèlo pour découler d’un ps (peut etre bestiaire) ?
%Semi C0
%Lemmes techniques Hilbert sys dyn et EDP et th spectrale
%Interpolation d’espaces 
%Analyse fonc exo compact approché par un ev de dim finie
%Astuce d’écrire u comme l’intégrale de sa dérivée 
% inégalité de sobolev

\subsection*{Contexte}

\subsection*{Méthode}

\subsection*{Définitions et propriétés élémentaires}

\subsection*{Résultats principaux}

\subsection*{Outils importants}


\subsection*{Autres résultats}


\newpage
\begin{center}  
\section*{Théorie spectrale} 
\end{center}


\subsection*{Contexte}

\subsection*{Méthode}

\subsection*{Définitions et propriétés élémentaires}

%def opérateur, domaine tjrs dense, prop pour etre un domaine, def spectre, def etre fermé.
%contre exemple : le décalage à droite sur l^2 est injectif mais pas surjectif, 0 est une valeur spectrale mais pas valeur propre.
%Le spectre est fermé.
%si pas fermé alors spectre vaut C
%Impulsion sur L^2(R) et fonctions tests est pas fermé.
%fermeture
%Le spectre n'est jamais vide !
%f(1/2) non fermable
%dans preuve th 2.16, dévisser l'argument de c'est égal au sens des distrib, (donc on problonge car D est dense dans L2, puis riesz donc unique représentation qui est dans L2, donc ok).
%def adjoint avec besoin de fermable (et bien sur domaine dense). Lien EDP ?
% Def symétrique plus inclusion dans l'adjoint.
%Remarque que les extensions symétriques (fermées) sont entre A (\bar{A}) et A^*.
%Lemme : tout opérateur symétrique est fermable, et sa fermeture est symétrique.
% <Au|u> est toujours réel quand A est symétrique car égal à son conjugué.
%Lemme 2.22
%def auto-adjoint, et si symétrique alors condition "faible" pour être auto adjoint.
% Un op auto adjoint n'a aucune extension ou restriction stricte qui est encore auto adjointe.
%PERRON-FROBENIUS.

\subsection*{Résultats principaux}

%Spectre des opérateurs symétriques, et bien dire en remarque que A-z est tjrs injectif quand im z non nul, i.e pas de valeur propre complexe. Au pire, ce n'est pas surjectif, mais si ça l'est, alors l'inverse exist et est borné.

\subsection*{Outils importants}


\subsection*{Autres résultats}

\newpage
\begin{center}  
\section*{Équations différentielles} 
\end{center}

%EDP :
%Preuve Feller-Miyadera-Phillips
%Comprendre exemples du cours (voir poly pour peut-être plus de détails).
% Mettre qqpart avec preuve le lemme sur la dérivée d'une distrib nulle => distrib cste

% Analyse (Hilbertienne) et EDOs/EDPs :  Résultats prépa sur EDO linéaire avec wronskien, variation de la constante, formule de Duhamel, etc ; Cauchy-Lipschitz (voir Brézis p199 pour version générale), Gronwall, Lax-Milgram + Stampaccia, Fourier, Cours d'EDP, Céa, densité C infini à support compact dans Lp et Sobolev.  
% Lemme d'explosion (cf Majda Bertozzi p103 Theoreme 3.3 avec ref Hartman).
%Variation de la constante en EDO
%Formule de Duhamel
%Les distributions en analyse/EDP jouent le même rôle que les indicatrices en théorie de la mesure.
%Le Laplacien apparait souvent dans les EDP car on est au voisinage d'un point d'équilibre, donc la dérivée première s'annule et en première approximation on est en équilibre dans un potentiel harmonique.
% Lire le bouquin de Tao "Dispersive equations" en particulier la partie 1, Cauchy-Kowalevski, Boostrap arguments, Noether invariance, Duhamel formula, monotonocity formula, etc ... 

\subsection*{Contexte}

\subsection*{Méthode}

\subsection*{Définitions et propriétés élémentaires}

\subsection*{Résultats principaux}

\subsection*{Outils importants}


\subsection*{Autres résultats}

\end{document}
