\documentclass[11pt,a4paper]{article}

\usepackage{../jedusor}
	
\renewcommand{\headrulewidth}{1pt} 
\renewcommand{\footrulewidth}{1pt}	
\fancyhead[C]{}
\fancyhead[L]{}
\fancyhead[R]{}
\fancyfoot[C]{\thepage} 
\fancyfoot[L]{Sacha Ben-Arous}
\fancyfoot[R]{E.N.S Paris-Saclay}

\begin{document}
ayaya

\begin{center}  
\section*{Analyse de Banach et Hilbert} 
\end{center}

% Topologie des espaces de Banach ; puis Géométrie des espaces de Hilbert

%Formule de polarisation (R et C) et suffisant d’avoir le parallèlo pour découler d’un ps (peut etre bestiaire) ?
%Semi C0
%Lemmes techniques Hilbert sys dyn et EDP et th spectrale
%Interpolation d’espaces 
%Analyse fonc exo compact approché par un ev de dim finie
%Astuce d’écrire u comme l’intégrale de sa dérivée 
% inégalité de sobolev
% Hilbert Schmidt , test de Schur
% L’inégalité de Poincaré c’est relié à la première v.p; du Laplacien sur le domaine.
% Fredholm par Tao : https://terrytao.wordpress.com/2011/04/10/a-proof-of-the-fredholm-alternative/

\subsection*{Contexte}

\subsection*{Méthode}

\subsection*{Définitions et propriétés élémentaires}

\subsection*{Résultats principaux}

\subsection*{Outils importants}


\subsection*{Autres résultats}

\end{document}
