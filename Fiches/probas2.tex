\documentclass[11pt,a4paper]{article}

\usepackage{../jedusor}	

\renewcommand{\headrulewidth}{0pt} 
\renewcommand{\footrulewidth}{1pt}
\fancyhead[C]{}
\fancyhead[L]{}
\fancyhead[R]{}
\fancyfoot[C]{\thepage} 
\fancyfoot[L]{Sacha Ben-Arous}
\fancyfoot[R]{E.N.S Paris-Saclay}
	
\begin{document}
\newpage
\begin{center}  
\section*{Martingales} 
\end{center}


% Interprétation loi espérance conditionnelle : valeur moyenne de X quand B est réalisé

% Réfléchir aux interprétations pour avoir de manière intuitive que E( E(X | B) ) = E(X)

% Prop début cours 8.1.3

% 10.1.3 (voir lien avec 12.5 ?)

% 12.5

% Scheffé

% Espace L^0 est complet

% Portmanteau ? (Chap 8 prop 26 Trouvé)


\subsection*{Contexte}
\begin{itemize}
\item[-] L'espérance conditonnelle est une v.a. dont la valeur sur un élément de la sous-tribu s'interprète comme la moyenne de la v.a. initiale sur cet élément.
\end{itemize}

\subsection*{Méthode}

\subsection*{Définitions et propriétés élémentaires}

\subsection*{Résultats principaux}

\begin{itemize}
\item calculs effectifs d'espérance et de loi conditionnelle 
\item Props sur l'uniforme intégrabilité
\item Super cv dominée
\item Lemme de Scheffé
\end{itemize}

\subsection*{Outils importants}


\subsection*{Autres résultats}

\end{document}