\documentclass[11pt,a4paper]{article}

\usepackage{../jedusor}	

\renewcommand{\headrulewidth}{0pt} 
\renewcommand{\footrulewidth}{1pt}
\fancyhead[C]{}
\fancyhead[L]{}
\fancyhead[R]{}
\fancyfoot[C]{\thepage} 
\fancyfoot[L]{Sacha Ben-Arous}
\fancyfoot[R]{E.N.S Paris-Saclay}
	
\begin{document}
\newpage
\begin{center}  
\section*{Bestiaire} 
\end{center}


%Tools/Bestiaire : Formule de Vandermonde avec preuve polynomes. Th de Weierstrass avec probas. Thm de Helly. Encadrement integral, equivalent log, transfo d’abel, regle de bioche, methode de la phase stationnaire (Laplace) ; Zorn ; Vitali ; Cantor-Bernstein, Césaro, interpolation dans les espaces fonctionnels. Kirszbraun theorem. Banach–Mazur theorem. CV ps n'est pas topologisable. Preuve krein-millman villani. Monotone au plus discontinuités dénombrables avec preuve TMI.
% Gronwall avec variantes et preuves.
%Classification groupes R
%Normes equiv en dim finie
%Vrai Ascoli
%Ptite partie théorie des ensembles ?
%Critère pour passer d’une norme à un ps dans un pré hilbert
% Fonctions maximales de Hardy-Littlewood et applications (points de lebesgue, systèmes dynamiques).
%Rédiger une preuve de Brouwer avec lemme non rétraction depuis le degré (TD 2 Exo 1)
% Preuve Stirling sup
% Thm spectral cas antisymétrique
% Mettre des preuves astucieuses, genre le calcul de zeta(2) via Fourier.
% Schur test

% Camus, "Mal nommer les choses, c'est ajouter au malheur du monde"

%Partie histoire et fun fact :
% Regarder qui est Olga Ladyzhenskaya, et approfondir pour Noether
% Leray le goat
% Contre toute attente, René Thom était proche de JL Godart et de Dali. Le premier a fait un documentaire sur lui : https://www.youtube.com/watch?v=B1t_o_CMA_E ; et le second a fait son dernier tableau en référence à Thom, et ses derniers croquis auraient été faits sur un livre de Thom. D'ailleurs, similarités entre la Nouvelle vague et Bourbaki ?


%--- Ressources online ---
%
%Interview de Laure Saint-Raymond : https://www.youtube.com/watch?v=biL25sElGII et https://www.youtube.com/watch?v=Q-WY_eIi1WM (Épisodes 1 et 2)
%Alain Connes sur France Inter avec Étienne Klein : https://www.radiofrance.fr/franceculture/podcasts/science-en-questions/comment-et-a-quoi-les-mathematiciens-revent-ils-2364346
%Hugo Duminil-Copin sur France Inter : https://www.radiofrance.fr/franceculture/podcasts/la-conversation-scientifique/etre-mathematicien-qu-est-ce-faire-7738990


\end{document}