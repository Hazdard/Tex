\documentclass[11pt,a4paper]{article}

\usepackage{../jedusor}
	
\renewcommand{\headrulewidth}{1pt} 
\renewcommand{\footrulewidth}{1pt}	
\fancyhead[C]{}
\fancyhead[L]{}
\fancyhead[R]{}
\fancyfoot[C]{\thepage} 
\fancyfoot[L]{Sacha Ben-Arous}
\fancyfoot[R]{E.N.S Paris-Saclay}

\begin{document}

% Savoir qu’est ce que c’est plutot que comment ça marche ? Savoir à quoi ça sert, pourquoi on le fait ?
% Adopte la structure minimale. Faire des exemples et les dévisser. Partir d’un problème central et développer des outils pour le résoudre, plutôt que l’inverse : se borner à utiliser un outil que l’on trouve cool pour résoudre des trucs randoms.

%Penser à : enveloppe convexe ; dualité ; entropie ; minimisation de l'énergie 
%Se rammener a prouver les props sur des trucs engendrant tout le monde ; un pi-syst ; indicatrices ; base de topo/voisinage ; partie dense ; etc ...
%Commencer par la fin. Essayer des cas particuliers, des cas triviaux, des problèmes proches qui ont la même nature. Comprendre à fond ces situations simples. Pour une conjecture, alterner entre les tentatives de preuve et de réfutation.
%Faire une fiche histoire des maths (et mettre Bourbaki, livre Villani, Ghys, etc ...)

%Mettre des contre-exemples quand c'est intéressant et utile.
%Mettre les résultats de TD dans les fiches.
%Rajouter les bonnes références dans les fiches et surtout dans le bestiaire.

%-------------------------------------------------SOURCES-----------------------------------------------%
%http://www-math.univ-poitiers.fr/~jlehec/
%https://www.math.ucla.edu/~tao/
%https://terrytao.wordpress.com/category/teaching/254a-ergodic-theory/page/2/
%https://terrytao.wordpress.com/category/teaching/255b-incompressible-euler-equations/
%https://terrytao.wordpress.com/tag/finite-time-blowup/
%https://djalil.chafai.net/blog/?s=julia

% Probabilités : LeGall
% Topo + Analyse : L3 --> Ulm --> Début Rudin --> M1 caca --> Villani IAF --> Alazard --> PGerard
% Fourier : utiliser JB et JBB et Alazard et Rudin et PbsD'Evol et Antoine Levitt et Villani
%-------------------------------------------------------------------------------------------------------%


% Stage : 
% Lire Wikipédia en première approximation (Euler, Navier-Stokes, material derivative ; Helmholtz decomposition)
%Revoir EDO et Cauchy Lipschitz, et appliquer à Navier Stokes
% Revoir le produit vectoriel de sup
% Revoir prépa formule de Stokes, de Green.
% Revoir Gronwall
% Dérivé de norme( f(c(t) ) ^2 = ?? pour montrer que le gradient est orthogonal aux lignes de niveaux.
% Récupérer le TD d'EDO pour simuler des solutions, par exemple celle basiques données au chapitre 1 qui sont égales à leur dev d'ordre 1 (translation + rotation + déformation).
% Chercher les grands theoremes qui disent que les solutions sont régulières pour NS



%EDP :
% RELIRE LE COURS EN CHECKANT SI ON A BIEN QUE LES OPÉRATEURS SONT FERMÉS POUR AVOIR QUE LE DOMAINE DE L'ADJOINT EST DENSE (ET DONC QU'ON PEUT CONSIDÉRER  LE DOUBLE ADJOINT).
%Preuve Feller-Miyadera-Phillips
%Comprendre exemples du cours (voir poly pour peut-être plus de détails).
% Mettre qqpart avec preuve le lemme sur la dérivée d'une distrib nulle => distrib cste


%Géométrie : suivre avec le poly pour compléments et faire exos poly avec leur corrigés
% exo recollements
%Poly bonne def simplement connexe
%morphisme avec type d'homotopie
%VOIR BONNE DEF DE REVETEMENT POLY (page 32 prop 2.1
%revêtement => homéo local
%homéo local + conditions => revêtement (voir poly pour explication)
%refaire le cours de L3 avec TD à partir d'inversion locale et comparer les caractérisation de sous-variétés. Rajouter que : la dimension d'une sous-variété est unique.
%Mettre les résultats et la méthodologie de la fin du DM1 dans la fiche
%Pourquoi gradient orthogonal au plan tangent ?


%Théorie spectrale :
%Lire le poly, le cours est guez.


%Miscellanous :
% https://terrytao.wordpress.com/2011/04/10/a-proof-of-the-fredholm-alternative/
% https://terrytao.wordpress.com/career-advice/write-down-what-youve-done/
% https://terrytao.wordpress.com/career-advice/learn-and-relearn-your-field/
% Bruit en arrière, ia generative pour dechiffrer crypto/ des mdp ?


\section{Questions \& Remarques}
\section{Exos}
\begin{itemize}
\item[-] montrer que les espaces sont bien des Frechet (genre Schwartz), cf cours d'analyse.
\item[-] Riemann integrable ssi discontinuités négligeables + lien avec Lusin
\item[-] Espace localement compact mais pas complet (pas EV à cause de Riesz) ? Distinction entre polonais et loc. compact + séparables 
\end{itemize}

\section{Revisions}
\begin{itemize}
\item[-] Gronwall ; formules de Taylor avec preuves.
\item[-] Resultat distrib sur R si distrib dérivée nulle alors distrib constante, et en déduire que $H^1(\R) \subset C^0(\R)$.
\end{itemize}

\section{Questions wtf}
\begin{itemize}
\item[-] Pourquoi la distance dans la vraie vie est la distance euclidienne ? Cela se traduit par exemple avec les planètes qui, pour minimiser le potentiel de gravité, prennent la forme de sphères (euclidiennes). \\
\item[-] Explication conceptuelle au fait que Fourier convertit parfaitement (isomorphisme isométrique) le continu, certe périodique, vers le discret ? (En fait c'est encore pire que ça car tous les Hilbert séparables sont isométriquement isomorphes, donc pas besoin d'être périodique).
\end{itemize}

\end{document}
