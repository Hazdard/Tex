\documentclass[11pt,a4paper]{article}

\usepackage{../jedusor}
	
\renewcommand{\headrulewidth}{1pt} 
\renewcommand{\footrulewidth}{1pt}	
\fancyhead[C]{}
\fancyhead[L]{}
\fancyhead[R]{}
\fancyfoot[C]{\thepage} 
\fancyfoot[L]{Sacha Ben-Arous}
\fancyfoot[R]{E.N.S Paris-Saclay}

\begin{document}

% Savoir qu’est ce que c’est plutot que comment ça marche ? Savoir à quoi ça sert, pourquoi on le fait ? Comprendre la nature des objets. Le plus important ce sont les définitions, et comprendre pourquoi on les a choisies comme telles.
%CCIF Comprehension curiosite imagination force
%Se poser les bonnes questions, avoir l’esprit large pour relier deux domaines lointains, puis avoir la force technique pour le faire en vrai.
% Quand on lit un papier, comprendre le paysage dans lequel il s’inscrit et quelle est la nouveauté ?
% Adopte la structure minimale. Faire des exemples et les dévisser. Partir d’un problème central et développer des outils pour le résoudre, plutôt que l’inverse : se borner à utiliser un outil que l’on trouve cool pour résoudre des trucs randoms.

%Penser à : enveloppe convexe ; dualité ; entropie ; minimisation de l'énergie 
%Se rammener a prouver les props sur des trucs engendrant tout le monde ; un pi-syst ; indicatrices ; base de topo/voisinage ; partie dense ; etc ...
%Commencer par la fin. Essayer des cas particuliers, des cas triviaux, des problèmes proches qui ont la même nature. Comprendre à fond ces situations simples. Pour une conjecture, alterner entre les tentatives de preuve et de réfutation.

% En fait le but de la science c'est de trouver le w à partir duquel l'univers a été généré (w=XEU ?)

%-- IMPORTANT (à mettre en plus gros) ---%

%Mettre des contre-exemples quand c'est intéressant et utile.
%Mettre les résultats de TD dans les fiches.
%Mettre les bonnes références dans les fiches et surtout dans le bestiaire.
%Regarder Wikipedia pour les nouveaux sujets
%Mettre les motivations historiques, l'évolution de la discipline

%-------------------------------------------------SOURCES-----------------------------------------------%
%simpletex.net
%http://www-math.univ-poitiers.fr/~jlehec/
%https://www.math.ucla.edu/~tao/
%https://terrytao.wordpress.com/category/teaching/254a-ergodic-theory/page/2/
%https://terrytao.wordpress.com/category/teaching/255b-incompressible-euler-equations/
%https://terrytao.wordpress.com/tag/finite-time-blowup/
%https://djalil.chafai.net/blog/?s=julia
%https://www.youtube.com/@mathematiquesdeconfinees8417/videos

% Probabilités : LeGall
% Topo + Analyse : L3 --> Ulm --> Début Rudin --> M1 nul --> Villani IAF --> livre Alazard --> cours PGerard 
% Fourier : utiliser JB et JBB et Alazard et Rudin et PbsD'Evol et Antoine Levitt et Villani ; et surtout Trigonometric series de Zygmund
% En analyse plus avancé, regarder Dispersive equations de Terence Tao, le début regroupe plein de technique de base très pratiques, l'appendice fait la bonne présentation de l'analyse et des injections de Sobolev.
% Alg Lin : cours fpi le goat (+TD !!) ; livre Lax
%-------------------------------------------------------------------------------------------------------%

% Ficher ce que j'ai vu en stage + déplacer les livres et papiers dans biblio !!!

%algèbre : dm de spé, début avec permutations, polynomes (et leurs racines), puis algèbre linéaire groupe de matrices. Réviser aussi un peu Sylow, et cours chelou Laszlo.
% Gauss : https://www.youtube.com/watch?v=swGgHLnG6Eg
% Cartan von-Neumann : https://www.youtube.com/watch?v=FaCaM6v9Q2o 
% Histoire : trouver de racines de polynômes, ...

%Fourier : (Analyse harmonique)
% Prendre les refs d'analyse ci-dessus, et surtout lire trigonometric series de Zygmund.
% Regarder la théorie des fonction harmoniques du coup ...


%Miscellanous :
% "Nirenberg and Joe Kohn created the theory of pseudo-differential operators, an extended version of the Calderon-Zygmund singular integral operators. c'est aux edp à coefficients variables ce que Fourier est aux edp avec des opérateurs aux coefficients constants." (Biographie de Peter Lax).
% https://terrytao.wordpress.com/2011/04/10/a-proof-of-the-fredholm-alternative/
% https://terrytao.wordpress.com/career-advice/write-down-what-youve-done/
% https://terrytao.wordpress.com/career-advice/learn-and-relearn-your-field/
% Bruit en arrière, ia generative pour dechiffrer crypto/ des mdp ? Il faut tomber sur un truc bien précis, donc perte de régularité, cf brownien conditionné pour aller à 0 en t=1 ?
%Faire une fiche histoire des maths (et mettre Bourbaki, livre Villani, etc ...)


\section{Questions \& Remarques}

\section{Exos}
\begin{itemize}
\item[-] montrer que les espaces sont bien des Frechet (genre Schwartz), cf cours d'analyse.
\item[-] Riemann integrable ssi discontinuités négligeables + lien avec Lusin
\item[-] Espace localement compact mais pas complet (pas EV à cause de Riesz) ? Distinction entre polonais et loc. compact + séparables 
\end{itemize}

\section{Revisions}
\begin{itemize}
\item[-] Gronwall ; formules de Taylor avec preuves.
\item[-] Resultat distrib sur R si distrib dérivée nulle alors distrib constante, et en déduire que $H^1(\R) \subset C^0(\R)$.
\end{itemize}

\section{Questions wtf}
\begin{itemize}
\item[-] Pourquoi la distance dans la vraie vie est la distance euclidienne ? Cela se traduit par exemple avec les planètes qui, pour minimiser le potentiel de gravité, prennent la forme de sphères (euclidiennes). \\
\item[-] Explication conceptuelle au fait que Fourier convertit parfaitement (isomorphisme isométrique) le continu, certe périodique, vers le discret ? (En fait c'est encore pire que ça car tous les Hilbert séparables sont isométriquement isomorphes, donc pas besoin d'être périodique).
\item[-] Pourquoi le temps serait continu ? On serait incapable de percevoir si ce n’était pas le cas, donc ce n’est pas nécessaire pour expliquer les phénomènes que l’on observe. 
\end{itemize}

\end{document}
