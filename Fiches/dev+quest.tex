\documentclass[11pt,a4paper]{article}

\usepackage{../jedusor}
	
\renewcommand{\headrulewidth}{1pt} 
\renewcommand{\footrulewidth}{1pt}	
\fancyhead[C]{}
\fancyhead[L]{}
\fancyhead[R]{}
\fancyfoot[C]{\thepage} 
\fancyfoot[L]{Sacha Ben-Arous}
\fancyfoot[R]{E.N.S Paris-Saclay}

\begin{document}



%EDP :
% RELIRE LE COURS EN CHECKANT SI ON A BIEN QUE LES OPÉRATEURS SONT FERMÉS POUR AVOIR QUE LE DOMAINE DE L'ADJOINT EST DENSE (ET DONC QU'ON PEUT CONSIDÉRER  LE DOUBLE ADJOINT).
%Preuve Feller-Miyadera-Phillips
%Comprendre exemples du cours (voir poly pour peut-être plus de détails).



%Géométrie : suivre avec le poly pour compléments et faire exos poly avec leur corrigés
% exo recollements
%Poly bonne def simplement connexe
%morphisme avec type d'homotopie
%VOIR BONNE DEF DE REVETEMENT POLY (page 32 prop 2.1
%revêtement => homéo local
%homéo local + conditions => revêtement (voir poly pour explication)
%refaire le cours de L3 avec TD à partir d'inversion locale et comparer les caractérisation de sous-variétés. Rajouter que : la dimension d'une sous-variété est unique.

%Théorie spectrale :
%Lire le poly, le cours est guez.







\section{Questions \& Remarques}
\section{Exos}
\begin{itemize}
\item[-] montrer que les espaces sont bien des Frechet (genre Schwartz), cf cours d'analyse.
\item[-] Riemann integrable ssi discontinuités négligeables + lien avec Lusin
\item[-] Espace localement compact mais pas complet (pas EV à cause de Riesz) ? Distinction entre polonais et loc. compact + séparables 
\end{itemize}

\section{Revisions}
\begin{itemize}
\item[-] Gronwall ; formules de Taylor avec preuves.
\item[-] Resultat distrib sur R si distrib dérivée nulle alors distrib constante, et en déduire que $H^1(\R) \subset C^0(\R)$.
\end{itemize}

\section{Questions wtf}
\begin{itemize}
\item[-] Pourquoi la distance dans la vraie vie est la distance euclidienne ? Cela se traduit par exemple avec les planètes qui, pour minimiser le potentiel de gravité, prennent la forme de sphères (euclidiennes). \\
\item[-] Explication conceptuelle au fait que Fourier convertit parfaitement (isomorphisme isométrique) le continu, certe périodique, vers le discret ? (En fait c'est encore pire que ça car tous les Hilbert séparables sont isométriquement isomorphes, donc pas besoin d'être périodique).
\end{itemize}

\end{document}
