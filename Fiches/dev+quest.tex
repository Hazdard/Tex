\documentclass[11pt,a4paper]{article}

\usepackage{../jedusor}
	
\renewcommand{\headrulewidth}{1pt} 
\renewcommand{\footrulewidth}{1pt}	
\fancyhead[C]{}
\fancyhead[L]{}
\fancyhead[R]{}
\fancyfoot[C]{\thepage} 
\fancyfoot[L]{Sacha Ben-Arous}
\fancyfoot[R]{E.N.S Paris-Saclay}

\begin{document}

%-------------------------------------------------MÉTHODOLOGIE------------------------------------------------------%

% Savoir qu’est ce que c’est plutot que comment ça marche ? Savoir à quoi ça sert, pourquoi on le fait ? Comprendre la nature des objets. Le plus important ce sont les définitions, et comprendre pourquoi on les a choisies comme telles.
% Se demander quelle est l'etape d'apres, ou ca nous mene ? Essayer de refaire le cours/l'intro des objets fondamentaux d'un theme sans regarder le cours.
% Trouver des c.e.x aux th en retirant des hyp. Se demander si des reciproques sont vraies.
% Ce que l'on sait faire est facile !!!
% L'analyse, les probas, et les maths en général ne sont " qu'un " outil pour comprendre, décrire et expliquer les phénomènes à l'oeuvre dans la nature.

% CCIF Comprehension curiosite imagination force
% Se poser les bonnes questions, avoir l’esprit large pour relier deux domaines lointains, puis avoir la force technique pour le faire en vrai.
% Quand on lit un papier, comprendre le paysage dans lequel il s’inscrit et quelle est la nouveauté ?
% Se demander quelle est l’étape d’après quand on suit une conf, un livre, un cours, … Quels sont les résultats qu’on veut ? Quels sont ceux qu’on peut avoir naturellement ? Comment faire les preuves ?
% Quand on lit un article, comprendre le paysage dans lequel le papier s’inscrit. Quelle est la nouveauté présente dans le papier ?


% Se rammener a prouver les props sur des trucs engendrant tout le monde ; un pi-syst ; indicatrices ; base de topo/voisinage ; partie dense ; etc ...
% Commencer par la fin. Essayer des cas particuliers, des cas triviaux, des problèmes proches qui ont la même nature. Comprendre à fond ces situations simples. Pour une conjecture, alterner entre les tentatives de preuve et de réfutation.
% Adopte la structure minimale. Faire des exemples et les dévisser. Partir d’un problème central et développer des outils pour le résoudre, plutôt que l’inverse, càd se borner à utiliser un outil que l’on trouve cool pour résoudre des trucs randoms.


% Regarder Wikipedia pour les nouveaux sujets.
% Mettre les motivations historiques, l'évolution de la discipline.
% Mettre les résultats de TD dans les fiches.
% Mettre des contre-exemples quand c'est intéressant et utile.
% Mettre les bonnes références dans les fiches et surtout dans le bestiaire.

%------------------------------------------------- Questions et problèmes-----------------------------------------------%

% Preuve cv proba donne cv loi par unique val d'adh
% Truc Ulm element dans F_t+ mais pas F_t cf exo thm limite

% Voir ineg maximale en syst dyn (p27)
% Existence/Définition tribu et mesure produit infini (dénombrable) ? Existence mesure semi produit (=annealed) ?


% A prouver dans bestiaire :
% Existence d'une famille infinie de v.a. indépendantes.

%Répondu :
% Processus cad ssi filtration can est cad ? Non les deux sens sont faux, l'un en prenant une fonction irrégulière déterministe, et l'autre en considérant le mouvement brownien et F_0 != F_0+ (sinon critère vu en large dev)

%------------------------------------------------- %TODO --------------------------------------------------------------%

% Ré-écrire cette fiche en triant tout bien, et fusionner avec fiches téléphone. Le téléphone sert juste à noter en première approche, puis tout est mis au propre dans Latex.

% Work in progress : 
% th mesure : retirer le numéro et point du thm et de la defin dans motivations
% tout le reste des autres fiches (choisir un ordre)

% Calcul Stochastique : pour calculer des esperance cond, on montre d'abord que c'est gaussien, et ensuite c'est facile car on juste besoin de la variance (on utilise la prop de proj) (moyenne souvent nulle) ; Preuve dev limité transformée de Fourier (cf début livre legall) ?

% RWRM : aller voir thm bonus kesten balistic qui donne la vitesse dans le cas transient en milieu aléatoire ; Relire debut RWRM et prop f(X_n) martingale ssi f harmonique, cf formule d'ito ou tu veut annuler le terme en dt.

% Limit theorems : Exo tribu cv unif pas egale tribu cvs ; continuous mapping thm (cv en loi mais avec des fonctions irrégulières aux points de discontinuité de la fonction de répartition) ; Remarque reg mesure ferme ouvert et pas l'inverse

% Ficher mail G cours M2

% Wikipédia : Lagrangian mechanics ; Lagrange multiplier ; Implicit function theorem ; Théorème de Lebesgue-Vitali ; Cameron–Martin theorem

% Histoire : Louis Bachelier 
% Leray oeuvres complètes
% Biographie de Lax
% Livres de Poincaré, en particulier ceux sur la philosophie des sciences
% Vladimir Zakharov, il fait quoi exactement ? Partie compréhension profonde des phénomènes physique pour les modéliser correctement en maths ? 

%-------------------------------------------------SOURCES-------------------------------------------------------------%
%simpletex.net
%http://www-math.univ-poitiers.fr/~jlehec/
%https://www.math.ucla.edu/~tao/
%https://terrytao.wordpress.com/category/teaching/254a-ergodic-theory/page/2/
%https://djalil.chafai.net/blog/?s=julia
%https://www.youtube.com/@mathematiquesdeconfinees8417/videos

% Probabilités : LeGall cours M1 puis livre mouvement brownien
% Topo + Analyse : L3 --> Ulm --> Début Rudin --> M1 nul --> Villani IAF --> livre Alazard --> cours PGerard 
% Fourier : utiliser JB et JBB et Alazard et Rudin et PbsD'Evol et Antoine Levitt et Villani ; et surtout Trigonometric series de Zygmund
% En analyse plus avancée, regarder Dispersive equations de Terence Tao, le début regroupe plein de technique de base très pratiques, l'appendice fait la bonne présentation de l'analyse et des injections de Sobolev.
% Alg Lin : cours fpi le goat (+TD !!) ; livre Lax

%------------------------------------------------------------------------------------------------------------------------%


\section{Questions \& Remarques}

\section{Exos}
\begin{itemize}
\item[-] montrer que les espaces sont bien des Frechet (genre Schwartz), cf cours d'analyse.
\item[-] Riemann integrable ssi discontinuités négligeables + lien avec Lusin
\item[-] Espace localement compact mais pas complet (pas EV à cause de Riesz) ? Distinction entre polonais et loc. compact + séparables 
\end{itemize}

\section{Revisions}
\begin{itemize}
\item[-] Gronwall ; formules de Taylor avec preuves.
\item[-] Resultat distrib sur R si distrib dérivée nulle alors distrib constante, et en déduire que $H^1(\R) \subset C^0(\R)$.
\end{itemize}

\section{Questions wtf}
\begin{itemize}
\item[-] Pourquoi la distance dans la vraie vie est la distance euclidienne ? Cela se traduit par exemple avec les planètes qui, pour minimiser le potentiel de gravité, prennent la forme de sphères (euclidiennes). \\
\item[-] Explication conceptuelle au fait que Fourier convertit parfaitement (isomorphisme isométrique) le continu, certe périodique, vers le discret ? (En fait c'est encore pire que ça car tous les Hilbert séparables sont isométriquement isomorphes, donc pas besoin d'être périodique).
\item[-] Pourquoi le temps serait continu ? On serait incapable de percevoir si ce n’était pas le cas, donc ce n’est pas nécessaire pour expliquer les phénomènes que l’on observe. 
\end{itemize}

\end{document}
