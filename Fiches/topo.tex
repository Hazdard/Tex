\documentclass[11pt,a4paper]{article}

\usepackage{../jedusor}	

\renewcommand{\headrulewidth}{0pt} 
\renewcommand{\footrulewidth}{1pt}
\fancyhead[C]{}
\fancyhead[L]{}
\fancyhead[R]{}
\fancyfoot[C]{\thepage} 
\fancyfoot[L]{Sacha Ben-Arous}
\fancyfoot[R]{E.N.S Paris-Saclay}
	
\begin{document}
\newpage
\begin{center}  
\section*{Topologie} 
\end{center}



\subsection*{Contexte et méthode}

%Se rammener a prouver les props sur des trucs engendrant tout le monde ; un pi-syst ; indicatrices ; base de topo/voisinage ; partie dense ; etc ...

\subsection*{Définitions et propriétés élémentaires}
\begin{definstar}
Soit $E$ un ensemble, et $\mathcal{T} \subset \mathcal{P}(E)$.  On dit que $(E,\mathcal{T})$ est un espace topologique si :
\begin{enumerate}
\item $\emptyset \in \mathcal{T}$, $E \in \mathcal{T}$,
\item $\mathcal{T}$ est stable par union quelconque,
\item $\mathcal{T}$ est stable par intersection finie.
\end{enumerate}
Les éléments de $\mathcal{T}$ sont appelés des ouverts, et leurs complémentaires sont des fermés.
\end{definstar}
\begin{rmq}
Une intersection de topologie étant encore une topologie, on en 
\end{rmq}
\begin{definstar}

\end{definstar}
\subsection*{Résultats principaux}

\subsection*{Outils importants}

\subsection*{Autres résultats}

\end{document}