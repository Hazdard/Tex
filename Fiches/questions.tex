\documentclass[11pt,a4paper]{article}

\usepackage{../jedusor}
	
\renewcommand{\headrulewidth}{1pt} 
\renewcommand{\footrulewidth}{1pt}	
\fancyhead[C]{}
\fancyhead[L]{}
\fancyhead[R]{}
\fancyfoot[C]{\thepage} 
\fancyfoot[L]{Sacha Ben-Arous}
\fancyfoot[R]{E.N.S Paris-Saclay}

\begin{document}

\section{Questions \& Remarques}
\section{Exos}
\begin{itemize}
\item[-] Suites bilatères est compact pour la distance définie dans le cours 
\item[-] Regarder Smale dans l'autre cours
\item[-] montrer que les espaces sont bien des Frechet (genre Schwartz), cf cours d'analyse.
\item[-] Riemann integrable ssi discontinuités négligeables + lien avec Lusin
\item[-] Espace localement compact mais pas complet (pas EV à cause de Riesz) ? Distinction entre polonais et loc. compact + séparables 
\end{itemize}

\section{Revisions}
\begin{itemize}
\item[-] Gronwall ; formules de Taylor avec preuves.
\end{itemize}

\section{Questions wtf}
\begin{itemize}
\item[-] Pourquoi la distance dans la vraie vie est la distance euclidienne ? Cela se traduit par exemple avec les planètes qui, pour minimiser le potentiel de gravité, prennent la forme de sphères. \\
\item[-] Explication conceptuelle au fait que Fourier convertit parfaitement (isomorphisme isométrique) le continu, certe périodique, vers le discret ?
\end{itemize}

\end{document}
