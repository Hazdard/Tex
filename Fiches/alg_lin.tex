\documentclass[11pt,a4paper]{article}

\usepackage{../jedusor}	

\renewcommand{\headrulewidth}{0pt} 
\renewcommand{\footrulewidth}{1pt}
\fancyhead[C]{}
\fancyhead[L]{}
\fancyhead[R]{}
\fancyfoot[C]{\thepage} 
\fancyfoot[L]{Sacha Ben-Arous}
\fancyfoot[R]{E.N.S Paris-Saclay}
	
\begin{document}
\newpage
\begin{center}  
\section*{Algèbre linéaire} 
\end{center}


%Algèbre linéaire : Cauchy-Schwarz, inégalité triangulaire, thm spectral, Cayley-Hamilton, décomposition polaire, min-max Courant-Fischer, Gershgorin, rayon spectral, Perron-Frobenius. Mettre les décomposition de Jordan (à travailler en détails, car est la plus importante), Frobenius, Dunford, et tout le bazard vu en prépa/L3. Dire que toute forme n-linéaire alternée est proportionnelle au det. Densité de Gl_n, équivalence de matrices, continuité d'une app linéaire ssi norme d'opérateur finie (i.e lipschitz "joli"). PERRON-FROBENIUS.


\subsection*{Contexte}

\subsection*{Méthode}

\subsection*{Définitions et propriétés élémentaires}

\subsection*{Résultats principaux}

\subsection*{Outils importants}


\subsection*{Autres résultats}

\end{document}