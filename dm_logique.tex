\documentclass[11pt,a4paper]{article}
\textheight245mm
\textwidth170mm
\hoffset-21mm
\voffset-15mm
\parindent0pt
\usepackage[utf8]{inputenc}
\usepackage{amsmath,amsfonts,amssymb}
\usepackage{dsfont}
\usepackage{graphicx}
\usepackage{caption}
\usepackage{fancyhdr}
\usepackage{proof}
\pagestyle{fancy}

\newcommand\ddfrac[2]{\frac{\displaystyle #1}{\displaystyle #2}}

\usepackage{array,multirow,makecell}
\setcellgapes{4pt}
\makegapedcells
\newcolumntype{R}[1]{>{\raggedleft\arraybackslash }b{#1}}
\newcolumntype{L}[1]{>{\raggedright\arraybackslash }b{#1}}
\newcolumntype{C}[1]{>{\centering\arraybackslash }b{#1}}


\renewcommand{\headrulewidth}{1pt}
\fancyhead[C]{Devoir maison de Logique}
\fancyhead[L]{L3 - 2022/2023}
\fancyhead[R]{D.E.R Informatique}

\renewcommand{\footrulewidth}{1pt}
\fancyfoot[C]{\thepage} 
\fancyfoot[L]{Sacha Ben-Arous}
\fancyfoot[R]{E.N.S Paris-Saclay}

\begin{document}
\textbf{Exercice 1 :}\\
1) \ On commence par prouver le sens direct  : 
\[\infer[\Rightarrow_d]{\vdash ((\forall x, P(x))\Rightarrow C) \Rightarrow (\exists x, (P(x) \Rightarrow C))}
{\infer[\Rightarrow_g]{(\forall x, P(x))\Rightarrow C \vdash \exists x, P(x) \Rightarrow C}
{\infer[\forall_d]{\vdash \exists x (P(x) \Rightarrow C) , \forall x P(x)}
{\infer[\exists_d]{\vdash \exists x (P(x) \Rightarrow C) , P(a)}{\infer[\Rightarrow_d]{\vdash P(a) \Rightarrow C, P(a)}{\infer[Hyp]{P(a) \vdash C, P(a)}{}}}}\ &\ C \infer[\exists_d]{\vdash \exists x (P(x) \Rightarrow C)}{\infer[\Rightarrow_d]{C \vdash P(a) \Rightarrow C}{\infer[Hyp]{C,P(a) \vdash C}{}}}}}
\] \\
Puis le sens réciproque : 
\[\infer[\Rightarrow_d]{\vdash (\exists x, (P(x) \Rightarrow C)) \Rightarrow ((\forall x, P(x))\Rightarrow C)}{\infer[\Rightarrow_d]{\exists x, (P(x) \Rightarrow C) \vdash (\forall x, P(x))\Rightarrow C}{\infer[\exists_g]{\exists x, (P(x) \Rightarrow C), \forall x P(x) \vdash C}{\infer[\Rightarrow_g]{P(a) \Rightarrow C, \forall x P(x) \vdash C}{ \infer[\forall_g]{\forall x P(x) \vdash C, P(a)}{\infer[Hyp]{P(a) \vdash C, P(a)}{}}             &     \infer[Hyp]{C,\forall x P(x) \vdash C}{}    }}}}\]
\\

2) 
\\

3) \ C'est le sens réciproque qui est prouvable de manière intuitionniste.
\\

4) \ On va montrer que $((\forall x, P(x))\Rightarrow C) \Rightarrow (\exists x, (P(x) \Rightarrow C)) $ n'est pas prouvable en logique intuitionniste. Pour cela, on considère deux interprétations $I$ et $J$, de domaines $\mathcal{D}_I = \{x\}$ et $\mathcal{D}_J =\{x,y\}$, avec les valeures de vérité suivantes :
\begin{enumerate}
\item $I \models P(x)$
\item $I \not\models C$
\item $J \models P(x)$
\item $J \not\models P(y)$
\item $J \models C$
\end{enumerate} 
On a bien que $I \preceq J$. On remarque que $J \models ((\forall x, P(x))\Rightarrow C)$ (car l'hypothèse de l'implication est fausse, puisque $J \not\models P(y)$), donc $I \not\models ((\forall x, P(x))\Rightarrow C)$. Mais de plus, comme $C$ est faux dans $I$ mais que $I \models P(x)$, on a que $I \not\models \exists x, (P(x) \Rightarrow C) $. Ainsi, $ I \not\models (\forall x, P(x))\Rightarrow C) \Rightarrow (\exists x, (P(x) \Rightarrow C) $. \\
On a trouvé un modèle de Kripke qui rend le sens réciproque faux, ce dernier n'est donc pas démontrable de manière intuitionniste. \\

\textbf{Exercice 2 :} \\
1) \begin{enumerate}
\item[(a)] $\forall x, \exists y,z, (P(y,x)\land P(z,x)\land y\neq z)\land (\forall t,P(t,x) \Rightarrow (t =y \lor t=z))$
\item[(b)] $\forall x, (\forall y, P(x,y) \Rightarrow V(y)) \Rightarrow H(x) $
\item[(c)] $\forall x, (\exists y, P(y,x) \land B(y)) \Rightarrow B(x) $
\item[(d)] $\forall x, H(x) \Rightarrow V(x) $ \\
\end{enumerate}

2) \begin{itemize}
\item[(a)] Les dragons qui volent et qui sont parents ont au plus un enfant.
\item[(b)] Tout dragon a un enfant heureux. \\
\end{itemize}

3)  On commence par mettre en FNN les formules $A,B,C,\lnot D$ :
\begin{itemize}
\item $A \equiv \forall x, (\exists y, P(x,y)\land \lnot V(y)) \lor H(x) $
\item $B \equiv \forall x, \lnot B(x) \lor V(x)$
\item $C \equiv \forall x, (\forall y, \lnot B(y) \lor \lnot P(y,x)) \lor B(x)$
\item $\lnot D \equiv \exists x, B(x) \land \lnot H(x)$ \\
\end{itemize}
On passe ensuite à l'étape de skolémisation. Pour ce faire, on introduit un nouveau symbole de fonction unaire $f$, ainsi qu'une constante $a$. $B, C$ sont inchangés, et $A$ et $\lnot D$ deviennent respectivement :
\begin{itemize}
\item $\forall x, (P(x,f(x))\land \lnot V(f(x))) \lor H(x)$
\item $B(a) \land \lnot H(a) $ \\
\end{itemize}

Ensuite, on met en les formuels en FNC pour obtenir les clauses suivantes :
\begin{itemize}
\item[(1)] $P(x,f(x))\lor H(x)$
\item[(2)] $\lnot V(f(x)) \lor H(x) $
\item[(3)] $\lnot B(x) \lor V(x) $
\item[(4)] $\lnot B(y) \lor \lnot P(y,x) \lor B(x) $
\item[(5)] $B(a)$
\item[(6)] $ \lnot H(a) $ 
\end{itemize}

Finalement, on constate que : $1$ et $6$, en substituant $a$ à $x$, donnent la clause $c_1 := P(a,f(a))$. Ensuite, $c_1$ et $4$, où l'on remplace $x$ par $f(a)$ et $y$ par $a$, donnent $c_2 := \lnot B(a) \lor B(f(a))$. Alors, $5$ et $c_2$ où $x$ est remplacé par $a$ donnent $c_3 := B(f(a))$. Puis $c_3$ et $3$ donnent $c_4 := V(f(a))$, qui combiné à $2$ donnent $c_5 := H(a)$ qui finalement, en se combinant avec $6$ aboutit à la clause vide, et donc à la contradiction voulue. \\ 

5) 
\begin{tabular}{|C{1cm}|C{1cm}|C{1cm}|C{1.2cm}|}
  \hline
   & Puff & Draco & Saphira \\
  \hline
  \textbf{H} & $V$ & $V$ & $F$ \\
  \hline
  \textbf{B} & $V$ & $F$ & $V$ \\
  \hline 
  \textbf{V} & $F$ & $F$ & $V$ \\
  \hline
\end{tabular}
\\
\\
\\
\\
\\

\textbf{Exercice 4 :} \\

1) \begin{itemize}
\item $A_0:=\top$
\item Pour $n \geq 1, $ $\displaystyle A_n := \exists x_1,\dots,x_n \ \bigwedge_{1\leq i<j \leq n}\lnot(x_i=x_j)$
\end{itemize} 
Il suffit de poser ensuite $B_n := \lnot A_{n+1}$ . \\

2) On sait que $F$ peut s'écrire sous forme normale disjonctive : $\displaystyle \bigvee \bigwedge_{i \in I} C_i $ où les $C_i$ sont de la forme 
\begin{enumerate}
\item $x_k=x_l$
\item $x_k\neq x_l$
\item $x\neq x_k$
\item $x=x_k$
\end{enumerate}
Quitte à renommer les termes, on sépare chacunes des conjonctions en trois morceaux : un qui ne comprend aucun $x$, un avec uniquement des formules de la forme $(3)$, et un dernier avec les formules de la forme $(4)$. \\ 
Ensuite, en utilisant que ($\exists x P(x) \lor Q) \Leftrightarrow (\exists x P(x))\lor Q$, on peut distribuer et simplifier le $\exists$ pour traiter le premier morceau de chaque conjonction. \\
 Finalement, en rassemblant les termes similaires,  il reste à traiter les termes de la forme $\exists x, \bigwedge_{i \in I} x\neq x_i$ ou $\exists x, \bigwedge_{i \in I} x= x_i$. Dans le premier cas, c'est le terme voulu, et dans le second, on peut choisir un $i\in I$, et on ré-écrit les égalités en remplaçant $x$ par $x_i$, ce qui donne une formule équivalente qui ne dépend plus de $x$, et on élimine alors le $\exists$ pour obtenir la forme voulue. \\
 
 3) a- Il y a $5$ partitions possibles : $\{\{1\},\{2,3\}\}$ ; $\{\{2\},\{1,3\}\}$ ; $\{\{3\},\{1,2\}\}$ ; $\{\{1,2,3\}\}$ et $\{\{1\},\{2\},\{3\}\}$. \\
Dans le premier cas ($2$ éléments), $\chi_p = (x_2 = x_3)\land (x_1 \neq x_2 \land x_1 \neq x_3)$ \\
Dans l'avant-dernier cas ($1$ élément), $\chi_p = (x_1 = x_2)\land(x_1=x_3)\land(x_2=x_3)$\\
Dans le dernier cas ($3$ élément), $\chi_p = (x_1 \neq x_2)\land(x_1 \neq x_3)\land(x_2 \neq x_3)$\\

b- Dans le sens direct, cette formule exprime que, en se donnant une partition d'un ensemble $\{x_1,\dots,x_n\}$ en $p$ classes, on a moins $p$ élément distincts dans cet ensemble, et donc si on trouve un nouvel élément distinct de ceux considérés précédemment, on a alors au moins $p+1$ éléments distincts. \\
Dans le sens réciproque, cette formule signifie que, toujours en se donnant une partition d'un ensemble $\{x_1,\dots,x_n\}$ en $p$ classes, si l'on sait qu'il existe au moins $p+1$ éléments distincts, ceux précédemments évoqués se réduisant à exactement $p$ éléments distincts, il doit alors exister un autre élément différent des $x_i$ pour qu'il y ait au moins $p+1$ éléments distincts dans l'interprétation.  \\

c- La relation $=$ étant une relation d'équivalence, on peut partitionner l'ensemble des éléments par cette relation, et ainsi obtenir la partition associée. Alors au moins un terme dans la disjonction est vrai, donc la disjonction globale est vraie. \\

d- Pour montrer le sens direct de l'équivalence, on suppose donc que $\exists x, \bigwedge_{i \in I} x \neq x_i$ est vraie et on veut montrer que pour toute partition $P$, $\chi_P \Rightarrow A_{p+1}$, on suppose donc aussi que $\chi_P$ est vraie. Alors, on a directement que $A_{p+1}$ est vraie en appliquant le sens direct de la question 3)b-. \\
Réciproquement, on suppose que la conjonction $\bigwedge \chi_P \Rightarrow A_{p+1}$ est vraie. Alors, en utilisant le résultat de 3)c-, on a qu'il existe au moins une partition $P$ telle que $\chi_p$ est vraie. Alors par hypothèse, comme $\chi_P \Rightarrow A_{p+1}$, on a que $A_{p+1}$ est vraie, et donc en appliquant le sens réciproque de 3)b-, on a bien que $\exists x, \bigwedge_{i \in I} x \neq x_i$ est vraie. \\

4) \\

5) \\

6)  Raisonnons par l'absurde en supposant qu'il existe un telle formule $F$. Alors avec la question 5), on obtient l'existence d'un booléen $b_F$ ainsi que d'un ensemble fini $C_F$ qui vérifient que pour toute interprétation, $(I \models F)$ si et seulement si, $b_F$ est vrai et $card(I) \in C_F$, ou $b_F$ est faux et $card(I) \notin C_F$. \\

\underline{Premier cas :} Si $b_F$ ets vrai, alors pour toute interprétation de domaine paire $J$, on a $J \models F$ donc $card(J) \in C_F$, donc $2\mathbb{N} \subset C_F$, ce qui contredit le fait que $C_F$ soit fini. \\

\underline{Second cas :} Si $b_F$ est faux, alors pour toute interprétation de domaine impaire $J$, on a $J \models F$ donc $card(J) \in C_F$, donc $C_F$ contient tous les entiers impairs, ce qui contredit encore sa finitude.
\end{document}





































