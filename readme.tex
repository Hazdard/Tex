\documentclass[10pt,a4paper]{article}
\textheight245mm
\textwidth170mm
\hoffset-21mm
\voffset-15mm
\parindent0pt
\usepackage[utf8]{inputenc}
\usepackage{amsmath,amsfonts,amssymb}
\usepackage{graphicx}
\usepackage{caption}
\usepackage{fancyhdr}
\pagestyle{fancy}

\renewcommand{\headrulewidth}{1pt}
\fancyhead[C]{Projet Shell}
\fancyhead[L]{L3 - 2022/2023}
\fancyhead[R]{D.E.R Informatique}

\renewcommand{\footrulewidth}{1pt}
\fancyfoot[C]{\thepage} 
\fancyfoot[L]{Sacha Ben-Arous}
\fancyfoot[R]{E.N.S Paris-Saclay}

\begin{document}

\section{Compilation}
La compilation s'effectue avec la commande \textbf{make}. Pour utiliser le shell, il suffit alors de lancer la commande \textbf{./shell} .

\section{Réponses :}
\begin{enumerate}
	\item Un exemple correspondant au cas C\_PLAIN est la commande suivante : \textbf{myshell $>$ ls}. J'ai choisi d'utiliser la commande \textbf{execvp}, qui a l'avantage de directement utiliser les fonctions indexées par la variable globale \textbf{\$PATH}.
	\item En bash, la commande pour une séquence de commandes est \textbf{;} . Cet opérateur se comporte différemment du \textbf{and}, comme on le constate avec les deux exemple suivants : \textbf{\$ false ; echo "Je suis affichée"} aboutira bien à l'execution du \textbf{echo}, tandis que : \textbf{\$ false and echo "Je suis affichée"} s'arretera au \textbf{false}, et renverra le code d'erreur 1.
	\item \
	\item L'utilisation des parenthèses permet de regrouper les commandes ensemble, et ainsi rediriger d'un seul coup toutes les erreurs, via \textbf{2$>$}, dans \textbf{/dev/null} . Voici un exemple : 
	\\ \textbf{ (grep 42 adresses.txt and echo "adresse interessante trouvée") 2$>$ /dev/null }
	\\ Cette commande va chercher si au moins une des adresses vérifie une condition spécifique, si c'est le cas elle le signale, et sinon les erreurs renvoyées par grep ne sont pas affichées.
	\item Quand on tape Ctrl + C , le shell se ferme.
	\item La commande \textbf{ls $>$ dump} a pour effet de renvoyer l'erreur suivante : \\ \textbf{error: I do not know how to redirect, please help me!}
	\item Une utilisation naïve de \textbf{dup2} conduirait à de mauvaises redirections lorsque au moins 3 commandes sont chainées, et la résolution de ces problèmes conduirait à un usage d'un grand nombre de \textbf{dup2}, il vaut donc mieux directement utiliser la fonction \textbf{pipe}.
\end{enumerate}
\section{Exemples }
	Voici quelques exemples d'une utilisation basique du shell :
\begin{enumerate}
	\item \textbf{myshell $>$} ls \\ global.h  lex.c  lex.l  main.c  main.o  Makefile  output.c  output.o  parse.c  parse.o  parse.y  readme.pdf  shell
	\item \textbf{myshell $>$} echo 42 \textbar \ cat \\ 42
	\item \textbf{myshell $>$} echo "Je suis un test" $>$ fichier.txt ; cat fichier.txt \\ Je suis un test
\end{enumerate}
\end{document}
