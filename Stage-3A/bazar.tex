\documentclass[11pt,a4paper]{article}

\usepackage{../jedusor}
	
\renewcommand{\headrulewidth}{1pt} 
\renewcommand{\footrulewidth}{1pt}	
\fancyhead[C]{}
\fancyhead[L]{}
\fancyhead[R]{}
\fancyfoot[C]{\thepage} 
\fancyfoot[L]{Sacha Ben-Arous}
\fancyfoot[R]{E.N.S Paris-Saclay}

\begin{document}

% Comprendre le paysage dans lequel le papier s’inscrit, quelle est la nouveauté ?
% Chercher les grands theoremes qui disent que les solutions sont régulières pour NS
% Quel est l'énoncé exact du problème de NS ?
% Vision probabiliste, champ de particules, modélisation particule-vorticité
%S’entrainer a expliquer le papier



% Comprendre comment dériver Euler (via wiki) et faire les calculs pour aller en polaire (papier1)
% Revoir cours sur l'eq des ondes et faire le lien avec le papier et les invariants de Riemann (papier1)
% Comprendre les méthode avec le flot Lagrangien (papier 1 corps et annexe, majda bertozzi ?, slides, ...)
% Comprendre le sens des lemmes en annexe du papier1
% Pourquoi le papier sur la dim3 est aussi gros ? La p7 sur la linéarisation revient à avoir une borne type lemme A.1 dans le premier papier ?




% La méthode de bootstrap /continuité revient à montrer qu'une zone "barrière" est interdite, et comme le système évolue continuement, il reste forcément d'un des côtés de la barrière, qui dépend juste de l'état initial.
% expliquer pourquoi le damping damp : lutte contre la dérivéée temporelle, et lemme A.1 donne une estimée précise.
% Deux trucs dans le papier : géométrie de la singularité stable (au sens C^4 pour la géométrie asymptotique auto similaire du choc) ; et purement azymuthale (en se ramenant à la sol 1D) ; et viscosite non nulle et bornee.














% Majda-Bertozzi avec fiches -> Eggers-Fontelos -> Tao chap1 et annexe -> papier et articles connexes 
% https://www.youtube.com/watch?v=mw_hhOqKDx4 (T.B. intro papier 2D)
% https://www.youtube.com/watch?v=g-qyjXOLV4o  (László Székelyhidi) 
% https://www.youtube.com/watch?v=o3NYsxCr7eg&list=PLmsGGxFhM5bzUCRCjQbQtolk-yzB2eFqn&index=27  (Max Planck Science)
% https://vimeo.com/539376697/bbc4763e80 (T.B. intro papier 3D)
% https://www.youtube.com/watch?app=desktop&v=JRBwdVFmcE4 (T.B. exemple conf avec op linéarisé) (en chercher d'autres)


% Lire Wikipédia en première approximation : 
% Euler, Navier-Stokes, material derivative ; wave equation (pour comprendre le lien avec invariants de riemann) ; Helmholtz decomposition, singular integral operator (et Calderon-Zygmund), Riemann invariants, 
% method of characteristics ; Polar coordinate system ; Rankine–Hugoniot conditions ; RKHS operator ; Hilbert–Schmidt operator


% Revoir EDO et Cauchy Lipschitz
% Revoir prépa formule de Stokes, de Green, th de la divergence.
% Revoir Gronwall



\end{document}
