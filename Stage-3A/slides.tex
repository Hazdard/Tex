\documentclass[10pt]{beamer}
\usetheme{metropolis}           % Use metropolis theme
\usepackage[utf8]{inputenc}
\usepackage{amsmath,amsfonts,amssymb}
\usepackage{dsfont}
\usepackage{graphicx}
\usepackage{caption}
\usefonttheme[onlymath]{serif}


\title{My title}
\date{\today}
\author{Sacha Ben-Arous}
\institute{ENS Paris-Saclay}
\begin{document}
  \maketitle
  

% Dans les slides : reprendre bootstrap MSRI pour convergence en temps long
  
% début : dire que j'étais à NY puis Princeton, que j'ai fait mon stage avec Tristan Buckmaster et ses doctorants, présenter Tristan.
% ensuite faire l'histoire de mécanique des fluides, les motivations philosophiques et physiques (comprendre les failles des modélisations car elles représentent les failles de notre compréhension), écrire Navier-Stokes et Euler avec le problème du millénaire pour faire saisir les enjeux.

% Intro aux singularités : dire qu'on se contente de Burgers, expliquer pourquoi c'est pertinent. Suivre les caractéristiques avec calcul, montrer que dès que pente négative, blowup.
% Faire dessin des caractéristique et vidéo de l'évolution qui atteint une singularité.

% Intro au self-similar : montrer la vidéo de zoom. Dire que l'interêt de self-similaire c'est de transformer une situation singulière en une situation régulière, ce qui n'est pas contradictoire car le changement de variable lui même est singulier.
%TODO : demander à Tristan quelles sont les motivations pour self-similaire, qu'est ce qui a guidé vers ce choix de théorie ?
% Écrire l'equation auto similaire et expliquer comment on choisi alpha et beta, donner les deux critères qui imposent lambda =1/(2i+2) pour i entier.

  
\begin{frame}
\tableofcontents
\end{frame}  
\section{First Section}

\begin{frame}{First Frame}
    Hello, world!
\end{frame}
\end{document}