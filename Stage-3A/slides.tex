\documentclass[10pt]{beamer}
\usetheme{metropolis}           % Use metropolis theme
\usepackage[utf8]{inputenc}
\usepackage{amsmath,amsfonts,amssymb}
\usepackage{dsfont}
\usepackage{graphicx}
\usepackage{caption}
\usefonttheme[onlymath]{serif}


\title{My title}
\date{\today}
\author{Sacha Ben-Arous}
\institute{ENS Paris-Saclay}
\begin{document}
  \maketitle
  

% Dans les slides : reprendre bootstrap MSRI pour convergence en temps long
  
% début : dire que j'étais à NY puis Princeton, que j'ai fait mon stage avec Tristan Buckmaster et ses doctorants, présenter Tristan.
% ensuite faire l'histoire de mécanique des fluides, les motivations philosophiques et physiques (comprendre les failles des modélisations car elles représentent les failles de notre compréhension), écrire Navier-Stokes et Euler avec le problème du millénaire pour faire saisir les enjeux.

% Intro aux singularités : dire qu'on se contente de Burgers, expliquer pourquoi c'est pertinent. Suivre les caractéristiques avec calcul, montrer que dès que pente négative, blowup.
% Faire dessin des caractéristique et vidéo de l'évolution qui atteint une singularité.

% Intro au self-similar : montrer la vidéo de zoom. Dire que l'interêt de self-similaire c'est de transformer une situation singulière en une situation régulière, ce qui n'est pas contradictoire car le changement de variable lui même est singulier.
%TODO : demander à Tristan quelles sont les motivations pour self-similaire, qu'est ce qui a guidé vers ce choix de théorie ?
% Écrire l'equation auto similaire et expliquer comment on choisi alpha et beta, donner les deux critères qui imposent lambda =1/(2i+2) pour i entier.

% Calculer la solution stationnaire.
% Ré-introduire le temps et expliquer le truc le plus important : la stabilité pour les solutions de l'équation auto-similaire est équivalente au blowup pour les solutions de l'équation initiale.
% On s'intéresse donc à la stabilité de l'eq autosimilaire. Écrire l'équation sur la perturbation et spécifier l'op linéaire, justifier pourquoi on s'intéresse qu'à cette partie (pour l'instant).
% On regarde donc le spectre de cet opérateur, on veut que des vp négatives. 1 et 0 sont toujours vp à cause des invariances. On fixe donc sans perte de généralité toutes les invariances.
% Théorème (besoin de citer Masmoudi ? ou c'est facile ?) : pour lambda = 1/2, toutes les valeurs propres de l'opérateur sont négatives.
% Energy inequality + idée de la preuve. On se ramène à un opérateur de rang fini. On se place dans le bon orthogonal pour avoir la stabilité linéaire.

% Faire du bootstrap pour la partie non linéaire ?
  
% A la fin : dire que la grosse équation étudiée est CCF, expliquer le lien avec NS : projection sur divergence free et opérateur intégral (de fourier dans le cas de NS).
% Expliquer que dans le cas de CCF, c'est dur de trouver une solution de l'eq auto similaire, et aussi de trouver des lambdas, que pour le faire on utilise de l'IA, mettre l'image de Tristan sur les PINNs 
 
\begin{frame}
\tableofcontents
\end{frame}  
\section{First Section}

\begin{frame}{First Frame}
    Hello, world!
\end{frame}
\end{document}