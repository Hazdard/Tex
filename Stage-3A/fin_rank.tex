\documentclass[11pt,a4paper]{article}

\usepackage{../jedusor}
\usepackage{mathabx}
\DeclareSymbolFont{yhlargesymbols}{OMX}{yhex}{m}{n} 
\DeclareMathAccent{\yhwidehat}{\mathord}{yhlargesymbols}{"62}
\renewcommand{\headrulewidth}{1pt} 
\renewcommand{\footrulewidth}{1pt}	
\fancyhead[C]{}
\fancyhead[L]{}
\fancyhead[R]{}
\fancyfoot[C]{\thepage} 
\fancyfoot[L]{Sacha Ben-Arous}
\fancyfoot[R]{E.N.S Paris-Saclay}

\begin{document}


\section{Definitions}

Let $W(x) := \left( -\frac{x}{2} + \left( \frac{1}{27} + \frac{x^2}{4} \right)^{1/2} \right)^{1/3} -\  \left( \frac{x}{2} + \left( \frac{1}{27} + \frac{x^2}{4} \right)^{1/2} \right)^{1/3}$ \\

%TODO be the solution of the self-similar Burgers' equation : ????

and define $Lz := -az - b\frac{\partial z}{\partial x} $, where $a(x):= 1+\frac{W(x)}{x}+\frac{\partial W}{\partial x} (x)$ and $b(x):= \frac{3x}{2} + W(x)$.

\section{Computation in Sobolev spaces}

Let $\langle f, g \rangle := \langle f, g \rangle_{L^2} = \displaystyle \int_\R  f(x)g(x)\,dx$ be the usual inner product of $L^2(\R)$.

Let $w,z \in H^k(\R)$ for $k$ large enough. For simplicity, we will denote $\frac{\partial z}{\partial x} := z'$.

\subsection{Symetric part in $L^2$ space}
In $L^2$, the symetric part is computed as follows :
\begin{align*}
  \langle Lz, w \rangle_{L^2}  &=   \langle -az - bz', w \rangle =  \langle z, -aw \rangle  -   \langle z, b'w+bw' \rangle    \\
                  &= \langle z, (-a+b')w + bw' \rangle = \langle z, L^*w \rangle
\end{align*}

Thus, $\frac{1}{2}(L+L^*)z = \frac{1}{2}(-az-bz' -az +b'z+bz') \displaystyle =-az+\frac{b'}{2}z$ in $L^2$.

\subsection{Quadratic form in $H^1$ space}
In $H^1$, the quadratic form is computed as follows :
\begin{align*}
  \langle Lz, z \rangle_{H^1}   &=   \langle -az - bz', z \rangle +  \langle -a'z-az' - b'z'-bz'', z' \rangle \\
  &=  \langle -az, z \rangle + \langle -bz - a'z, z' \rangle + \langle - az'-b'z', z' \rangle + \langle - bz', z'' \rangle \\
  &=  \langle -az, z \rangle + \langle \frac{1}{2}(b'+a'')z, z \rangle + \langle (-a-b')z', z' \rangle + \langle \frac{1}{2} b' z', z' \rangle \\
  &=  \langle (-a+ \frac{b'}{2}+\frac{a''}{2})z, z \rangle +  \langle (-a-\frac{b'}{2})z', z' \rangle 
\end{align*}

\begin{rmq}
The operator $(Lz)'$ is not defined on $H^1$ as it involves second derivatives of $z$, but it is a classical fact that the quadratic form of an operator as a larger domain that the operator itself.
\end{rmq}

\subsection{Quadratic form in $H^2$ space}
In $H^2$, the quadratic form is computed as follows :
\begin{align*}
  \langle (Lz)'', z'' \rangle   &=    \langle -a''z-a'z'-a'z'-az''- b''z'-b'z''-b'z''-bz^{(3)}, z'' \rangle \\
  &= \langle -a''z, z'' \rangle +  \langle (-2a'- b'')z', z'' \rangle +  \langle (-a-2b')z'', z'' \rangle + \langle -bz^{(3)}, z'' \rangle \\
  &= \langle a^{(3)}z+ a''z', z' \rangle +  \langle \frac{1}{2} (2a''+ b^{(3)})z', z' \rangle +  \langle (-a-2b')z'', z'' \rangle + \langle \frac{1}{2} b'z'', z'' \rangle \\
  &= \langle - \frac{1}{2}a^{(4)}z, z \rangle +  \langle 2a''+ \frac{1}{2} b^{(3)})z', z' \rangle +  \langle (-a- \frac{3}{2} b')z'', z'' \rangle 
\end{align*}

Thus, we have in $H^2$ :
\begin{align*}
  \langle Lz, z \rangle_{H^2}   &=  \langle (-a+ \frac{b'}{2}+\frac{a''}{2}-\frac{a^{(4)}}{2})z, z \rangle +  \langle (-a-\frac{b'}{2} +2a''+ \frac{ b^{(3)}}{2})z', z' \rangle +  \langle (-a- \frac{3}{2} b')z'', z'' \rangle 
\end{align*}



\subsection{Quadratic form in $H^3$ space}
In $H^3$, the quadratic form is computed as follows :
\begin{align*}
\langle (Lz)^{(3)} , z^{(3)}  \rangle
&= \langle -a'''z - 3a''z' - 3a'z'' - a z^{(3)} - b'''z' - 3b''z'' - 3b'z^{(3)} - b z^{(4)}, z^{(3)} \rangle \\
&= \langle -a'''z, z^{(3)} \rangle + \langle (-3a'' - b''')z', z^{(3)} \rangle + \langle (-3a' - 3b'')z'', z^{(3)} \rangle + \langle (-a - 3b')z^{(3)}, z^{(3)} \rangle \\
&\quad + \langle -b z^{(4)}, z^{(3)} \rangle \\
&= \langle a^{(4)}z + a'''z', z'' \rangle + \langle (3a''' + b^{(4)})z' + (3a'' + b''')z'', z'' \rangle + \langle \tfrac{3}{2}(a'' + b''')z'', z'' \rangle  \\
&\quad + \langle (-a - 3b')z^{(3)}, z^{(3)} \rangle  + \langle \tfrac{1}{2}b' z^{(3)}, z^{(3)} \rangle \\
&= \langle -a^{(5)}z-a^{(4)}z', z' \rangle + \langle  -\frac{1}{2} a^{(4)}z', z' \rangle  + \langle \frac{1}{2}(-3a^{(4)}  - b^{(5)})z', z' \rangle   + \langle (3a'' + b''')z'', z'' \rangle  \\
&\quad + \langle \tfrac{3}{2}(a'' + b''')z'', z'' \rangle + \langle (-a - 3b')z^{(3)}, z^{(3)} \rangle  + \langle \tfrac{1}{2}b' z^{(3)}, z^{(3)} \rangle \\
&= \langle \frac{a^{(6)}}{2}z , z \rangle +  \langle (-3a^{(4)}  - \frac{1}{2} b^{(5)} )z', z' \rangle  +  \langle (\tfrac{9}{2}a'' +\tfrac{5}{2} b^{(3)})z'', z'' \rangle  + \langle (-a - \frac{5}{2} b')z^{(3)}, z^{(3)} \rangle  
\end{align*}

Thus, we have in $H^3$ :
\begin{align*}
  \langle Lz, z \rangle_{H^3}   &=  \langle (-a+ \frac{b'}{2}+\frac{a''}{2}-\frac{a^{(4)}}{2} + \frac{a^{(6)}}{2})z, z \rangle +  \langle (-a-\frac{b'}{2} +2a''+ \frac{ b^{(3)}}{2}-3a^{(4)} - \frac{1}{2} b^{(5)} ))z', z' \rangle \\
  &\quad +  \langle (-a- \frac{3}{2} b'+\tfrac{9}{2}a'' +\tfrac{5}{2} b^{(3)})z'', z'' \rangle + \langle (-a - \frac{5}{2} b')z^{(3)}, z^{(3)} \rangle 
\end{align*}

\section{Compact part of the quadratic form}
%TODO Montrer que les fonctions en jeu sont intégrables, L^2, dérivable et toutes les props nécessaires à l'étude ci-dessous.
%TODO Faire les vraies preuves sur le signe
We proved in the previous section that the quadratic form associated with $L$ in $H^3$ is of the form :
\[ \langle Lz, z \rangle_{H^3} =  \langle \varphi_0 z, z \rangle +\langle \varphi_1 z', z' \rangle +\langle \varphi_2 z'', z'' \rangle +\langle \varphi_3 z^{(3)}, z^{(3)} \rangle\]

In the next section, we will show that $\varphi_3$ has a sign and is bounded. This leaves to study the lower order terms, and we will prove that there exists a compact operator $M$ such that 
\[\langle Mz, z \rangle_{H^3} =  \langle \varphi_0 z, z \rangle +\langle \varphi_1 z', z' \rangle +\langle \varphi_2 z'', z'' \rangle\]

Combining those results yield the following energy estimate :
\begin{equation}\label{energy}
\langle Lz, z \rangle_{H^3} \leq -\delta \|z\|_{H^3}  + \langle Mz, z \rangle_{H^3} 
\end{equation}
We will use the Fourier transform, with the following convention :
\[
\hat{f}(\xi) := \mathcal{F}(f)(\xi) = \int_{-\infty}^{\infty} f(x) \, e^{-2\pi i x \xi} \, dx
\]
and we will denote \[\mathcal{F}^{-1}(f)(x) := \int_{-\infty}^{\infty} f(\xi) \, e^{2\pi i x \xi} \, d\xi\]
the inverse Fourier transform.

\subsection{Base case}
%TODO check si il manque pas une conjuguaison dans ma def du prod L^2 pour appliquer parseval, et vérifier que les calculs ont du sens car z est dérivé une fois de trop à la fin des IPP.
We want to find $M_0$ such that
\begin{equation}\label{c0}
 \langle M_0z, z \rangle_{H^3} =  \langle \varphi_0 z, z \rangle\
\end{equation} 

The Parseval identity gives : \[\int \hat{z}(\xi) \widehat{M_0z}(\xi)(1+\xi^2)^3 \mathrm{d}\xi= \int \hat{z}(\xi)\widehat{\varphi_0z}(\xi)\mathrm{d}\xi \]

Thus, choosing $M_0$ such that $\widehat{M_0z}(\xi)= \frac{1}{(1+\xi^2)^3}\widehat{\varphi_0z}(\xi)$ would give the equality. \\
Defining $\lambda_0(\xi) := \frac{1}{(1+\xi^2)^3}$, this condition is equivalent to : \[\widehat{M_0z}= \widehat{\mathcal{F}^{-1}(\lambda_0)}\widehat{\varphi_0z}=\yhwidehat{\mathcal{F}^{-1}(\lambda_0)*\varphi_0z}\]
i.e. $M_0z = \mathcal{F}^{-1}(\lambda_0)*\varphi_0z$ satisfies \fcref{c0}.

\subsection{First order case}

We want to find $M_1$ such that
\begin{equation}\label{c1}
 \langle M_1z, z \rangle_{H^3} =  \langle \varphi_1 z', z' \rangle\
\end{equation} 

Integrating by parts and applying the Parseval identity, we have the equivalence 
\begin{eqnarray*}
\langle M_1z, z \rangle_{H^3} &=& -  \langle \varphi_1' z'+\varphi_1 z'', z \rangle = -  \langle \varphi_1'z, z' \rangle - \langle \varphi_1z, z'' \rangle \\
\Leftrightarrow \int \hat{z}(\xi) \widehat{M_1z}(\xi)(1+\xi^2)^3 \mathrm{d}\xi &=& - \int (2\pi i \xi) \hat{z}(\xi)\yhwidehat{\varphi_1'z}(\xi)\mathrm{d}\xi + \int (4\pi^2 \xi^2)\hat{z}(\xi)\yhwidehat{\varphi_1z}(\xi)\mathrm{d}\xi \\
\Leftrightarrow \int \hat{z}(\xi) \widehat{M_1z}(\xi)(1+\xi^2)^3 \mathrm{d}\xi &=&  \int\hat{z} \left[-(2\pi i \xi) \yhwidehat{\varphi_1'z}(\xi) + (4\pi^2 \xi^2)\hat{z}(\xi)\yhwidehat{\varphi_1z}(\xi) \right]\mathrm{d}\xi 
\end{eqnarray*}
Defining $\lambda_1(\xi):= - \frac{2\pi i \xi}{(1+\xi^2)^3 }$ and $\lambda_2(\xi) := \frac{4\pi^2 \xi^2}{(1+\xi^2)^3 }$, we have that 
\[M_1z :=  \left(\mathcal{F}^{-1}(\lambda_1)*\varphi_1'z\right)+\left(\mathcal{F}^{-1}(\lambda_2)*\varphi_1z\right) \]
satisfies \fcref{c1}.


\subsection{Second order case}

We want to find $M_2$ such that
\begin{equation}\label{c2}
 \langle M_2z, z \rangle_{H^3} =  \langle \varphi_2 z'', z'' \rangle\
\end{equation} 

Integrating by parts twice and applying the Parseval identity, we have the equivalence 
\begin{eqnarray*}
\langle M_2z, z \rangle_{H^3} &=&  \langle \varphi_2''z'' +2\varphi_2'z^{(3)}+\varphi_2z^{(4)}, z \rangle = \langle \varphi_2''z, z'' \rangle + \langle 2 \varphi_2'z, z^{(3)} \rangle + \langle \varphi_2z, z^{(4)} \rangle \\
\Leftrightarrow \int \hat{z}(\xi) \widehat{M_2z}(\xi)(1+\xi^2)^3 \mathrm{d}\xi &=& -  \int (4\pi^2 \xi^2)\hat{z}(\xi)\yhwidehat{\varphi_2''z}(\xi)\mathrm{d}\xi - \int (i16\pi^3\xi^3)\hat{z}(\xi)\yhwidehat{\varphi_2'z}(\xi)\mathrm{d}\xi + \int (16\pi^4\xi^4)\hat{z}(\xi)\yhwidehat{\varphi_2z}(\xi)\mathrm{d}\xi \\
\Leftrightarrow \int \hat{z}(\xi) \widehat{M_2z}(\xi)(1+\xi^2)^3 \mathrm{d}\xi &=&  \int\hat{z} \left[-(4\pi^2 \xi^2) \yhwidehat{\varphi_2''z}(\xi) - (i16\pi^3\xi^3)\hat{z}(\xi)\yhwidehat{\varphi_2'z}(\xi) + (16\pi^4\xi^4)\yhwidehat{\varphi_2z}(\xi)\right]\mathrm{d}\xi 
\end{eqnarray*}
Defining $\lambda_3(\xi) := - \frac{i16\pi^3\xi^3}{(1+\xi^2)^3 }$ and $\lambda_4(\xi) :=  \frac{16\pi^4\xi^4}{(1+\xi^2)^3 }$ , we have that 
\[M_2z :=  \left(-\mathcal{F}^{-1}(\lambda_2)*\varphi_2''z\right)+\left(\mathcal{F}^{-1}(\lambda_3)*\varphi_2'z\right)+\left(\mathcal{F}^{-1}(\lambda_4)*\varphi_2z\right) \]
satisfies \fcref{c2}.

\newpage

\section{Quality of the approximation}
The compact operators that we are studying are of the form: \[K(f)(x) = \int_\R K(x,y)f(y)\mathrm{d}y \]
Assuming that we can bound the growth of the kernel at infinity, a natural choice of a finite rank approximation on a bounded domain $[-A,A]$ would be to use a Riemann sum : \[K_n(f)(x) := \sum_{i=0}^n \delta K(x,y_i)f(y_i)\]
where $\delta := \frac{2A}{n}$ is the integration step and $\begin{cases} y_0=-A \\ y_{i+1}=y_i + \delta \end{cases}$ are the sample points. \\

Now, we want to get a precise bound on the quality of this approximation in order to compute a relevant upper bound in the energy estimate \ref{energy}. \\
Assuming that $f\in H^3(\R)$ is also in $L^1(\R)$, we can use the inversion formula and work in Fourier space :
\begin{equation}\label{int_finite}
K_n(f)(x) =  \sum_{i=1}^n\delta K(x,y_i)f(y_i) = \int_\R \hat{f}(\xi)\sum_{i=0}^n \delta K(x,y_i) e^{2i\pi\xi y_i}\mathrm{d}\xi
\end{equation}
On the other hand, the operator can be written as :
\begin{eqnarray*}
K(f)(x) &=& \int_\R K(x,y)f(y)\mathrm{d}y = \int_\R \hat{K}(x,\xi)\hat{f}(\xi)\mathrm{d}\xi \\
&=& \int_\R \hat{f}(\xi) \int_\R K(x,y)e^{-2i\pi\xi y}\mathrm{d}y \mathrm{d}\xi 
\end{eqnarray*}

There is a sign problem in the phase of the exponential, but if we can assume that the function $f$ is even, then we can also reverse the sign in the approximate kernel formula, and now the difference between the kernel and its approximation is easily bounded using the mean value theorem on $[-A,A]$, and by giving an explicit bound on the decay of the kernel away from zero. 

Now, taking the difference of the two expressions, we have :

\begin{eqnarray*}
\left| K(f)(x) - K_n(f)(x)\right| &\leq& \left|  \int_\R \hat{f}(\xi) \left(\int_{-A}^A K(x,y)e^{-2i\pi\xi y}\mathrm{d}y - \sum_{i=0}^n \delta K(x,y_i) e^{2i\pi\xi y_i} \right)\mathrm{d}\xi \right| \\ + \left|\int_\R \hat{f}(\xi) \int_{\R \setminus [-A,A]} K(x,y)e^{2i\pi\xi y}\mathrm{d}y\mathrm{d}\xi \right| \\
&\leq& (I) + (II)
\end{eqnarray*}

Let's first bound the difference on the compact domain $[-A,A]$ by applying the mean value theorem:

\begin{eqnarray*}
(I) &\leq&  \left|  \int_\R \hat{f}(\xi)  \sum_{i=0}^n \left(\int_{y_i}^{y_{i+1}} K(x,y)e^{2i\pi\xi y} - K(x,y_i) e^{2i\pi\xi y_i}\mathrm{d}y \right)\mathrm{d}\xi \right| \\
&\leq&   \int_\R \left| \hat{f}(\xi) \right|  \sum_{i=0}^n \left(\int_{y_i}^{y_{i+1}}  \delta \sup_{y\in[-A,A]} {\left|\partial_yK(x,y)e^{2i\pi\xi y} +2i\pi\xi K(x,y)e^{2i\pi\xi y} \right|} \mathrm{d}y \right)\mathrm{d}\xi \\
&\leq&  n  \delta^2  \int_\R \left| \hat{f}(\xi) \right|  \sup_{y\in[-A,A]}\Big( {\left|\partial_yK(x,y)\right|+ \left|2\pi\xi K(x,y)\right|}\Big) \mathrm{d}\xi \\
&\leq&  n  \delta^2 \Big(\| \hat{f}\|_{L^1}  \sup_{y\in[-A,A]}{\left|\partial_yK(x,y)\right|} +  \| \xi\hat{f}(\xi)\|_{L^1} \sup_{y\in[-A,A]}{2\pi|K(x,y)|} \Big) \\
&\leq&  C\frac{4A^2}{n}  \Big(\sup_{y\in[-A,A]}\left|\partial_yK(x,y)\right| + \sup_{y\in[-A,A]}\left|2\pi K(x,y)\right| \Big) \| f\|_{H^2} \\
\end{eqnarray*}

Where $C := \max\Big(\displaystyle \int_\R \frac{1}{(1+\xi^2)} \mathrm{d}\xi,\int_\R \frac{\xi^2}{(1+\xi^2)^2} \mathrm{d}\xi\Big)$ and we used the fact that, because $f\in H^3(\R)$, we have: \[\| \hat{f}(\xi)\|_{L^1} = \| \frac{1}{(1+\xi^2)^{1/2}}(1+\xi^2)^{1/2}\hat{f}(\xi)\|_{L^1} \leq \| \frac{1}{(1+\xi^2)^{1/2}}\|_{L^2} \|(1+\xi^2)^{1/2}\hat{f}(\xi)\|_{L^2} = \| \frac{1}{(1+\xi^2)^{1/2}}\|_{L^2} \| f\|_{H^1}  \] 
and  \[\| \xi\hat{f}(\xi)\|_{L^1} = \| \frac{\xi}{(1+\xi^2)}(1+\xi^2)\hat{f}(\xi)\|_{L^1} \leq \| \frac{\xi}{(1+\xi^2)}\|_{L^2} \|(1+\xi^2)\hat{f}(\xi)\|_{L^2} = \| \frac{\xi}{(1+\xi^2)}\|_{L^2} \| f\|_{H^2}  \] 

~\\
Then for the second part, by Fubini's theorem:
\begin{eqnarray*}
(II) &=&  \left|\int_{\R \setminus [-A,A]} K(x,y) \int_\R \hat{f}(\xi) e^{2i\pi\xi y}\mathrm{d}\xi \mathrm{d}y\right|  =\left| \int_{\R \setminus [-A,A]} K(x,y) f(y)\mathrm{d}y \right|  \\
&=& \left| \int_{\R \setminus [-A,A]} \lambda(x-y)\varphi(y) f(y)\mathrm{d}y \right| 
\end{eqnarray*}

Since we can show that $\lambda(x)=\frac{\pi}{4}e^{-2\pi \lvert x\rvert}\left(2\pi^2x^2+3\pi \lvert x\rvert +\frac{3}{2}\right)$, we have the following bound :
\begin{lemma}
Ayaya
\end{lemma}

Assuming we can show that $\varphi$ is bounded, we have :
% We can also integrate by part here, since f is in H^3
\begin{eqnarray*}
(II) &\leq& \|\varphi\|_{L^\infty} \|f\|_{L^2} \int_{\R \setminus [x-A,x+A]} \lambda(u)^2\mathrm{d}u
\end{eqnarray*}






















\end{document}
