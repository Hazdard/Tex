\documentclass[12pt,a4paper]{article}

\usepackage{../jedusor}
	
\renewcommand{\headrulewidth}{1pt} 
\renewcommand{\footrulewidth}{1pt}	
\fancyhead[C]{}
\fancyhead[L]{}
\fancyhead[R]{}
\fancyfoot[C]{\thepage} 
\fancyfoot[L]{Sacha Ben-Arous}
\fancyfoot[R]{E.N.S Paris-Saclay}

\begin{document}

\section{Definitions}

Let $W(x) := \left( -\frac{x}{2} + \left( \frac{1}{27} + \frac{x^2}{4} \right)^{1/2} \right)^{1/3} -\  \left( \frac{x}{2} + \left( \frac{1}{27} + \frac{x^2}{4} \right)^{1/2} \right)^{1/3}$ \\

be the solution of the self-similar Burgers' equation : \\

and define $Lz := -az - b\frac{\partial z}{\partial x} $, where $a(x):= 1+\frac{W(x)}{x}+\frac{\partial W}{\partial x} $ and $b(x):= \frac{3x}{2} + W(x)$.

\section{Computation in Sobolev spaces}

Let $\langle f, g \rangle := \langle f, g \rangle_{L^2} = \displaystyle \int_\R  f(x)g(x)\,dx$ be the usual inner product of $L^2(\R)$.

Let $w,z \in H^k(\R)$ for $k$ large enough. For simplicity, we will denote $\frac{\partial z}{\partial x} := z'$.

\subsection{Symetric part in $L^2$ space}
In $L^2$, the symetric part is computed as follows :
\begin{align*}
  \langle Lz, w \rangle_{L^2}  &=   \langle -az - bz', w \rangle =  \langle z, -aw \rangle  -   \langle z, b'w+bw' \rangle    \\
                  &= \langle z, (-a+b')w + bw' \rangle = \langle z, L^*w \rangle
\end{align*}

Thus, $\frac{1}{2}(L+L^*)z = \frac{1}{2}(-az-bz' -az +b'z+bz') \displaystyle =-az+\frac{b'}{2}z$ in $L^2$.

\subsection{Quadratic form in $H^1$ space}
In $H^1$, the quadratic form is computed as follows :
\begin{align*}
  \langle Lz, z \rangle_{H^1}   &=   \langle -az - bz', z \rangle +  \langle -a'z-az' - b'z'-bz'', z' \rangle \\
  &=  \langle -az, z \rangle + \langle -bz - a'z, z' \rangle + \langle - az'-b'z', z' \rangle + \langle - bz', z'' \rangle \\
  &=  \langle -az, z \rangle + \langle \frac{1}{2}(b'+a'')z, z \rangle + \langle (-a-b')z', z' \rangle + \langle \frac{1}{2} b' z', z' \rangle \\
  &=  \langle (-a+ \frac{b'}{2}+\frac{a''}{2})z, z \rangle +  \langle (-a-\frac{b'}{2})z', z' \rangle 
\end{align*}

\begin{rmq}
The operator $(Lz)'$ is not defined on $H^1$ as it involves second derivatives of $z$, but it is a classical fact that the quadratic form of an operator as a larger domain that the operator itself.
\end{rmq}

\subsection{Quadratic form in $H^2$ space}
In $H^2$, the quadratic form is computed as follows :
\begin{align*}
  \langle (Lz)'', z'' \rangle   &=    \langle -a''z-a'z'-a'z'-az''- b''z'-b'z''-b'z''-bz^{(3)}, z'' \rangle \\
  &= \langle -a''z, z'' \rangle +  \langle (-2a'- b'')z', z'' \rangle +  \langle (-a-2b')z'', z'' \rangle + \langle -bz^{(3)}, z'' \rangle \\
  &= \langle a^{(3)}z+ a''z', z' \rangle +  \langle \frac{1}{2} (2a''+ b^{(3)})z', z' \rangle +  \langle (-a-2b')z'', z'' \rangle + \langle \frac{1}{2} b'z'', z'' \rangle \\
  &= \langle - \frac{1}{2}a^{(4)}z, z \rangle +  \langle 2a''+ \frac{1}{2} b^{(3)})z', z' \rangle +  \langle (-a- \frac{3}{2} b')z'', z'' \rangle 
\end{align*}

Thus, we have in $H^2$ :
\begin{align*}
  \langle Lz, z \rangle_{H^2}   &=  \langle (-a+ \frac{b'}{2}+\frac{a''}{2}-\frac{a^{(4)}}{2})z, z \rangle +  \langle (-a-\frac{b'}{2} +2a''+ \frac{ b^{(3)}}{2})z', z' \rangle +  \langle (-a- \frac{3}{2} b')z'', z'' \rangle 
\end{align*}



\subsection{Quadratic form in $H^3$ space}
In $H^3$, the quadratic form is computed as follows :
\begin{align*}
  \langle (Lz)^{(3)}, z^{(3)} \rangle   &=    
\end{align*}

Thus, we have in $H^3$ :
\begin{align*}
  \langle Lz, z \rangle_{H^2}   &=  \langle -a^{(3)}z - 3a''z' - 3a'z'' - a z^{(3)} - b^{(3)}z' - 3b''z'' - 3b'z^{(3)} - b z^{(4)}, z^{(3)} \rangle  \\
   &=  \langle -a^{(3)}z, z^{(3)} \rangle + \langle (- 3a'' - b^{(3)})z', z^{(3)} \rangle +  \langle (- 3a' - 3b'')z'', z^{(3)} \rangle + \langle  (- a - 3b')z^{(3)}, z^{(3)} \rangle \\ + \langle - b z^{(4)}, z^{(3)} \rangle \\
   &=  \langle a^{(4)}z+a^{(3)}z', z'' \rangle + \langle (3a^{(3)} - b^{(4)})z'+( 3a''+ b^{(3)})z'', z'' \rangle +  \langle (- 3a' - 3b'')z'', z^{(3)} \rangle \\ + \langle  (- a - 3b')z^{(3)}, z^{(3)} \rangle \\ + \langle - b z^{(4)}, z^{(3)} \rangle \\
\end{align*}


\end{document}
