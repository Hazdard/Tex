\documentclass[11pt,a4paper]{article}
\textheight245mm
\textwidth170mm
\hoffset-21mm
\voffset-15mm
\parindent0pt
\usepackage[utf8]{inputenc}
\usepackage{dsfont}
\usepackage{graphicx}
\usepackage{caption}
\usepackage{fancyhdr}
\usepackage{amsmath,amsfonts,amssymb}
\usepackage[french]{babel}
\usepackage[hidelinks]{hyperref} 
\hypersetup{
  colorlinks   = true,    % Colours links instead of ugly boxes
  urlcolor     = blue,    % Colour for external hyperlinks
  linkcolor    = black,    % Colour of internal links
  citecolor    = black      % Colour of citations
}

\pagestyle{fancy}

\usepackage{array,multirow,makecell}
\setcellgapes{4pt}
\makegapedcells
\newcolumntype{R}[1]{>{\raggedleft\arraybackslash }b{#1}}
\newcolumntype{L}[1]{>{\raggedright\arraybackslash }b{#1}}
\newcolumntype{C}[1]{>{\centering\arraybackslash }b{#1}}

\renewcommand{\headrulewidth}{1pt}
\fancyhead[C]{}
\fancyhead[L]{L3 - 2022/2023}
\fancyhead[R]{D.E.R Informatique}

\renewcommand{\footrulewidth}{1pt}
\fancyfoot[C]{\thepage} 
\fancyfoot[L]{Sacha Ben-Arous}
\fancyfoot[R]{E.N.S Paris-Saclay}

\begin{document}

Si $(M_i)_{i \leq t}$ est une famille de matrices, on désigne par $[M_{i\leq t}]$ la matrice par blocs qui correspond à l'empilement des  $(M_i)_{i \leq t}$ : 
$$ \left[\begin{array}{c}
M_1 \\ M_2 \\ \dots \\ M_t
\end{array}\right] $$
 \\

On considère les $(a_i)_{i \leq t}$ variables aléatoires uniformes sur $(\mathbb{Z}/q\mathbb{Z})^n[X]$ (i.e chaque coefficient est choisi uniformément). \\

On défini $A:=[\text{Toep}^{d}(a_i)_{i\leq t}]$, où  $\text{Toep}^{d}(P)$ est une matrice à $d$ lignes et $\text{deg}(P)+d$ colonnes, dont la $j$-ème ligne est constituée des coefficients de $x^{j-1}P$. Par exemple, \\

$\text{Toep}^{3}(X^2 + 3X + 1) = \left[\begin{array}{ccccc}
1&3&2&0&0 \\
0&1&3&2&0 \\ 
0&0&1&3&2 
\end{array}\right]$ \\

Pour donner du contexte, en pratique on a $t=O(\log{n})$, $d=n/2$ et $q \geq n^{2.5}\log{n}$ \\
Ainsi les Toeplitz sont des matrices environ 3 fois plus larges que longues, et $A$ est une matrice très longue. Le rang maximum de cette dernière est donc $n+d$ \\


\textbf{Théorème : \\}
Avec une probabilité $\geq 1 - (\frac{n+d}{q})^{\lfloor t/\lceil\frac{n+d}{d}\rceil\rfloor}$, on a que $A$ est de rang plein. \\
Si on utilise les ordres de grandeurs proposés dans le schéma de chiffrement, on a:  \[\mathbb{P}(\text{rg}(A)=n+d) \geq 1 - (\frac{3}{2 n^{1.5}\log{n}})^{\frac{\log{n}}{3}}\] \\


\textbf{Preuve :}
On commence par rappeller de lemme de Schwartz-Zippel : pour un polynôme multivarié non nul de degré $n$, à coefficients dans $\mathbb{Z}/p\mathbb{Z}$, la probabilité d'annuler ce polynôme en choisissant les variables uniformément est au plus $\frac{n}{p}$. \\
On considère donc la matrice carré constituée des $n+d$ premières lignes de $A$, que l'on note $A_1$. Le déterminant de cette sous-matrice est un polynôme à plusieurs variables, de degré $n+d$, dont les variables sont choisies uniforméments dans $\mathbb{Z}/q\mathbb{Z}$. Ce polynôme est non nul, par exemple on peut choisir $a_1=1$, $a_2= x^d$, $a_3=x^{2d}$, $\dots$ et alors $A_1 = \text{Id}$ et donc $\det(A_1)=1\neq 0$.
\\ Alors, d'après le lemme , on a que : \[\mathbb{P}[\text{det}(A_1)=0] \leq \frac{n+d}{q}\]
On peut ensuite répéter ce processus pour les sous-matrices suivantes. Cependant, afin de conserver l'indépendance, il faut faire attention à ne pas reprendre une Toeplitz déjà utilisée. Ainsi, $A_k$ sera la sous matrice carré commençant à la $ (k-1)d\lceil \frac{n+d}{d} \rceil $-ème ligne. Au total, on pourra donc avoir au moins $\lfloor t/\lceil\frac{n+d}{d}\rceil\rfloor$ sous matrices carrés dont les entrées sont mutuellements indépendantes. Alors : 
\begin{eqnarray*}
\mathbb{P}(\text{rg}(A)<n+d) &\leq& \mathbb{P}(\bigcap_{k\leq\lfloor t/\lceil\frac{n+d}{d}\rceil\rfloor} \det{A_k} = 0) \\
&=&\mathbb{P}[\text{det}(A_1)=0]^{\lfloor t/\lceil\frac{n+d}{d}\rceil\rfloor} \\
&=&  (\frac{n+d}{q})^{\lfloor t/\lceil\frac{n+d}{d}\rceil\rfloor}
\end{eqnarray*}
Ce qui donne bien l'inégalité voulue.

\end{document}
