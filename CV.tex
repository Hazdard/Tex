\documentclass[11pt,a4paper]{article}
\textheight245mm
\textwidth170mm
\hoffset-21mm
\voffset-15mm
\parindent0pt
\usepackage[utf8]{inputenc}
\usepackage{amsmath,amsfonts,amssymb}
\usepackage{dsfont}
\usepackage{graphicx}
\usepackage{caption}
\usepackage{fancyhdr}
\pagestyle{fancy}

\renewcommand{\headrulewidth}{1pt}
\fancyhead[C]{}
\fancyhead[L]{}
\fancyhead[R]{}

\renewcommand{\footrulewidth}{1pt}
\fancyfoot[C]{\thepage} 
\fancyfoot[L]{}
\fancyfoot[R]{E.N.S Paris-Saclay}

\begin{document}

\title{Curriculum vitæ}
\date{}
\author{Sacha BEN AROUS}
\maketitle 

\section{Formation}

\begin{itemize}
\item 2020 - 2022 : Lycée Janson-de-Sailly, Classes préparatoires aux grandes écoles (CPGE), MPSI-MP*
\item 2022 - 2026 : École normale supérieure Paris-Saclay, Département Informatique
\end{itemize}

\section{Compétences}
\begin{itemize}
\item Languages de programmation maitrisés : Python - C - OCaml - x86 Assembly - Coq - Go - Java 
\item Projets réalisés : \begin{itemize}
\item[-] Compilateur d'expressions arithmétiques (en OCaml et Assembly x86\_64)
\item[-] Compilateur d'une version réduite de Go (en OCaml et Assembly x86\_64)
\item[-] Simulation d'un ordinateur primitif (CPU/Mémoire morte/Mémoire vive) à partir de portes logiques (en pseudo-code)
\item[-] Réalisation d'un shell basique (en C)
\end{itemize}
\item Pratique du Pentesting : Cryptographie, Réseau, Exploitation Web \ (classé top 5\% sur TryHackMe.com)
\end{itemize}

\section{Contact}

email : sacha.ben-arous@ens-paris-saclay.fr \\
Tel. : 0644391099

\end{document}
