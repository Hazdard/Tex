\documentclass[11pt,a4paper]{article}
\textheight245mm
\textwidth170mm
\hoffset-21mm
\voffset-15mm
\parindent0pt
\usepackage[utf8]{inputenc}
\usepackage{amsmath,amsfonts,amssymb}
\usepackage{dsfont}
\usepackage{graphicx}
\usepackage{caption}
\usepackage{fancyhdr}
\pagestyle{fancy}

\renewcommand{\headrulewidth}{1pt}
\fancyhead[C]{Langages formels}
\fancyhead[L]{L3 - 2022/2023}
\fancyhead[R]{D.E.R Informatique}

\renewcommand{\footrulewidth}{1pt}
\fancyfoot[C]{\thepage} 
\fancyfoot[L]{Sacha Ben-Arous}
\fancyfoot[R]{E.N.S Paris-Saclay}
\newcommand{\squo}[2]{{\raisebox{.2em}{$#1$}\left/\raisebox{-.2em}{$#2$}\right.}}
\begin{document}

\section{Introduction}
\ \ \ \ \ - Un \textbf{automate fini déterministe} est un 4-uplet $\mathcal{A} = (Q,\delta, i, F)$ où $Q$ est un ensemble \textit{fini} d'états, $\delta : Q \times \Sigma \to Q \ $ est la fonction de transition, $i$ l'état initial, et $F \subseteq Q$ l'ensemble des états finaux. \\

\ \ \ \ \ - Le \textbf{language reconnu} par $\mathcal{A}$ est $\mathcal{L}(\mathcal{A}) := \{u \in \Sigma^* \ | \ \delta(i,u) \in F\}$. Un langage $L \subseteq \Sigma^*$ est dit \textbf{reconnaissable} si il existe un automate fini $\mathcal{A}$ tel que $L = \mathcal{L}(\mathcal{A})$\\

\ \ \ \ \ - Un \textbf{automate fini non-déterministe} est un 4-uplet $\mathcal{A} = (Q,T, i, F)$ où $Q$ est un ensemble \textit{fini} d'états, $ T \subseteq Q \times \Sigma	\times Q$ est la table de transition, $i$ l'état initial, et $F \subseteq Q$ l'ensemble des états finaux. \\

\ \ \ \ \ - On peut \textbf{déterminiser} tout automate fini. \\

\underline{Rq} : La preuve se fait à l'aide de l'automate des parties. \\
\underline{Rq} : On peut tester en $O(|\omega|\cdot|Q|^2)$ si $\omega \in \mathcal{L}(\mathcal{A})$\\

\ \ \ \ \ - On peut \textbf{émonder} tout automate fini. \\

\ \ \ \ \ - Les automates avec des $\epsilon$-transitions sont équivalents aux automates classiques.  \\

\underline{Rq} : On note Rec$(\Sigma^*)$ la famille des langages reconnaissable sur $\Sigma^*$. \\

\ \ \ \ \ - Rec$(\Sigma^*)$ est \textbf{fermée} par union, intersection, complément et quotient. \\

\underline{Rappel} : $K^{-1}L := \{\omega \in \Sigma^*\ | \ \exists (k,l) \in K\times L, \ \omega =kl \}$ est appelé quotient (à gauche) de $L \in Rec(\Sigma^*)$ par $K \subseteq \Sigma^*$.  \\
\underline{Rq} : Rec($\Sigma^*$) est de même fermée par préfixe, suffixe et facteur. \\

\ \ \ \ \ - Soit $f : A^* \to B^*$ un morphisme, $L_1 \in$ Rec($A^*$),  $L_2 \in$ Rec($B^*$), on a : $\begin{cases} f(L_1) \in \text{Rec}(B^*) \\ f^{-1}(L_2) \in \text{Rec}(A^*) \end{cases}$. \\

\ \ \ \ \ - Une \textbf{substitution} est une application $\sigma : A \to \mathcal{P}(B^*)$. Elle s'étend naturellement en morphisme de $A^*$ vers $\mathcal{P}(B^*)$. \\ On adopte les notations suivantes : $\begin{cases} \text{Pour } L \subseteq A^*, \ \sigma(L) := \bigcup\limits_{u \in L} \sigma(u) \\ \text{Pour } L \subseteq B^*, \ \sigma^{-1}(L) := \{u \in A^* \ | \ \sigma(u)\cap L \neq \emptyset \} \end{cases}$ \\

Une substitution est \textbf{rationnelle} si elle est définie par une application $\sigma : A \to \text{Rec}(B^*)$. \\

\ \ \ \ \ - La famille des langages reconnaissables est fermée par substitution rationnelle et substitution rationnelle inverse, i.e si $\sigma : A \to \text{Rec}(B^*)$, on a :$\begin{cases} \forall L \in \text{Rec}(A^*), \sigma(L) \in \text{Rec}(B^*) \\ \forall L \in \text{Rec}(B^*), \sigma^{-1}(L) \in \text{Rec}(A^*) \end{cases}$. \\

\underline{Résumé :} Les langages reconnaissables sont clos par union, intersection, complément, quotient, préfixe, suffixe, facteur, morphisme et substitution (inverses). \\ \\ \\ \\ \\

\section{Langages rationnels}
\ \ \ \ \ - L'ensemble des expressions \textbf{rationnelles} $\mathcal{E}$ est le plus petit ensemble qui contient $\Sigma$, et qui est stable par $+, \cdot$ et  $*$. Un langage $L$ est dit \textbf{rationnel} si il existe une expression rationnelle $e$ telle que $L = \mathcal{L}(e)$. On définit : $ \begin{cases} \mathcal{L}(e+f) :=\mathcal{L}(e)\cup \mathcal{L}(f) \\ \mathcal{L}(e) \cdot \mathcal{L}(f):= \mathcal{L}(e)\cdot \mathcal{L}(f) \\ \mathcal{L}(e^*):=\mathcal{L}(e)^* \end{cases}$ \\

\underline{Rq} : Deux e.r sont dites \textbf{équivalentes} si leurs langages sont égaux. \\

\textbf{Théorème (Kleene) :} Rec$(\Sigma^*) = $ Rat$(\Sigma^*)$ \\

\underline{Rq} : Une démonstration du sens réciproque peut se faire avec l'algo de McNaughton-Yamada, décrit ci-dessous. \\

\textbf{Algorithme (McNaughton-Yamada) :}

\ \ \ \ \ \ Soit $\mathcal{A} = \lbrace Q,T,I,F \rbrace $ un automate ND. On va construire $e\in \mathcal{E}$ telle que $\mathcal{L}(e)=\mathcal{L}(\mathcal{A})$. \\
On dénote les états de $\mathcal{A}$  par $\{1,\dots,n\}$. On note $L_{p,q}$ (resp. $L_{p,q}^{(k)}$) le langage accepté par $\mathcal{A}$ avec $p$ initial et $q$ final (resp. et ne passant que par les états $1,\dots, k$ entre $p$ et $q$). \\
On va exprimer $L_{p,q}^{(k)}$ par une expression rationnelle $e_{p,q}^{(k)}$, calculée récursivement :
\begin{itemize}
\item $e_{p,q}^{(0)} = \Sigma\{a \ | \ p \xrightarrow{a} q \}$
\item Pour $ 1 \leq k \leq n$, $e_{p,q}^{(k)}$ = $e_{p,q}^{(k-1)} + e_{p,q}^{(k-1)}(e_{k,k}^{(k-1)})^*e_{p,q}^{(k-1)}$ 
\end{itemize}
Alors, $L_{p,q}^{(k)} = \begin{cases} \mathcal{L}(e_{p,q}^{(n)}) \ \ \ \text{si} \ p \neq q \\ \mathcal{L}(e_{p,q}^{(n)} + \emptyset^*) \ \ \ \text{si} \ p = q \end{cases}$ et donc $\mathcal{L}(\mathcal{A})=\bigcup\limits_{i \in I}\bigcup\limits_{f \in F} L_{i,f}$ . \\ \\


\ \ \ \ \ \ - \textbf{Lemme (Étoile) :} Si $L \in$ Rec$(\Sigma^*)$, alors il existe $N \geq 0$ tel que pour tout $x \in L$, on ait :
\begin{itemize}
\item Si $|x| \geq N$, alors $\exists u_1, u_2, u_3 \in \Sigma^*$ tels que $x=u_1u_2u_3$, $u_2 \neq \epsilon$ et $u_1u_2^*u_3 \subseteq L$
\item Si $x=w_1w_2w_3$ avec $|w_2| \geq N$ alors $\exists u_1, u_2, u_3 \in \Sigma^*$ tels que $w_2=u_1u_2u_3$,$u_2 \neq \epsilon$ et $w_1u_1u_2^*u_3w_3\subseteq L$
\item Si $x=u_1v_1\dots v_Nw$ avec $|v_i| \geq 1$ alors il existe $0 \leq j < k \leq N$ tels que \\ $u_1v_1\dots v_j(v_{j+1}\dots v_k)^*v_{k+1}\dots v_Nw \subseteq L$ \\
\end{itemize}


\section{Résiduels}

\ \ \ \ \ \ - Soient $u \in \Sigma^*$, et $L \subseteq \Sigma^*$. Le \textbf{résiduel} de $L$ par $u$ est le quotient $u^{-1}L=\{v \in \Sigma^* \ | \ uv \in L \}$. \\

\ \ \ \ \ \ - Soit $L \subseteq \Sigma^*$. L'\textbf{automate des résiduels} de $L$ est $\mathcal{R}(L) = (Q_L, \delta_L, i_L, F_L)$ avec : \\
 $\begin{cases} Q_L = \{u^{-1}L \ | \ u \in \Sigma^* \} \\ \delta_L(u^{-1}L,a)=a^{-1}(u^{-1}L)=(ua)^{-1}L \\ i_L = L = \epsilon^{-1}L \\ F_L = \{u^{-1}L \ | \ \epsilon \in u^{-1}L \} = \{u^{-1}L \ | \ u \in L \} \end{cases}$ \\ \\

\textbf{Théorème :} Un langage est reconnaissable ssi il a un nombre fini de résiduels. \\ 

\ \ \ \ \ \ - Soit $\mathcal{A}$ un automate DC. Une relation d'équivalence $\sim$ sur $Q$ est une congruence si :
\begin{itemize}
\item $\forall p,q \in Q, \ \forall a \in \Sigma, \ p \sim q \Rightarrow \delta(p,a) \sim \delta(q,a)$
\item $F$ est saturé par $\sim$, i.e : $\forall p \in F, \ [p]=\{q \in Q \ | \ p \sim q \} \subseteq F$
\end{itemize}
Le \textbf{quotient} de $\mathcal{A}$ par $\sim$ est $A/\mathord\sim = (Q/\mathord\sim,\delta_{\sim},[i],F/\mathord\sim)$, où $\delta_{\sim}$ est définie par $\delta_{\sim}([p],a)=[\delta	(p,a)]$ \\

\textbf{Théorème :} $\mathcal{L}(\mathcal{A}/\mathord\sim) = \mathcal{L}(\mathcal{A})$ \\

\ \ \ \ \ \ - Soit $\mathcal{A}=(Q,\delta,i,F)$ un automate DCA (DC et accessible) reconnaissant $L$. Pour $q \in Q$, on note $\mathcal{L}(\mathcal{A},q)=\{u \in \Sigma^* \ | \ \delta(q,u) \in F \}$. L'\textbf{équivalence de Nerode} de $\mathcal{A}$ est définie par $p \sim q$ si $\mathcal{L}(\mathcal{A},p)=\mathcal{L}(\mathcal{A},q)$.
\begin{itemize}
\item L'équivalence de Nerode est une congruence. 
\item L'automate $\mathcal{A}/\mathord\sim$ est appelé \textbf{quotient de Nerode} de $\mathcal{A}$. \\
\end{itemize}

\textbf{Théorème :}  $\mathcal{A}/\mathord\sim = \mathcal{R}(L)$, i.e : Le quotient de Nerode est isomorphe à l'automate des résiduels. \\
$\varphi : Q/\mathord\sim \to Q_L$ définie par $\varphi([q]) = \mathcal{L}(\mathcal{A},q)$ est un isomorphisme de $\mathcal{A}/\mathord\sim$ sur $\mathcal{R}(L)$. \\

\textbf{Théorème :} Soit $L \in $ Rec$(\Sigma^*)$.
\begin{itemize}
\item Si $\mathcal{A}$ est un automate DCA qui reconnaît $L$, alors $\mathcal{R}(L)$ est un quotient de $\mathcal{A}$.
\item $\mathcal{R}(L)$ est minimal parmis les automates DCA reconnaissant L (en nombre d'états).
\item Si $\mathcal{A}$ est un automate DC reconnaissant $L$ avec un nombre minimal d'états, alors $\mathcal{A}$ est isomorphe à $\mathcal{R}(L)$ (i.e l'automate minimal est unique). \\
\end{itemize}

\textbf{Algorithme de Moore :} Pour $n \geq 0$, on définie $\sim_n$ sur $Q$ par:  $p \sim_n q $ si $\mathcal{L}(\mathcal{A},p)\cap \Sigma^{\leq n} = \mathcal{L}(\mathcal{A},q)\cap \Sigma^{\leq n} $ \\
Alors, en notant $\sim\  = \bigcap\limits_{n\geq 0} \sim_n$ , on a les propriétés suivantes :
\begin{itemize}
\item $p \sim_{n+1} q  \ \Leftrightarrow \ p \sim_n q $ et $  \forall a \in \Sigma, \ \delta(p,a) \sim_n \delta(q,a)$
\item Si $\sim_n \ = \ \sim_{n+1}$ alors $\sim \ = \ \sim_n$
\item $\sim \ = \ \sim_{|Q|-2}$ \\
\end{itemize}

\underline{Rq} : $\sim_0$ a pour classes d'équivalences $F$ et $Q \setminus F$. 

\section{Monoïdes et congruences}

\ \ \ \ \ \ - Soit $M$ un monoïde fini, $\varphi : \Sigma^* \to M$ un morphisme, et $L \subseteq \Sigma^*$. On donne les définitions suivantes : 
\begin{itemize}
\item $L$ est \textbf{reconnu} (ou saturé) par $\varphi$ si $L = \varphi^{-1}(\varphi(L))$ ($\subseteq$ est toujours vraie)
\item Un langage est reconnu par un monoïde fini si il existe un morphisme de ce monoïde qui reconnait ce langage.
\item Un langage est \textbf{reconnaissable par morphisme} si il existe un monoïde fini qui reconnait ce langage. \\
\end{itemize}


\ \ \ \ \ \ - Soit $\mathcal{A} = (Q,\delta, i, F)$ un automate DC. Le \textbf{monoïde de transitions} de $\mathcal{A}$ est le sous monoïde de $(Q^Q,*)$ engendré par les applications $\delta_a : Q \to Q \ (a \in \Sigma) $ définies par $\delta_a(q)=\delta(q,a)$ et muni de la composition. \\

\textbf{Lemme :} Le monoïde de transitions de $\mathcal{A}$ reconnait $\mathcal{L}(\mathcal{A})$. \\

\underline{Rq} : Si $L$ est reconnu par $\varphi : \Sigma^* \to M$, alors l'automate $\mathcal{A}=(M,\delta,\varphi(\epsilon),\varphi(L))$, avec $\delta(m,a) := m \cdot \varphi(a)$ reconnaît $L$.

\textbf{Théorème :} Un langage est reconnaissable par morphisme ssi il est reconnaissable par automate. \\

\underline{Rq} : On en déduit que Rec$(\Sigma^*)$ est fermé par morphisme inverse. \\

\ \ \ \ \ \ - Une relation d'équivalence $\equiv$ sur $\Sigma^*$ s'appelle \textbf{congruence} si : $u \equiv v \Rightarrow \forall x,y \ xuy \equiv xvy$. \\

\ \ \ \ \ \ - Soit $L \subseteq \Sigma^*$ et $\equiv$ une congruence sur $\Sigma^*$. $L$ est \textbf{saturé} par $\equiv$ si : $\forall u,v \in \Sigma^*, u \equiv v \Rightarrow (u\in L \Leftrightarrow v \in L)$ \\

\textbf{Théorème :} Soit $L \subseteq \Sigma^*$. $L$ est reconnaissable ssi $L$ est saturé par une congruence d'index fini. \\

\ \ \ \ \ \ - Soit $L \subseteq \Sigma^*$ , on considère $\equiv_L$ définie par $u \equiv_L v$ si $\forall x,y \in\Sigma^*, xuy \in L \Leftrightarrow xvy \in L$. On a les propriétés suivantes :
\begin{itemize}
\item $\equiv_L$ est une congruence qui sature $L$
\item  $\equiv_L$ est la plus grande congruence qui sature $L$
\item $L$ est reconnaissable ss $\equiv_L$ est d'index fini. \\
\end{itemize}

\ \ \ \ \ \ - Soit $L \subseteq \Sigma^*$, on considère $M_L = \Sigma^*/\mathord\equiv_L$ le \textbf{monoïde syntaxique} de L. On a les propriétés suivantes :
\begin{itemize}
\item $M_L$ est le monoïde des transitions de l'automate minimal de $L$.
\item $M_L$ divise (i.e est quotient d'un sous-monoïde) tout monoïde qui reconnaît $L$.
\end{itemize}

















\end{document}
