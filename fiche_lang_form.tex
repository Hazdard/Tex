\documentclass[11pt,a4paper]{article}
\textheight245mm
\textwidth170mm
\hoffset-21mm
\voffset-15mm
\parindent0pt
\usepackage[utf8]{inputenc}
\usepackage{amsmath,amsfonts,amssymb}
\usepackage{dsfont}
\usepackage{graphicx}
\usepackage{caption}
\usepackage{fancyhdr}
\pagestyle{fancy}

\renewcommand{\headrulewidth}{1pt}
\fancyhead[C]{Languages formels}
\fancyhead[L]{L3 - 2022/2023}
\fancyhead[R]{D.E.R Informatique}

\renewcommand{\footrulewidth}{1pt}
\fancyfoot[C]{\thepage} 
\fancyfoot[L]{Sacha Ben-Arous}
\fancyfoot[R]{E.N.S Paris-Saclay}

\begin{document}

\section{Définitions :}
- Un \textit{automate fini déterministe} est un 4-uplet $\mathcal{A} = (Q,\delta, i, F)$ où $Q$ est un ensemble \textit{fini} d'états, $\delta : Q \times \Sigma \to Q \ $ est la fonction de transition, $i$ l'état initial, et $F \subseteq Q$ l'ensemble des états finaux. \\

- Le \textit{language reconnu} par $\mathcal{A}$ est $\mathcal{L}(\mathcal{A}) := \{u \in \Sigma^* \ | \ \delta(i,u) \in F\}$. \\

- Un \textit{automate fini non-déterministe} est un 4-uplet $\mathcal{A} = (Q,T, i, F)$ où $Q$ est un ensemble \textit{fini} d'états, $ T \subseteq Q \times \Sigma	\times Q$ est la table de transition, $i$ l'état initial, et $F \subseteq Q$ l'ensemble des états finaux. \\

\end{document}
