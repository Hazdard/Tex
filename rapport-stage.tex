\documentclass[11pt,a4paper]{article}
\textheight245mm
\textwidth170mm
\hoffset-21mm
\voffset-15mm
\parindent0pt
\usepackage[utf8]{inputenc}
\usepackage{dsfont}
\usepackage{graphicx}
\usepackage{caption}
\usepackage{fancyhdr}
\usepackage{amsmath,amsfonts,amssymb}
\usepackage[french]{babel}
\usepackage[maxalphanames=99, maxnames=99, backend=bibtex, style=alphabetic, sorting=ynt]{biblatex}
\addbibresource{rapport-stage.bib}
\usepackage[hidelinks]{hyperref} 
\hypersetup{
  colorlinks   = true,    % Colours links instead of ugly boxes
  urlcolor     = blue,    % Colour for external hyperlinks
  linkcolor    = black,    % Colour of internal links
  citecolor    = black      % Colour of citations
}
\usepackage{zephyr}
\pagestyle{fancy}

\usepackage{array,multirow,makecell}
\setcellgapes{4pt}
\makegapedcells
\newcolumntype{R}[1]{>{\raggedleft\arraybackslash }b{#1}}
\newcolumntype{L}[1]{>{\raggedright\arraybackslash }b{#1}}
\newcolumntype{C}[1]{>{\centering\arraybackslash }b{#1}}

\renewcommand{\headrulewidth}{1pt}
\fancyhead[C]{}
\fancyhead[L]{L3 - 2022/2023}
\fancyhead[R]{D.E.R Informatique}

\renewcommand{\footrulewidth}{1pt}
\fancyfoot[C]{\thepage} 
\fancyfoot[L]{Sacha Ben-Arous}
\fancyfoot[R]{E.N.S Paris-Saclay}

\title{\textbf{Étude du problème MP-LWE}}
\date{}
\author{Sacha Ben-Arous, sous la direction d'Alice Pellet-Mary}

\begin{document}
\maketitle 
\begin{abstract}
Insérer abstract ici \\
\end{abstract}
\tableofcontents
\newpage
\section{Introduction}
\; Les réseaux euclidiens sont une construction algébrique permettant entre autres de définir des problèmes mathématiques dont la résolution algorithmique est conjecturée difficile, même pour des ordinateurs quantiques. Cela les rend donc particulièrement intéressants pour construire des protocoles sûrs en cryptographie post-quantique. \\
Deux exemples fondamentaux de problèmes sur les réseaux sont le \textit{Small Integer Solutions problem} (SIS) introduit par Ajtai en 1996, et le \textit{Learning With Errors problem} (LWE), découvert par Regev en 2005. \\
Durant mon stage, j'ai travaillé sur la variante \textit{Middle-Product Learning With Errors} (MP-LWE) du problème de Regev, initialement présenté par Roşca \textit{et al.} \cite{mplwe}
\\

\begin{defin} 
Un \textit{réseau euclidien} de $\mathbb{R}^m$ est l'ensemble des combinaisons à coefficients entiers de vecteurs linéairements indépendants $b_1, \dots, b_n$, que l'on note :
\[\mathcal{L}(b_1,\dots,b_n) := \left\{ \sum_{i=1}^n x_ib_i, x_i \in \mathbb{Z} \right\} \]

Le réseau est alors de \textit{dimension} $n$, et la famille des $(b_i)_{1\leq i \leq n}$ est appelée \textit{base} de ce réseau. \\

En notant $B:=[b_1,\dots,b_n]$ la matrice dont les colonnes sont formées par les $(b_i)_{1\leq i \leq n}$, on considérera de manière équivalente : \[\mathcal{L}(B) := \left\{ Bx, x \in \mathbb{Z}^n \right\} \]
\end{defin}

On peut alors considérer le problème algorithmique suivant, qui apparait lors de l'étude des réseaux sur lesquels se basent les protocoles utilisant LWE : 

\begin{defin} 
Soit $B$ une base d'un réseau de dimension $n$, et $\delta \in \mathbb{R}^+$. Une instance du problème \textit{Bounded Distance Decoding} est un vecteur $t \in \mathbb{R}^m$ de la forme $t=x+e$, où $x\in \mathcal{L}(B)$ et $e \leq \delta$. Le problème consiste à retrouver $x$ (ou $e$) à partir de $t$.
\end{defin}

\begin{conj}
Dans des réseaux de dimension $n$, pour $\delta$ polynomial en $n$, le problème BDD est conjecturé exponentiellement (en $n$) dur à résoudre, même sur des ordinateurs quantiques.
\end{conj}


\section{Développements}
\subsection{Preuve de correction}
On se propose tout d'abord de détailler la preuve de correction du schéma de chiffrement proposé dans \cite{mplwe} : \\

On rappelle le cadre de la preuve : $s \hookleftarrow \mathcal{U}(\mathbb{Z}^{n+d+k-1}[X])$, pour $i\leq t$ on a $a_i \hookleftarrow \mathcal{U}(\mathbb{Z}^n[X])$ ; $e_i \hookleftarrow \lfloor D_{\alpha q}\rceil[X]^{<k+d}$ et finalement $r_i \hookleftarrow \mathcal{U}(\{0,1\}^{<k+1}[X])$. On note $e_i(j)$ le $j$-ème coefficient de $e_i$. Le but est d'avoir avec bonne probabilité que : 
\[\|\mu + 2\sum_{i \leq t}r_i \odot_d e_i  \|_\infty < q/2 \] où $\mu$ est le message à chiffrer. 
 
On commence par donner une borne classique sur la distribution gaussienne : 
\begin{eqnarray*}
\mathbb{P}_{X \hookleftarrow \mathcal{N}(\sigma^2)}(|X| \geq M) &=& \frac{2}{\sigma \sqrt{2 \pi}} \int_{M}^\infty e^{-\frac{t^2}{2 \sigma^2}} \, \mathrm{d}t \\
&\leq &  \frac{2\sigma^2}{\sigma \sqrt{2 \pi}}\int_{M}^\infty \frac{t}{\sigma^2 M} e^{-\frac{t^2}{2 \sigma^2}} \, \mathrm{d}t \\
& \leq & \frac{2\sigma}{M \sqrt{2 \pi}} [-e^{-\frac{t^2}{2 \sigma^2}}]_M^\infty \\
& \leq & \frac{2\sigma}{M \sqrt{2 \pi}} e^{-\frac{M^2}{2 \sigma^2}}
\end{eqnarray*}


De plus, si on note $E := \max_{i,j} |e_i(j)|$, on a que :  \\
\begin{eqnarray*}
|(r_i \odot_d e_i)_{l\text{-eme}}| &=& |(r_i\times e_i)_{l+k\text{-eme}}| \\
&=& |\sum_{j=0}^{k+l}r_i(j)e_i(k+l-j)| \\
&\leq& (k+d)E
\end{eqnarray*}

Donc $|(\sum_{i \leq t}r_i \odot_d e_i)_{l\text{-eme}}|\leq t(k+d)E$. \\

Alors, on obtient que : 
\begin{eqnarray*}
\mathbb{P}(\|\mu + 2\sum_{i \leq t}r_i \odot_d e_i  \|_\infty \geq q/2 ) &\leq& \mathbb{P}(E \geq \frac{q}{4t(k+d)}) \\
&\leq& \mathbb{P}(\bigcup_{i,j}|e_i(j)| \geq \frac{q}{4t(k+d)}) \\
&\leq& (k+d)t \ \mathbb{P}_{X \hookleftarrow \mathcal{N}(\alpha^2 q^2)}(X \geq \frac{q}{4t(k+d)}) \\
&\leq& \frac{8t^2(k+d)^2\alpha q}{q \sqrt{2\pi}} e^{-\frac{q^2}{32\alpha^2 q^2 (k+d)^2 t^2}} \\
&\leq& \frac{8t^2(k+d)^2\alpha}{\sqrt{2\pi}} e^{-\frac{1}{2(4\alpha (k+d) t)^2}} \\
\end{eqnarray*}

De plus, les hypothèses du théorème sont que $\alpha \leq 1/(16\sqrt{\lambda t k})$ et $q \geq 16t(k+1)$. En injectant alors les ordres de grandeurs des facteurs de sécurité, on obtient que le facteur devant l'exponentielle est de la forme $o(d)$, et que l'exposant est de la forme $\frac{1}{16 \alpha^2 n^2 \log{n}^2}$. Il suffit alors de renforcer la première hypothèse, par exemple en $\alpha \leq 1/(16\lambda\sqrt{t k})$ pour finalement avoir \[\mathbb{P}(\|\mu + 2\sum_{i \leq t}r_i \odot_d e_i  \|_\infty \geq q/2 ) \leq d \ e^{- \frac{\lambda^2}{n\log{n}}}\] et un choix de $\lambda$ adapté (par exemple $\lambda = n^{\frac{2}{3}}$) permet d'obtenir la borne souhaitée.





\printbibliography[heading=bibintoc, title={Références}]
\end{document}







































