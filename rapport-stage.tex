\documentclass[11pt,a4paper]{article}
\textheight245mm
\textwidth170mm
\hoffset-21mm
\voffset-15mm
\parindent0pt
\usepackage[utf8]{inputenc}
\usepackage{amsmath,amsfonts,amssymb}
\usepackage{dsfont}
\usepackage{graphicx}
\usepackage{caption}
\usepackage{fancyhdr}
\usepackage[francais]{babel}
\usepackage[hidelinks]{hyperref} 
\hypersetup{
  colorlinks   = true,    % Colours links instead of ugly boxes
  urlcolor     = blue,    % Colour for external hyperlinks
  linkcolor    = black,    % Colour of internal links
  citecolor    = black      % Colour of citations
}
\pagestyle{fancy}

\usepackage{array,multirow,makecell}
\setcellgapes{4pt}
\makegapedcells
\newcolumntype{R}[1]{>{\raggedleft\arraybackslash }b{#1}}
\newcolumntype{L}[1]{>{\raggedright\arraybackslash }b{#1}}
\newcolumntype{C}[1]{>{\centering\arraybackslash }b{#1}}

\renewcommand{\headrulewidth}{1pt}
\fancyhead[C]{}
\fancyhead[L]{L3 - 2022/2023}
\fancyhead[R]{D.E.R Informatique}

\renewcommand{\footrulewidth}{1pt}
\fancyfoot[C]{\thepage} 
\fancyfoot[L]{Sacha Ben-Arous}
\fancyfoot[R]{E.N.S Paris-Saclay}


\title{\textbf{Étude du problème MP-LWE}}
\date{}
\author{Sacha Ben-Arous, sous la direction d'Alice Pellet-Mary}

\begin{document}
\maketitle 
\begin{abstract}
Insérer abstract ici \\
\end{abstract}
\tableofcontents
\newpage
\section{Introduction}
\ \ \ \ \ Les réseaux euclidiens sont une construction algébrique permettant entre autres de définir des problèmes mathématiques dont la résolution algorithmique est conjecturée difficile, même pour des ordinateurs quantiques.

Deux exemples fondamentaux de problèmes sont le \textit{Small Integer Solutions problem} (SIS) introduit par Ajtai en 1996, et le \textit{Learning With Errors problem} (LWE), découvert par Regev en 2009. 

Durant mon stage, j'ai travaillé sur la variante \textit{Middle-Product Learning With Errors} (MP-LWE) du problème de Regev, initialement présenté par Roşca \textit{et al.} [RSSS17]
\\

\textbf{Définition :} Un \textit{réseau euclidien} de $\mathbb{R}^m$ est l'ensemble des combinaisons à coefficients entiers de vecteurs linéairements indépendants $b_1, \dots, b_n$, que l'on note :
\[\mathcal{L}(b_1,\dots,b_n) := \left\{ \sum_{i=1}^n x_ib_i, x_i \in \mathbb{Z} \right\} \]

Le réseau est alors de \textit{dimension} $n$, et la famille des $(b_i)_{1\leq i \leq n}$ est appelée \textit{base} de ce réseau.




\section{Références}

[RSSS17] M. Rosca, A. Sakzad, D. Stehlé, and R. Steinfeld. Middle-product learning with errors. In
CRYPTO, pages 283–297. 2017.



\end{document}







































