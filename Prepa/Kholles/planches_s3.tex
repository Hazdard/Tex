\documentclass[11pt,a4paper]{article}
\textheight245mm
\textwidth170mm
\hoffset-21mm
\voffset-15mm
\parindent0pt
\usepackage[utf8]{inputenc}
\usepackage{dsfont}
\usepackage{graphicx}
\usepackage{caption}
\usepackage{fancyhdr}
\usepackage{amsmath,amsfonts,amssymb}
\usepackage[french]{babel}
\usepackage[hidelinks]{hyperref} 
\hypersetup{
  colorlinks   = true,    % Colours links instead of ugly boxes
  urlcolor     = blue,    % Colour for external hyperlinks
  linkcolor    = black,   % Colour of internal links
  citecolor    = black    % Colour of citations
}
\usepackage{../../zephyr}
\pagestyle{fancy}

\usepackage{array,multirow,makecell}
\setcellgapes{4pt}
\makegapedcells
\newcolumntype{R}[1]{>{\raggedleft\arraybackslash }b{#1}}
\newcolumntype{L}[1]{>{\raggedright\arraybackslash }b{#1}}
\newcolumntype{C}[1]{>{\centering\arraybackslash }b{#1}}

\renewcommand{\headrulewidth}{1pt}
\fancyhead[C]{Oraux blancs}
\fancyhead[L]{}
\fancyhead[R]{}

\renewcommand{\footrulewidth}{1pt}
\fancyfoot[C]{\thepage} 
\fancyfoot[L]{Sacha Ben-Arous}
\fancyfoot[R]{E.N.S Paris-Saclay}

\begin{document}



\textbf{Exercice 1 :} \\
On note : \[f : M\in M_n(\mathbb{R}) \mapsto (\text{Tr}(M),\text{Tr}(M^2),\dots,\text{Tr}(M^n))\]
On munit $ M_n(\mathbb{R})$ de sa structure euclidienne canonique.
\begin{enumerate}
\item Montrer que $f$ est différentiable en tout $M$ et expliciter $df(M)$.
\item Comparer le rang de $df(M)$ au degré du polynôme minimal de $M$.
\item Montrer que l'ensemble $\left \{ M \in M_n(\mathbb{R}), \ \chi_M = \mu_M \right \} $ est un ouvert de $M_n(\mathbb{R})$. \\
\end{enumerate}


\textbf{Exercice 2 (avec préparation) :} 
\begin{enumerate}
\item Donner un développement asymptotique à deux termes de 
\[u_n = \sum_{k=2}^n \frac{\ln{k}}{k}\]
\item À l'aide de la constante d'Euler, calculer 
\[\sum_{n=1}^{+\infty} (-1)^n\frac{\ln{n}}{n}\]
\end{enumerate}

\textbf{Exercice 3 :}
\begin{enumerate}
\item Soit $A\in GL_n(\mathbb{R})$, montrer qu'il existe $O\in \mathcal{O}_n(\mathbb{R})$ et $S \in \mathcal{S}_n(\mathbb{R})$ tels que $A=OS$. Étendre ce résultat aux matrices non-inversibles.
\item Soit $M\in M_n(\mathbb{R})$, justifier l'existence de $\displaystyle \sup_{O\in \mathcal{O}_n(\mathbb{R})} \left | Tr(OM) \right |$ et calculer sa valeur. \\
\item Montrer que l'ensemble des matrices de $M_n(\mathbb{R})$ à déterminant $> 0$ est connexe par arcs.
\end{enumerate}
~\\

\textbf{Exercice 4 (avec préparation) :}
\begin{enumerate}
\item Montrer que $GL_n(\mathbb{K})$ est dense dans $M_n(\mathbb{K})$.
\item Justifier que l'ensemble des matrices diagonalisables dans $\mathbb{C}$ est dense dans $M_n(\mathbb{C})$.
\item 
\begin{itemize}
\item[a)] Montrer que l’ensemble des matrices diagonalisables dans $\mathbb{R}$ n’est pas dense dans $M_n(\mathbb{R})$. On se contentera du cas $n=2$.
\item[b)] Montrer que $P\in \mathbb{R}[X]$ unitaire de degré $n\in \mathbb{N}$ sur $\mathbb{R}$ est scindé si et seulement si \\ $ \forall z \in \mathbb{C},|Im(z)|^n \leq |P(z)|$.
\item[c)] En déduire que l'ensemble des matrices trigonalisables dans $\mathbb{R}$ est fermé dans $M_n(\mathbb{R})$.
\item[d)]  Montrer que l’adhérence des matrices diagonalisables dans $\mathbb{R}$ est l’ensemble des matrices
trigonalisables dans $\mathbb{R}$.
\end{itemize}
\end{enumerate}
\textit{Bonus : Calculer l'adhérence des matrices de rang $r$ dans $M_n(\mathbb{K})$} \\
~\\

\textbf{Exercice 5 :}
\begin{enumerate}
\item Montrer qu'un sous-groupe additif de $\mathbb{R}$ est soit monogène soit dense.
\item Montrer que $ \lbrace \sin{n}, n \in \mathbb{Z} \rbrace $ est dense dans $[-1,1]$.
\item Montrer que $ \lbrace \sin{n}, n \in \mathbb{N} \rbrace $ est dense dans $[-1,1]$.
\end{enumerate}
\newpage

\textbf{Exercice 6 :}
Soit $(E,\|.\|)$ un e.v.n. On dit qu'une suite $(u_n)_{n \in \mathbb{N}}$ est de Cauchy si ; \[\forall \varepsilon >0, \ \exists n_0, \ \forall p,q \geq n_0,\ \| u_p - u_q \|\leq \varepsilon\]
On dit que $E$ est complet si toute suite de Cauchy est convergente.
\begin{enumerate}
\item Justifier que toute suite convergente est de Cauchy.
\item Montrer qu'une suite de Cauchy est convergente si et seulement si elle admet une suite extraite convergente.
\item Montrer que $(\mathbb{R},|.|)$ est complet. Que dire de $(\mathbb{Q},|.|)$ ?
\item Montrer qu'un e.v.n est complet si et seulement si la convergence absolue des séries entraine la convergence simple.
\end{enumerate}
~\\

\textbf{Exercice bonus :} \\
Montrer que tout sous-groupe fini des inversibles d'un corps commutatif est cyclique. \
\end{document}
