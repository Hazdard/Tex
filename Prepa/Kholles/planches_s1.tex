\documentclass[11pt,a4paper]{article}
\textheight245mm
\textwidth170mm
\hoffset-21mm
\voffset-15mm
\parindent0pt
\usepackage[utf8]{inputenc}
\usepackage{dsfont}
\usepackage{graphicx}
\usepackage{caption}
\usepackage{fancyhdr}
\usepackage{amsmath,amsfonts,amssymb}
\usepackage[french]{babel}
\usepackage[hidelinks]{hyperref} 
\hypersetup{
  colorlinks   = true,    % Colours links instead of ugly boxes
  urlcolor     = blue,    % Colour for external hyperlinks
  linkcolor    = black,   % Colour of internal links
  citecolor    = black    % Colour of citations
}
\usepackage{../../zephyr}
\pagestyle{fancy}

\usepackage{array,multirow,makecell}
\setcellgapes{4pt}
\makegapedcells
\newcolumntype{R}[1]{>{\raggedleft\arraybackslash }b{#1}}
\newcolumntype{L}[1]{>{\raggedright\arraybackslash }b{#1}}
\newcolumntype{C}[1]{>{\centering\arraybackslash }b{#1}}

\renewcommand{\headrulewidth}{1pt}
\fancyhead[C]{Oraux blancs}
\fancyhead[L]{}
\fancyhead[R]{}

\renewcommand{\footrulewidth}{1pt}
\fancyfoot[C]{\thepage} 
\fancyfoot[L]{Sacha Ben-Arous}
\fancyfoot[R]{E.N.S Paris-Saclay}

\begin{document}
\textbf{Exercice 1 :}  \\
Pour $x>1$, on note $ \zeta (x) := \displaystyle \sum_{n=1}^{+\infty} \frac{1}{n^x}$ \\

\begin{enumerate}
\item Calculer la limite de $\zeta (x)$ quand $x \to +\infty$ .
\item Déterminer l'ensemble des réels $x$ tels que $ F(x) := \displaystyle \sum_{n=2}^{+\infty} \frac{\zeta(n)}{n}x^n$ soit bien définie.
\item Montrer que $F$ est continue sur $[-1;1[$, et de classe $\mathcal{C}^1$ sur $\left ] -1;1 \right [$.
\item Donner une expression plus simple de $F(x)$. \\
\end{enumerate} 


\textbf{Exercice 2 :} \\
On note $f(z) := \displaystyle \sum_{n=0}^{+\infty} a_n z^n$ une série entière de rayon de convergence $R >0$.
\begin{enumerate}
\item Montrer que pour $0<r<R$, on a : $\displaystyle \sum_{n=0}^{+\infty}|a_n|^2r^{2n}=\frac{1}{2\pi}\int_0^{2\pi} |f(re^{i\theta})|^2\mathrm{d}\theta $
\item Que dire sur $f$ si $|f|$ admet un maximum local en 0 ?
\item On suppose maintenant que $R=+\infty$ et qu'il existe $P \in \mathbb{R}_d[X]$ tel que $|f(z)|\leq P(|z|)$ pour tout $z \in \mathbb{C}$. Montrer que $f\in \mathbb{C}_d[X]$. \\
\end{enumerate}

\textbf{Exercice 3 :} \\
 On souhaite modéliser le nombre d'arrivées de clients dans une boutique, durant un laps de temps $T$. \\
Pour $n\in \mathbb{N}$, et $s,t \in \mathbb{R}$ avec $0\leq s\leq t$, on note $A(n,s,t)$ l'évènement :
\[\text{`` Il arrive } n \text{ clients dans l'intervalle de temps} [s;t[ \text{ ''}\]

On admet qu'il existe un espace probabilisé permettant d'étudier ces événements, en supposant : \\
\begin{itemize}
\item[(H1)] Pour tout $m,n \in \mathbb{N}$, et tous réels $0\leq r \leq s \leq t$, les événements $A(m,r,s)$ et $A(n,s,t)$ sont indépendants.
\item[(H2)] La probabilité de l'événement $A(n,s,t)$ ne dépend que de $n$ et du réel $t-s$. On note : \[p_n(t) := \mathbb{P}(A(n,0,t))\]
\item[(H3)] La fonction $p_0$ est continue et $p_0(0)=1$
\item[(H4)] Pour tout $t\in \mathbb{R}_+$, on a \[\sum_{n=0}^{+\infty}p_n(t)=1\]
\item[(H5)] On a le développement asymptotique quand $t \to 0^+$ : \[1-p_0(t)-p_1(t) = o(p_1(t))\]
\end{itemize}

Cette dernière hypothèse signifie que, durant un laps de temps minime, la probabilité d'arrivée d'au moins deux clients est négligeable devant la probabilité d'arrivée d'un seul client.

\begin{enumerate}
\item Justifier que la fonction $p_0$ est décroissante, et que : \[\forall s,t \in \mathbb{R}_+, \ p_0(s+t)=p_0(s)p_0(t)\]
\item Montrer que $p_0$ est à valeurs strictement positives et qu'il existe un réel $\lambda \geq 0$ vérifiant \[\forall t \in \mathbb{R}_+, \ p_0(t) = e^{-\lambda t}\]
\textit{Indication : On pourra considérer la fonction $f(t) = \ln{(p_0(t))}$ }.
\item Justifier que, quand  $t \to 0^+$ : \[p_1(t) = \lambda t + o(t) \ \ \  \text{ et } \ \ \ \forall n \geq 2, \ p_n(t) = o(t)\]
\item Pour $n\in \mathbb{N}^*$, montrer : \[\forall s,t \geq 0, \  p_n(s+t)= \sum_{k=0}^np_k(s)p_{n-k}(t)\] En déduire que $p_n$ est dérivable et que \[\forall t \geq 0, \ p_n'(t)=\lambda(p_{n-1}(t) - p_n(t))\]
\item En considérant $q_n(t) := e^{\lambda t} p_n(t)$, obtenir l'expression de $p_n(t)$.
\item On note $X$ la variable aléatoire déterminant le nombre de clients arrivant durant le laps de temps $T >0$. Déterminer la loi de $X$. Quelle est l'interprétation du paramètre $\lambda$ ? \\
\end{enumerate}

\textbf{Exercice 4 :}
\begin{enumerate}
\item Montrer que $GL_n(\mathbb{K})$ est dense dans $M_n(\mathbb{K})$.
\item Montrer que l'ensemble des matrices diagonalisables dans $\mathbb{C}$ est dense dans $M_n(\mathbb{C})$.
\item 
\begin{itemize}
\item[a)] Montrer que l’ensemble des matrices diagonalisables dans $\mathbb{R}$ n’est pas dense dans $M_n(\mathbb{R})$.
\item[b)] Montrer que $P\in \mathbb{R}[X]$ unitaire de degré $n\in \mathbb{N}$ sur $\mathbb{R}$ est scindé si et seulement si \\ $ \forall z \in \mathbb{C},|Im(z)|^n \leq |P(z)|$.
\item[c)] En déduire que l'ensemble des matrices trigonalisables dans $\mathbb{R}$ est fermé dans $M_n(\mathbb{R})$.
\item[d)]  Montrer que l’adhérence des matrices diagonalisables dans $\mathbb{R}$ est l’ensemble des matrices
trigonalisables dans $\mathbb{R}$.
\end{itemize}
\end{enumerate}
\textit{Bonus : Calculer l'adhérence des matrices de rang $r$ dans $M_n(\mathbb{K})$} \\

\textbf{Exercice 5 :}
\begin{enumerate}
\item Soit $A\in GL_n(\mathbb{R})$, montrer qu'il existe $O\in \mathcal{O}_n(\mathbb{R})$ et $S \in \mathcal{S}_n(\mathbb{R})$ tels que $A=OS$. Étendre ce résultat aux matrices non-inversibles.
\item Montrer que l'ensemble des matrices de $M_n(\mathbb{R})$ à déterminant $> 0$ est connexe par arcs.
\item Soit $M\in M_n(\mathbb{R})$, justifier l'existence de $\displaystyle \sup_{O\in \mathcal{O}_n(\mathbb{R})} \left | Tr(OM) \right |$ et calculer sa valeur. \\
\end{enumerate}

\textbf{Exercice 6 :} \\
On note : \[f : M\in M_n(\mathbb{R}) \mapsto (\text{Tr}(M),\text{Tr}(M^2),\dots,\text{Tr}(M^n))\]
On munit $ M_n(\mathbb{R})$ de sa structure euclidienne canonique.
\begin{enumerate}
\item Montrer que $f$ est différentiable en tout $M$ et expliciter $df(M)$.
\item Comparer le rang de $df(M)$ au degré du polynôme minimal de $M$.
\item Montrer que l'ensemble $\left \{ M \in M_n(\mathbb{R}), \ \chi_M = \mu_M \right \} $ est un ouvert de $M_n(\mathbb{R})$. \\
\end{enumerate}

\textbf{Exercice 7 :} \\
Montrer que tout sous-groupe fini des inversibles d'un corps commutatif est cyclique.
\end{document}
