\documentclass[11pt,a4paper]{article}
\textheight245mm
\textwidth170mm
\hoffset-21mm
\voffset-15mm
\parindent0pt
\usepackage[utf8]{inputenc}
\usepackage{dsfont}
\usepackage{graphicx}
\usepackage{caption}
\usepackage{fancyhdr}
\usepackage{amsmath,amsfonts,amssymb}
\usepackage[french]{babel}
\usepackage[hidelinks]{hyperref} 
\hypersetup{
  colorlinks   = true,    % Colours links instead of ugly boxes
  urlcolor     = blue,    % Colour for external hyperlinks
  linkcolor    = black,    % Colour of internal links
  citecolor    = black      % Colour of citations
}
\usepackage{zephyr}
\pagestyle{fancy}

\usepackage{array,multirow,makecell}
\setcellgapes{4pt}
\makegapedcells
\newcolumntype{R}[1]{>{\raggedleft\arraybackslash }b{#1}}
\newcolumntype{L}[1]{>{\raggedright\arraybackslash }b{#1}}
\newcolumntype{C}[1]{>{\centering\arraybackslash }b{#1}}

\renewcommand{\headrulewidth}{1pt}
\fancyhead[C]{DM 1}
\fancyhead[L]{L3 - 2023/2024}
\fancyhead[R]{Intégration-Probabilités}

\renewcommand{\footrulewidth}{1pt}
\fancyfoot[C]{\thepage} 
\fancyfoot[L]{Sacha Ben-Arous}
\fancyfoot[R]{E.N.S Paris-Saclay}

\begin{document}
\textbf{Exercice 10} : \\

1) Par définition de $f$, on a : $\forall K \geq 0, \ \displaystyle \lim_{n\to \infty}  \sum_{k = 0}^K f_n(k)=\sum^K_{k = 0} f(k)$. \\

 Supposons que la somme $\displaystyle \sum_{k \geq 0} f(k)$ soit finie. Alors, on peut appliquer le théorème de double limite dans sa version discrète, car toutes les sommes considérées sont finies, ce qui donne : 
 \[\lim_{K \to \infty} \lim_{n\to \infty}  \sum_{k = 0}^K f_n(k) = \lim_{n\to \infty}  \lim_{K \to \infty} \sum_{k = 0}^K f_n(k)\]
 
 i.e : \[\displaystyle \sum_{k \geq 0} f(k) =  \lim_{n\to \infty} \sum_{k \geq 0} f_n(k) \]
 
 Dans le cas où $\displaystyle \sum_{k \geq 0} f(k)$ est infinie, on a pour tout $K$ positif : \[ \displaystyle \lim_{n\to \infty} \sum_{k \geq 0} f_n(k) \geq \lim_{n\to \infty} \sum_{k = 0}^K f_n(k) = \sum_{k = 0}^K f(k) \] L'existence de la limite (dans $\overline{\mathbb{R}} $) est due à la croissance des $(f_n)_{n\in \mathbb{N}}$, et l'inégalité provient de la positivité de cette suite. Or cette inégalité est vraie pour tout $K$, dont le majorant ne dépend pas, et par hypothèse le minorant tend vers $+\infty$ avec $K$. Ainsi, en passant à la limite, on a bien :
\[ \lim_{n\to \infty} \sum_{k \geq 0} f_n(k) = + \infty\]

2) On note $\displaystyle \tilde{f}_n(k) := \inf_{m\geq n} f_m(k)$. Alors d'une part $\displaystyle \varliminf_{n\to \infty}{f_n(k)} = \lim_{n \to \infty} \tilde{f}_n(k)$, et d'autre part la suite des $(\tilde{f}_n)_{n\in \mathbb{N}}$ est clairement croissante (et positive), on peut donc lui appliquer le résultat de la question précédente, ce qui donne : 
\begin{equation}
\sum_{k\geq 0} \varliminf_{n\to \infty}{f_n(k)} = \lim_{n \to \infty} \sum_{k \geq 0} \tilde{f}_n(k) 
\end{equation}
Or, pour $K\geq 0$, on a : \[\sum_{k = 0}^K \tilde{f}_n(k) \leq \inf_{m\geq n}\sum_{k = 0}^K f_m(k) \leq \inf_{m\geq n}\sum_{k \geq 0} f_m(k) \leq\varliminf_{n\to \infty} \sum_{k\geq 0} {f_n(k)}  \]
Les infimum et sommes infinis existent dans  $\overline{\mathbb{R}} $ grâce à la positivité des $(f_n)_{n\in \mathbb{N}}$. La première inégalité est une propriété classique des infimum (cf. exercice 8), la seconde est due à la positivité des $(f_n)_{n\in \mathbb{N}}$, et la troisième est donnée par la croissance d'une suite d'infimum. Finalement, le majorant étant indépendant du $K$ choisi, on peut directement injecter l'ingalité dans (1) en passant à la limite et ainsi obtenir le résultat voulu : 
\[\sum_{k\geq 0} \varliminf_{n\to \infty}{f_n(k)} \leq \varliminf_{n\to \infty} \sum_{k\geq 0} {f_n(k)}  \]
\qed

\end{document}
