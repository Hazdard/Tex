%====================TodoInit====================%

% Importer le .ini dans ~/.config/xm1/
% Mettre des users tags pour les commandes récurrentes.
% Utiliser Customize Completion pour rajouter les begin des nouveaux theoremes (les ends sont mis automatiquement) et leurs versions star ; les alias en fin de fichier ; la coloration du texte ;
% Rajouter des raccourcis clavier pour : mathbb, mathcal, et autres ; la coloration du texte ;


%====================Roadmap====================%

% Rajouter des alias utiles

% Liens utiles :

%https://cedricvillani.org/sites/dev/files/old_images/2012/10/ecriture.pdf

%https://www.overleaf.com/learn/latex/Theorems_and_proofs
  
%https://tex.stackexchange.com/questions/391443/new-theorem-environment-with-manual-theorem-number

%https://tex.stackexchange.com/questions/234676/new-theorem-style

%====================AppelPackages====================%

\usepackage[french]{babel}
\usepackage[T1]{fontenc}
\usepackage[utf8]{inputenc}
% Une page A4 fait 210mm de large pour 297mm de haut.
\usepackage[a4paper,total={180mm,275mm},left=15mm,top=10mm,includeheadfoot]{geometry}
\usepackage{graphicx,amsmath,amsfonts,amssymb,amsthm,fancyhdr,caption,dsfont,mathrsfs}
\usepackage[hidelinks,pdfstartview=FitH,bookmarksopen=false,pdfpagemode=UseNone,pdftoolbar=false,pdfmenubar=false]{hyperref}
\usepackage{cleveref}
\pagestyle{fancy}

%====================CustomStyles====================%

\newtheoremstyle{bourbaki_it}%            % Name
  {10pt}%                                 % Space above
  {10pt}%                                 % Space below
  {\itshape}%                             % Body font
  {17pt}%                                 % Indent amount
  {\scshape}%                             % Theorem head font
  {.~---}%                                % Punctuation after theorem head
  { }%                                    % Space after theorem head, ' ', or \newline
  {}%                                     % Theorem head spec (can be left empty, meaning `normal')

\newtheoremstyle{bourbaki_plain}%         % Name
  {10pt}%                                 % Space above
  {10pt}%                                 % Space below
  {}%                                     % Body font
  {17pt}%                                 % Indent amount
  {\scshape}%                             % Theorem head font
  {.}%                                    % Punctuation after theorem head
  { }%                                    % Space after theorem head, ' ', or \newline
  {}%                                     % Theorem head spec (can be left empty, meaning `normal')
  
%====================CustomTheorems====================%

\renewcommand{\thesection}{\Roman{section}} 

\renewenvironment{proof}[1][\proofname]{{\scshape #1. }}{~\\ \qed}

\numberwithin{equation}{section}  %Numero des equations en fonction de la section
\renewcommand{\theequation}{\arabic{section}.\arabic{equation}}

\theoremstyle{bourbaki_plain}

\newtheorem{defin}{Definition}[section]
\renewcommand{\thedefin}{\thesection-\arabic{defin}}
\newtheorem{definstar}{Definition}[section]
\renewcommand{\thedefinstar}{\arabic{definstar}}

\newtheorem*{rmq}{Remark}

\theoremstyle{bourbaki_it}

\newtheorem{thm}{Theorem}[section]
\renewcommand{\thethm}{\thesection-\arabic{thm}} %Met un - a la place du .
\newtheorem{thmstar}{Theorem}[section]
\renewcommand{\thethmstar}{\arabic{thmstar}} % Retire le tirer pour les sections * sans numéro

\newtheorem{cor}[thm]{Corollary}
\renewcommand{\thecor}{\thesection-\arabic{cor}}
\newtheorem{corstar}[thmstar]{Corollary}
\renewcommand{\thecorstar}{\arabic{corstar}}

\newtheorem{lemma}[thm]{Lemma}
\renewcommand{\thelemma}{\thesection-\arabic{lemma}}
\newtheorem{lemmastar}[thmstar]{Lemma}
\renewcommand{\thelemmastar}{\arabic{lemmastar}}

\newtheorem{prop}[thm]{Proposition}
\renewcommand{\theprop}{\thesection-\arabic{prop}}
\newtheorem{propstar}[thmstar]{Proposition}
\renewcommand{\thepropstar}{\arabic{propstar}}


\newtheorem{mtheoreminner}{Theorem}
\newenvironment{mthm}[1]{%
  \renewcommand\themtheoreminner{#1}%
  \mtheoreminner
}{\endmtheoreminner}

%% Exemple :

%\begin{mthm}{A}[title]\label{foo}
% This is a theorem.
%\end{mthm}

%====================Misc====================%

\hypersetup{
  colorlinks   = true,    % Colours links instead of ugly boxes
  urlcolor     = blue,    % Colour for external hyperlinks
  linkcolor    = black,   % Colour of internal links
  citecolor    = black    % Colour of citations
}

%Remplace proof par preuve
\addto\captionsfrench{\renewcommand\proofname{Proof}}

% Configuration pour rendre le texte entier cliquable dans une référence
\newcommand{\fcref}[1]{\hyperref[#1]{\cref*{#1}}}

% Écrire des limites proprement
\newcommand{\Lim}[1]{\raisebox{0.5ex}{\scalebox{0.8}{$\displaystyle \lim_{#1}\;$}}}
% Sinon \lim\limits_{x \to y}

% Pour écrire des chiffres romains
\newcommand*{\rom}[1]{\expandafter\@slowromancap\romannumeral #1@}

%Majuscules pour cleveref
\crefname{defin}{Definition}{Definitions}
\crefname{rmq}{Remark}{Remarks} 
\crefname{thm}{Theorem}{Theorems}
\crefname{cor}{Corollary}{Corollaries}
\crefname{lemma}{Lemma}{Lemmas}
\crefname{prop}{Proposition}{Propositions}
\crefname{mtheoreminner}{Theorem}{Theorems} 

%====================Shortcuts====================%

\def\C{\mathbb C}
\def\R{\mathbb R}
\def\Q{\mathbb Q}
\def\Z{\mathbb Z}
\def\N{\mathbb N}
\def\T{\mathbb T}
\def\P{\mathbb P}
\def\setbar{\ \middle| \ }
